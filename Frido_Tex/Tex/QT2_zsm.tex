\documentclass[10pt,a4paper,notitlepage]{scrartcl}

\usepackage[utf8]{inputenc}
\usepackage{amsmath}
\usepackage{amsfonts}
\usepackage{amssymb}
\usepackage{graphicx}
\usepackage{latexsym}
\usepackage{xfrac}
\usepackage{tikz}
\usepackage{bbold}
\usepackage{cancel}
\usepackage[left=3cm,right=2cm,top=1cm,bottom=1.2cm]{geometry}

\begin{document}
\section{Ein-Teilchen}
\subsection*{Lorentzgruppe}

\subsection*{Generatoren}

\begin{align*} 
    k_x = i \begin{pmatrix}
        0 & 1 & 0 & 0 \\
        1 & 0 & 0 & 0 \\
        0 & 0 & 0 & 0 \\
        0 & 0 & 0 & 0 \\
    \end{pmatrix} 
    &&
    k_y = i \begin{pmatrix}
        0 & 0 & 1 & 0 \\
        0 & 0 & 0 & 0 \\
        1 & 0 & 0 & 0 \\
        0 & 0 & 0 & 0 \\
    \end{pmatrix}&
    &
     k_z = i \begin{pmatrix}
        0 & 0 & 0 & 1 \\
        0 & 0 & 0 & 0 \\
        0 & 0 & 0 & 0 \\
        1 & 0 & 0 & 0 \\
    \end{pmatrix}\\
     l_x = i \begin{pmatrix}
        0 & 0 & 0 & 0 \\
        0 & 0 & 0 & 0 \\
        0 & 0 & 0 & -1 \\
        0 & 0 & 1 & 0 \\
    \end{pmatrix} 
    &&
    l_y = i \begin{pmatrix}
        0 & 0 & 0 & 0 \\
        0 & 0 & 0 & 1 \\
        0 & 0 & 0 & 0 \\
        0 & -1 & 0 & 0 \\
    \end{pmatrix}&
    &
     l_z = i \begin{pmatrix}
        0 & 0 & 0 & 0 \\
        0 & 0 & -1 & 0 \\
        0 & 1 & 0 & 0 \\
        0 & 0 & 0 & 0 \\
    \end{pmatrix}\\
    [l_i, l_j] = \epsilon_{ijk}l_k &&
    [k_x, k_j] = -i\epsilon_{ijk}l_k &&
    [l_i, k_j] = i\epsilon_{ijk}k_k\\
\end{align*}

\subsection*{Lorentztransform}

\begin{align*}
    \Lambda^\mu_\nu &= \delta^\mu_\nu + \omega^\mu_\nu\\
    \text{or } S(\Lambda) &:=  \mathbb{1} - \frac{i}{2} \omega^{\mu\nu}L_{\mu\nu})\\
    &\omega \ldots \text{antisymmetrisch}
\end{align*}

\subsection*{Spinoren}

\begin{equation}
    \text{Drehung von Spinorlsg:   } S(R_i(\theta)) = \text{e}^{-i\theta S_i}
\end{equation}

\begin{align*} 
    S_{\mu\nu} = \frac{i}{4}[\gamma_\mu, \gamma_\nu] &&
    S^\dagger_{\mu\nu} = \gamma^0 S_{\mu\nu}\gamma^0 &&
    S^1(\Lambda)=\gamma^0 S^\dagger(\Lambda)\gamma^0 = 1 + \frac{i}{2}\omega^{\mu\nu} S_{\mu\nu}\\
\end{align*}

\subsection*{Spinoridentitäten unter Lorentztransform}
\begin{align*}
    \Psi &\mapsto S(\Lambda) \Psi\\
    \overline{\Psi} &\mapsto \overline{\Psi}S^{-1}(\Lambda)\\
    \overline{\Psi}\Psi &\mapsto \overline{\Psi}\Psi\\
    \overline{\Psi}\gamma^\mu\Psi &\mapsto \Lambda^\mu_\nu \overline{\Psi}\gamma^\Psi\\
    S^{-1}(\Lambda)\gamma^\mu S(\Lambda) &\mapsto \Lambda^\mu_\nu \gamma^\nu\\
\end{align*}

\subsection*{Matrixidentitäten}
\begin{align*}
    &\textbf{Gamma}&&\gamma^{\mu\ast} = \gamma^2\gamma^\mu\gamma^2 && \gamma^{i\dagger} = \gamma^0\gamma^i\gamma^0i \\
    &&&\{\gamma^\mu, \gamma^\nu\} = 2 g^{\mu\nu}\mathbb{1} &&\gamma^i\gamma^i = - \mathbb{1}
    &&\text{ with } i \in \{1,2,3\} \\    
    &\textbf{Sigma}&&\sigma^i\sigma^j = \delta_{i,j}\mathbb{1} + i \sum_{k=1}^3 \epsilon_{ijk}\sigma^k &&
    [\sigma_j, \sigma_k] = 2i\epsilon_{jkl}\sigma_l &&
    \{\sigma_j, \sigma_k\} = 2 \delta_{jk} \mathbb{1}\\
\end{align*}

\subsection*{Relativistisch}
\begin{align*}
    &\textbf{Klein-Gordon-Gleichung:} && \Box\Phi(x) + m^2\Phi(x) = 0 \\
    &\textbf{4-Stromdichte:}&& j^\mu = \frac{i}{2m}[\Psi^\ast\partial^\mu\Psi - \Psi\partial^\mu\Psi^\ast]
    \qquad j^0 = \rho  \qquad j^i = \vec{j}\\
    &\textbf{Dirac-Gleichung:} && (i\partial_\mu\gamma^\mu-m)\psi = 0 = (i\cancel{\partial}-m)\Psi = (p - m) \Psi\\
    &\text{Ansatz:} && \Psi(x) = \omega(p) \text{e}^{\mp i p_\mu x^\mu} \to (\pm \cancel{p}-m)\omega{p}=0\\
    &+\cancel{p} - m\textbb{1} && = E\gamma^0-p_x\gamma^1 -p_y\gamma^2-p_z\gamma^3 = 
    \begin{pmatrix}
        E-m           &     0         &       -p_z        & -p_x + i p_y \\
        0             &     E-m       &   p_x - i p_y     & p_z \\
        p_z           & p_x - i p_y   &      -E-m         & 0 \\
        p_x + i p_y   &     -p_z      &     0             & -E-m \\
    \end{pmatrix}\\
    &&&\text{mit} \begin{cases}
                                 \text{Teilchenspinoren} u \\
                                 \text{Antiteilchenspinoren} v \\
    \end{cases} \\
\end{align*}
\begin{align*}
    &\text{LSG: } && u_1 = N \begin{pmatrix} 1 \\ 0 \\ \frac{p_z}{E+m} \\ \frac{p_x+ip_y}{E+m}\end{pmatrix}
    && u_2 = N \begin{pmatrix} 0 \\ 1 \\ \frac{p_x-ip_y}{E+m} \\ \frac{-p_z}{E+m}\end{pmatrix}
    && \text{mit } N = \frac{1}{\sqrt(E+m) \to u\overline{u} = 2E}\\
    &&& v_1 = N \begin{pmatrix} \frac{p_z}{E+m} \\ \frac{p_x+ip_y}{E+m} \\1 \\0 \end{pmatrix}
    && v_2 = N \begin{pmatrix} \frac{p_x-ip_y}{E+m} \\ \frac{-p_z}{E+m} \\ 0 \\ 1\end{pmatrix}
    &&\\
\end{align*}
\begin{align*}
    &\textbf{Dirac-Hamiltonian:} && (\gamma^0 \text{v. links multiplizieren}) &&\\
    & && i\partial_t \psi = (-i\gamma^0\gamma^i\partial_i + m \gamma^0)\psi = \hat H_D \psi\\
    & \textbf{Electronmagn. Eichinvarianz:}\\
    &&&A^\mu(x) \to A^\mu(x) + \partial^\mu\theta(x) \qquad \psi(x) \to \text{e}^ie\theta(x)\psi(x)\\
    &&&D^\mu \psi := (\partial^\mu-ieA^\mu)\psi \qquad P^\mu\psi \to \text{e}^ie\theta(x)D\mu \psi\\
\end{align*}

\section*{Viel-Teilchen}
\subsection*{Symetrisierungsoperator}
\begin{align*}
    & S_N^\pm := \frac{1}{N!}\sum_\mathcal{P}(\pm)^p \mathcal{P} && \text{mit } \mathcal{P} =
    \Pi \mathcal{P} \text{ Alle Permutationen}\\
    & hermitisch && S_N^{\pm}S_N^{\pm}=S_N^{\pm}\\
    & [P_{ij}, S_N] = 0 && P_{ij}S_N^{\pm} = \pm S_N^{\pm} \\
    & [P_{ij},\hat A_N]=0 \quad \forall sinnvolle \hat A_N\\
\end{align*}

\subsection*{Fock-Raum:}
\begin{align*}
   &\text{Bosonen:} \mathcal{H}_0 \oplus \mathcal{H}_1 \oplus \mathcal{H}_2^\pm \oplus \mathcal{H}_3^\pm\ldots
   && \text{Fermionen:} \mathcal{H}_0 \oplus \mathcal{H}_1 \oplus \mathcal{H}_2^\mp \oplus \mathcal{H}_3^\mp\ldots\\
\end{align*}
\end{document}
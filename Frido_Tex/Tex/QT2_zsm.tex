\documentclass[10pt,a4paper,notitlepage]{scrartcl}

\usepackage[utf8]{inputenc}
\usepackage{amsmath}
\usepackage{amsfonts}
\usepackage{amssymb}
\usepackage{graphicx}
\usepackage{latexsym}
\usepackage{xfrac}
\usepackage{physics}
\usepackage{tikz}
\usepackage{bbold}
\usepackage{cancel}
\usepackage[left=3cm,right=2cm,top=1cm,bottom=1.2cm]{geometry}

\newcommand{\dif}{\mathrm{d}}

\begin{document}
\section{Ein-Teilchen}
\subsection*{Lorentzgruppe}

\subsection*{Generatoren}

\begin{align*} 
    k_x = i \begin{pmatrix}
        0 & 1 & 0 & 0 \\
        1 & 0 & 0 & 0 \\
        0 & 0 & 0 & 0 \\
        0 & 0 & 0 & 0 \\
    \end{pmatrix} 
    &&
    k_y = i \begin{pmatrix}
        0 & 0 & 1 & 0 \\
        0 & 0 & 0 & 0 \\
        1 & 0 & 0 & 0 \\
        0 & 0 & 0 & 0 \\
    \end{pmatrix}&
    &
     k_z = i \begin{pmatrix}
        0 & 0 & 0 & 1 \\
        0 & 0 & 0 & 0 \\
        0 & 0 & 0 & 0 \\
        1 & 0 & 0 & 0 \\
    \end{pmatrix}\\
     l_x = i \begin{pmatrix}
        0 & 0 & 0 & 0 \\
        0 & 0 & 0 & 0 \\
        0 & 0 & 0 & -1 \\
        0 & 0 & 1 & 0 \\
    \end{pmatrix} 
    &&
    l_y = i \begin{pmatrix}
        0 & 0 & 0 & 0 \\
        0 & 0 & 0 & 1 \\
        0 & 0 & 0 & 0 \\
        0 & -1 & 0 & 0 \\
    \end{pmatrix}&
    &
     l_z = i \begin{pmatrix}
        0 & 0 & 0 & 0 \\
        0 & 0 & -1 & 0 \\
        0 & 1 & 0 & 0 \\
        0 & 0 & 0 & 0 \\
    \end{pmatrix}\\
    [l_i, l_j] = \epsilon_{ijk}l_k &&
    [k_x, k_j] = -i\epsilon_{ijk}l_k &&
    [l_i, k_j] = i\epsilon_{ijk}k_k\\
\end{align*}

\subsection*{Lorentztransform}

\begin{align*}
    \Lambda^\mu_\nu &= \delta^\mu_\nu + \omega^\mu_\nu\\
    \text{or } S(\Lambda) &:=  \mathbb{1} - \frac{i}{2} \omega^{\mu\nu}L_{\mu\nu})\\
    &\omega \ldots \text{antisymmetrisch}
\end{align*}

\subsection*{Spinoren}

\begin{equation}
    \text{Drehung von Spinorlsg:   } S(R_i(\theta)) = \text{e}^{-i\theta S_i}
\end{equation}

\begin{align*} 
    S_{\mu\nu} = \frac{i}{4}[\gamma_\mu, \gamma_\nu] &&
    S^\dagger_{\mu\nu} = \gamma^0 S_{\mu\nu}\gamma^0  &&
    S^{-1}(\Lambda)=\gamma^0 S^\dagger(\Lambda)\gamma^0 = 1 + \frac{i}{2}\omega^{\mu\nu} S_{\mu\nu}\\
\end{align*}

\subsection*{Spinoridentitäten unter Lorentztransform}
\begin{align*}
    \Psi &\mapsto S(\Lambda) \Psi\\
    \overline{\Psi} &\mapsto \overline{\Psi}S^{-1}(\Lambda)\\
    \overline{\Psi}\Psi &\mapsto \overline{\Psi}\Psi\\
    \overline{\Psi}\gamma^\mu\Psi &\mapsto \Lambda^\mu_\nu \overline{\Psi}\gamma^\Psi\\
    S^{-1}(\Lambda)\gamma^\mu S(\Lambda) &\mapsto \Lambda^\mu_\nu \gamma^\nu\\
\end{align*}

\subsection*{Matrixidentitäten}
\begin{align*}
    &\textbf{Gamma}&&\gamma^{\mu\ast} = \gamma^2\gamma^\mu\gamma^2 && \gamma^{i\dagger} = \gamma^0\gamma^i\gamma^0i \\
    &&&\{\gamma^\mu, \gamma^\nu\} = 2 g^{\mu\nu}\mathbb{1} &&\gamma^i\gamma^i = - \mathbb{1}
    &&\text{ with } i \in \{1,2,3\} \\    
    &\textbf{Sigma}&&\sigma^i\sigma^j = \delta_{i,j}\mathbb{1} + i \sum_{k=1}^3 \epsilon_{ijk}\sigma^k &&
    [\sigma_j, \sigma_k] = 2i\epsilon_{jkl}\sigma_l &&
    \{\sigma_j, \sigma_k\} = 2 \delta_{jk} \mathbb{1}\\
\end{align*}

\subsection*{Relativistisch}
\begin{align*}
    &\textbf{Klein-Gordon-Gleichung:} && \Box\Phi(x) + m^2\Phi(x) = 0 \\
    &\textbf{4-Stromdichte:}&& j^\mu = \frac{i}{2m}[\Psi^\ast\partial^\mu\Psi - \Psi\partial^\mu\Psi^\ast]
    \qquad j^0 = \rho  \qquad j^i = \vec{j}\\
    &\textbf{Dirac-Gleichung:} && (i\partial_\mu\gamma^\mu-m)\psi = 0 = (i\cancel{\partial}-m)\Psi = (p - m) \Psi\\
    &\text{Ansatz:} && \Psi(x) = \omega(p) \text{e}^{\mp i p_\mu x^\mu} \to (\pm \cancel{p}-m)\omega{p}=0\\
    &+\cancel{p} - m\textbb{1} && = E\gamma^0-p_x\gamma^1 -p_y\gamma^2-p_z\gamma^3 = 
    \begin{pmatrix}
        E-m           &     0         &       -p_z        & -p_x + i p_y \\
        0             &     E-m       &   p_x - i p_y     & p_z \\
        p_z           & p_x - i p_y   &      -E-m         & 0 \\
        p_x + i p_y   &     -p_z      &     0             & -E-m \\
    \end{pmatrix}\\
    &&&\text{mit} \begin{cases}
                                 \text{Teilchenspinoren}\quad u \\
                                 \text{Antiteilchenspinoren}\quad v \\
    \end{cases} \\
\end{align*}
\begin{align*}
    &\text{LSG: } && u_1 = N \begin{pmatrix} 1 \\ 0 \\ \frac{p_z}{E+m} \\ \frac{p_x+ip_y}{E+m}\end{pmatrix}
    && u_2 = N \begin{pmatrix} 0 \\ 1 \\ \frac{p_x-ip_y}{E+m} \\ \frac{-p_z}{E+m}\end{pmatrix}
    && \text{mit } N = \sqrt{E+m} \to u\overline{u} = 2E\\
    &&& v_1 = N \begin{pmatrix} \frac{p_z}{E+m} \\ \frac{p_x+ip_y}{E+m} \\1 \\0 \end{pmatrix}
    && v_2 = N \begin{pmatrix} \frac{p_x-ip_y}{E+m} \\ \frac{-p_z}{E+m} \\ 0 \\ 1\end{pmatrix}
    &&\\
\end{align*}
\begin{align*}
    &\textbf{Dirac-Hamiltonian:} && (\gamma^0 \text{v. links multiplizieren}) &&\\
    & && i\partial_t \psi = (-i\gamma^0\gamma^i\partial_i + m \gamma^0)\psi = \hat H_D \psi\\
    & \textbf{Electronmagn. Eichinvarianz:}\\
    &&&A^\mu(x) \to A^\mu(x) + \partial^\mu\theta(x) \qquad \psi(x) \to \text{e}^ie\theta(x)\psi(x)\\
    &&&D^\mu \psi := (\partial^\mu-ieA^\mu)\psi \qquad P^\mu\psi \to \text{e}^ie\theta(x)D\mu \psi\\
\end{align*}

Sinnvolle Rechenregel: $(\vec{\sigma} \vec{A}) (\vec{\sigma} \vec{B}) = \vec{A} \vec{B} + i \vec{\sigma} (\vec{A} \times \vec{B})$

\section*{Viel-Teilchen}
\subsection*{Symetrisierungsoperator}
\begin{align*}
    & S_N^\pm := \frac{1}{N!}\sum_\mathcal{P}(\pm)^p \mathcal{P} && \text{mit } \mathcal{P} =
    \Pi \mathcal{P} \text{ Alle Permutationen}\\
    & hermitisch && S_N^{\pm}S_N^{\pm}=S_N^{\pm}\\
    & [P_{ij}, S_N] = 0 && P_{ij}S_N^{\pm} = \pm S_N^{\pm} \\
    & [P_{ij},\hat A_N]=0 \quad \forall sinnvolle \hat A_N\\
\end{align*}

\subsection*{Fock-Raum:}
\begin{align*}
   &\text{Bosonen:} \mathcal{H}_0 \oplus \mathcal{H}_1 \oplus \mathcal{H}_2^\pm \oplus \mathcal{H}_3^\pm\ldots
   && \text{Fermionen:} \mathcal{H}_0 \oplus \mathcal{H}_1 \oplus \mathcal{H}_2^\mp \oplus \mathcal{H}_3^\mp\ldots\\
\end{align*}

\subsection*{Erzeugungs- und Vernichtungsoperatoren}

Bosonen: $[a_n, a_m^\dagger] = \delta_{n m}$, $[a_n^\dagger, a_m^\dagger] = 0$, $[a_n, a_m] = 0$.

Fermionen: $\{a_n, a_m^\dagger\} = \delta_{n m}$, $\{a_n^\dagger, a_m^\dagger\} = 0$, $\{a_n, a_m\} = 0$.

In Teilchenzahldarstellung: 
\begin{align*}
    a_r^\dagger \ket{\ldots n_r \ldots}^{(+)} &= \sqrt{n_r + 1} \ket{\ldots n_r+1 \ldots}^{(+)} \\
    a_r \ket{\ldots n_r \ldots}^{(+)} &= \sqrt{n_r} \ket{\ldots n_r-1 \ldots}^{(+)} \\
    a_r^\dagger \ket{\ldots n_r \ldots}^{(-)} &= (-1)^{N_r} \delta_{n_r, 0} \ket{\ldots n_r+1 \ldots}^{(-)} \\
    a_r \ket{\ldots n_r \ldots}^{(-)} &= (-1)^{N_r} \delta_{n_r, 1}  \ket{\ldots n_r-1 \ldots}^{(-)}
\end{align*}

Besetzungszahloperator: $n_m = a_m^\dagger, a_m$

Teilchenzahloperator: $N = \sum_m n_m$ mit den Vertauschungsrelationen $[N, a_j^\dagger] = a_j^\dagger$, $[N, a_j] = -a_j$, $[N, H] = 0$.

\section*{Streuung}

\begin{align*}
    &\text{Ansatz}& \Psi &= e^{i k x} + f(\theta, \phi) \frac{e^{i k r}}{r}
\end{align*}
löst die Schrödingergleichung $H\Psi = E\Psi$ mit $E = \frac{\hbar^2 k^2}{2 m}$.
\begin{align*}
    &\text{Diff. Wirkungsquerschnitt}& \dif \sigma &= |f(\theta, \phi)|^2 \dif \Omega \\
    &\text{Lippmann-Schwinger-Gl.} & \Psi_k(x) &= \underbrace{e^{i k x}}_{\text{hom. Lsg.}} + \underbrace{\int \dif^3 x' G(x - x') v(x') \Psi_k(x')}_{\text{Falt. mit Greenfkt. von } (\Delta + k^2)} \\
    & \text{Greensche Funktion} & G(x) &= -\frac{e^{i k |x|}}{4\pi |x|} \\
    &\text{Streuamplitude} & f(\theta, \phi) &= -\frac{1}{4\pi} \int \dif^3 x' e^{-i \vec{k} \vec{x}'} v(\vec{x}) \psi_k(\vec{x})
\end{align*}

Bornsche Näherung durch rekursives Einsetzen:
\begin{align*}
    &\text{1. Näherung}& f^{(1)}(\vec{k}' - \vec{k} := \vec{q}) &= -\frac{m}{2\pi} \int \dif^3 x' e^{i \vec{q} \vec{x}'} V(\vec{x}')
\end{align*}

Beispiele:
\begin{table}[htb]
    \centering
    \begin{tabular}{lccc}
        & $V$ & DGl & $\tilde{V}$ \\
        Coulomb & $\frac{1}{4 \pi r}$ & $\Delta V = -\delta^3$ & $\frac{1}{\vec{q}^2}$ \\
        Yukawa & $\frac{e^{-M r}}{4 \pi r}$ & $(\Delta - M^2) V = -\delta^3$ & $\frac{1}{\vec{q}^2 + M^2}$ \\
        Delta & $\delta^3(\vec{x})$ &  & $1$ \\
        Ladungsvert. & $\int \dif^3 x \frac{\rho(\vec{x})}{4 \pi |\vec{x} - \vec{x}'|}$ & $\Delta V = - \rho$ & $\frac{\tilde{\rho}}{\vec{q}^2}$ \\
    \end{tabular}
\end{table}

\subsection*{Partialwellenmethode}

Lösung der Schrödingergleichung $\frac{1}{r} \partial_r^2 (r R(r)) + (k^2 - \frac{l(l+1)}{r})R(r) = v(r) R(r)$ für den Radialteil $R(r)$ der Wellenfunktion über Bessel- $j_l$ und Neumannfunktionen $n_l$.

Für $r\rightarrow \infty$ ist $r R(r) \propto \sin \left(kr - l\frac{\pi}{2} + \delta_l\right)$.

Falls $\delta_l$ bekannt gilt für $r \rightarrow \infty$:
\begin{align*}
    \psi(r, \theta, \phi) &= \sum_l \frac{2 l + 1}{2 k} \left( \left[ -i + 2 e^{i \delta_l} \sin \delta_l \right] \frac{e^{i k r}}{r} + i (-1)^l \frac{e^{-i k r}}{r}\right) P_l(\cos \theta) \\
    e^{i \mathbf{k} \mathbf{x}} &= \sum_l \frac{2 l + 1}{2 k} \left( -i \frac{e^{i k r}}{r} + i (-1)^l \frac{e^{-i k r}}{r}\right) P_l(\cos \theta) \\
    f(\theta) &= \sum_l \frac{2 l + 1}{ k}  e^{i \delta_l} \sin \delta_l  P_l(\cos \theta) \\
    \frac{\mathrm{d} \sigma}{\mathrm{d} \Sigma} &= |f(\theta)|^2 = \sum_{l l'} \frac{(2 l + 1)(2 l' + 1)}{k^2} e^{i \delta_l - i \delta_{l'}} \sin \delta_l \sin \delta_{l'} P_l P_{l'} \\
    \sigma &= \int \mathrm{d}\sigma = \int \mathrm{d} \Sigma |f(\theta)|^2 = 2\pi \int_{-1}^1 \mathrm{d}\cos \theta |f(\theta)|^2 =\frac{4\pi}{k^2} \sum_l (2 l + 1) \sin^2 \delta_l 
\end{align*}
Optisches Theorem: $\sigma = \frac{4\pi}{k} Im f(\theta = 0)$

\end{document}
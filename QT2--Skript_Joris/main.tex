\documentclass[11pt,a4paper]{report}
\usepackage[utf8]{inputenc}
\usepackage[german]{babel}
\usepackage{amsmath}
\usepackage{amsfonts}
\usepackage{amssymb}
\usepackage{slashed}
\usepackage{xfrac}
\usepackage{amsbsy}
\usepackage[left=2cm,right=2cm,top=2cm,bottom=2cm]{geometry}

\setlength{\parindent}{0cm}
\setlength{\parskip}{0.2cm}

\renewcommand{\baselinestretch}{1.2}

\renewcommand{\vec}{\boldsymbol}

\author{Joris Josiek}
\title{Skript QT2}

\setcounter{chapter}{-1}
\begin{document}
\maketitle

\tableofcontents

\chapter{Grundstruktur der Quantenmechanik}

\section{Postulate}

\textbf{Essenz:} Doppelspaltexperiment / Stern-Gerlach-Experiment\par 
\textbf{Zustand:} eindeutig / maximal präpariertes physikalisches System, reproduzierbares Verhalten, eindeutige Zeitentwicklung. Beschreibung durch $\left|\psi\right\rangle$ eines Hilbertraums. Linearkombinationen erlaubt!\par 
\textbf{Observablen:} Operatoren $\hat{A}$ (hermitesch, da reelle Eigenwerte $\leftrightarrow$ mögliche Messwerte)\par 
\textbf{Wahrscheinlichkeit:} Für ein Messergebnis $a_n$ ist die Wahrscheinlichkeit $\left|\left\langle a_n|\psi\right\rangle\right|^2$ (normierte Zustände).\par 
\textbf{Erwartungswert:} (Korrollar) $\langle\psi|\hat{A}|\psi\rangle$\par 
\textbf{Zeitentwicklung:} $\hat{H}$ (Hamiltonoperator), $\hat{H}$ sei nicht expl. zeitabh.
$$i\hbar\frac{\mathrm{d}}{\mathrm{d}t}\langle\psi_1|\hat{A}|\psi_2\rangle = \langle\psi_1|[\hat{A},\hat{H}]|\psi_2\rangle$$
\textbf{Schrödinger-Bild}
$$i\hbar\frac{\mathrm{d}}{\mathrm{d}t}|\psi (t)\rangle = \hat{H}|\psi (t)\rangle$$
\textbf{Heisenberg-Bild}
$$|\psi_H\rangle = e^{i\hat{H}t}|\psi (t)\rangle$$
$$\hat{A}_H(t)=e^{i\hat{H}t}\hat{A}e^{-i\hat{H}t}$$
$$i\hbar\frac{\mathrm{d}}{\mathrm{d}t}\hat{A}_H(t)=[\hat{A}_H(t),\hat{H}]$$

\section{Ortsraum, Teilchen in 1D}

Operatoren $\hat{x}$, $\hat{p}$, $[\hat{x},\hat{p}]=i\hbar$.\par 
EZe: $|x\rangle$, $|p\rangle$ (bilden jeweils Basis)\par 

Wellenfunktionen: $\psi (x) :=\langle x|\psi\rangle$, $\tilde{\psi}(p):=\langle p|\psi\rangle$

\chapter{Relativistische Quantenmechanik}

\newcommand{\dt}{\frac{\mathrm{d}}{\mathrm{d}t}}
\newcommand{\dif}{\mathrm{d}}
\newcommand{\vf}[1]{\mathbf{#1}}


\section{Kontinuierliche Symmetrien (Bsp. Rotationsinvarianz)}

Frage: Was ist Drehimpuls?

\subsection{Drehungen in 3D}

($\rightarrow$ Liegruppe $SO(3)$)\par 

Aktive Drehung: Bsp. $\vf{v}'=R_z(\theta )\vf{v}$ (Drehung um Winkel $\theta$ um $z$-Achse)\par 

Infinitesimale Drehungen, $\theta = \varepsilon\rightarrow 0$:
$$R_z(\varepsilon)=\mathbf{1} -i\varepsilon\begin{pmatrix}
0 & -i & 0 \\
i & 0 & 0 \\
0 & 0 & 0
\end{pmatrix} = \mathbf{1} - i\varepsilon \ell_z$$
$$\ell_z = \begin{pmatrix}
0 & -i & 0 \\
i & 0 & 0 \\
0 & 0 & 0  
\end{pmatrix}\qquad\ell_x = \begin{pmatrix}
0 & 0 & 0 \\
0 & 0 & -i \\
0 & i & 0  
\end{pmatrix}\qquad\ell_y = \begin{pmatrix}
0 & 0 & i \\
0 & 0 & 0 \\
-i & 0 & 0  
\end{pmatrix}\qquad (\ell_k)_{i,j}=-i\varepsilon_{ijk}$$

``Generatoren der zugehörigen Lie-Algebra"\par 

Charakteristische Kommutatorrelation: $[\ell_i, \ell_j]=i\varepsilon_{ijk}\ell_k$\par 

Endliche Drehungen: $R_z(\theta ) = \exp\:(-i\theta\ell_z)$

\subsection{Darstellungen}

Eine Darstellung einer Gruppe ist eine Zuordnung: $R\mapsto D(R)$ = Matrix / linearer Operator, mit 
$$D(R_1R_2) = D(R_1)D(R_2)$$
Physikalische Idee: Viele physikalische Größen $\rightarrow$ angeben, wie sie sich unter Drehungen verhält.
\begin{itemize}
\item Impuls: $\vf{p}\longmapsto \vf{p}'=R\vf{p}$
\item Energie: $E\longmapsto E' = E=D(R)E$ mit $\forall R: D(R)=1$
\item Ladung: $Q\longmapsto Q' = Q$
\item Dichte: $\rho \longmapsto \rho ' :\rho'(R\vf{x})=\rho (\vf{x})$
\item Quantenzustand $|\psi\rangle\longmapsto |\psi '\rangle = \hat{D}(R)|\psi\rangle$
\end{itemize}

Generatoren für Darstellungen: $\theta=\varepsilon\rightarrow 0$\par 
$$D(R_z(\varepsilon ))=\vf{1} - i\varepsilon J_z\qquad\text{(Analog für x, y)}$$
mit Operatoren $J_x, J_y, J_z$ wie $D(R_z(\varepsilon ))$, diese sind spezifisch für die Darstellung.
$$D(R_z(\theta ))=\exp\:(-i\theta J_z)$$
$$[J_i, J_j] = i\varepsilon_{ijk}J_k$$
\emph{Die Generatoren jeder Darstellung erfüllen dieselben Vertauschungsrelationen.}

\subsection{Drehungen in der Quantenmechanik}

\newcommand{\ket}[1]{|#1\rangle}
\newcommand{\braket}[2]{\langle #1|#2\rangle}
\newcommand{\erwop}[3]{\langle #1|#2|#3\rangle}

Darstellung von Drehungen:
$$\hat{D}(R_k(\theta )):\ket{\psi}\mapsto\ket{\psi'}=\hat{D}(R_k(\theta ))\ket{\psi}$$
Gruppenstruktur:
$$\hat{D}(R_1R_2)=\hat{D}(R_1)\hat{D}(R_2)$$
Falls Symmetrie:
$$\braket{\psi '}{\phi '} = \braket{\psi}{\phi} \Leftrightarrow\erwop{\psi}{\hat{D}^\dagger\hat{D}}{\phi}$$
$\hat{D}(R)$ ist ein \textit{unitärer} Operator. $[\hat{D}(R), H]=0$.\par 

Infinitesimale Drehung:
$$\hat{D}(R_k(\varepsilon ))=\vf{1}-i\varepsilon\hat{J}_k$$
Falls Symmetrie:
$$[\hat{J}_k, \hat{H}] = 0\qquad [\hat{J}_i, \hat{J}_j]=i\varepsilon_{ijk}\hat{J}_k$$
Per Definition: $\hat{\vf{J}}$ is Drehimpuls dieser Quantentheorie.\par 

Konsequenzen bei solchen $\hat{\vf{J}}$-Operatoren: (QT1)
$$[\hat{J}_z, \hat{\vf{J}}]=0\qquad \hat{J}_\pm =\hat{J}_x\pm i\hat{J}_y$$
Mögliche Eigenzustände: $\ket{j, m}$ mit $j = 0, \frac{1}{2}, 1, \frac{3}{2}, \ldots$ und $m=-j, \ldots , j$\par 

Einfachste nicht-triviale Darstellung: $j=\frac{1}{2}$, d.h. 2-Zustandssystem $\ket{\pm}:=\ket{j=\frac{1}{2},m=\pm\frac{1}{2}}$.
$$\ket{\psi}=\psi_+\ket{+}+\psi_-\ket{-}$$
$$\psi\overset{R_k(\theta )}{\longmapsto}\psi' = \left(\vf{1}-i\theta\frac{\sigma_k}{2}\right)\psi$$
mit Pauli-Matrizen $\sigma_k$.

\section{Lorentzinvarianz}

\subsection{Lorentzgruppe}

Drehungen: $(t,\vf{r})\longmapsto (t,R(\vf{r}))$\par 
Boosts in x-Richtung:
$$\begin{pmatrix}
t \\ x \\ y \\ z
\end{pmatrix}\longmapsto\begin{pmatrix}
\cosh\beta & \sinh\beta & 0 & 0 \\
\sinh\beta & \cosh\beta & 0 & 0 \\
0 & 0 & 1 & 0 \\
0 & 0 & 0 & 1
\end{pmatrix}\begin{pmatrix}
t \\ x \\ y \\ z
\end{pmatrix}$$
Generatoren: $\ell_x, \ell_y, \ell_z$ wie gehabt. Boosts: $\Lambda_x(\beta ) = \vf{1} - i\beta k_x+\mathcal{O}(\beta^2)$
$$k_x=i \begin{pmatrix}
0 & 1 & 0 & 0 \\
1 & 0 & 0 & 0 \\
0 & 0 & 0 & 0 \\
0 & 0 & 0 & 0 
\end{pmatrix}\qquad k_y=i \begin{pmatrix}
0 & 0 & 1 & 0 \\
0 & 0 & 0 & 0 \\
1 & 0 & 0 & 0 \\
0 & 0 & 0 & 0 
\end{pmatrix}\qquad k_z=i \begin{pmatrix}
0 & 0 & 0 & 1 \\
0 & 0 & 0 & 0 \\
0 & 0 & 0 & 0 \\
1 & 0 & 0 & 0 
\end{pmatrix}$$
6 Generatoren:
Vertauschungsrelationen (und zyklisch):
$$[\ell_x, \ell_y]=i\ell_z$$
$$[k_x,k_y]=-i\ell_z$$
$$[\ell_x,k_y]=ik_z$$

\subsection{Darstellungen}

\textbf{Def. Darstellung:} Matrizen/Operatoren $J_i$, $K_i$, mit $[J_x,J_y]=iJ_z$, $[K_x,K_y]=-iJ_z$, $[J_x,K_y]=iK_z$.\par 

Triviale Darstellung: $J_i=0$, $K_i=0$\par 

Spin $\frac{1}{2}$: $J_i=\sigma^i/2$, $K_i=-i\sigma^i/2$. Die Elemente des 2D Darstellungsraumes nennt man \textit{linkshändige Weyl-Spinoren}. (Andere Variante mit $K_i=+i\sigma^i/2$: Elemente sind \textit{rechtshändige Weyl-Spinoren})\par 

\paragraph{Partität/Raumspiegelung $P$:} $\vf{x}\mapsto -\vf{x}$, $\vf{p}\mapsto -\vf{p}$, $\vf{J}\mapsto\vf{J}$, $\vf{K}\mapsto -\vf{K}$. Falls $P$-Transformation genutzt werden soll, sind beide Darstellungen nötig $\Rightarrow$ 4D komplexer Spinorraum aus \textit{Dirac-Spinoren} notwendig.
$$\Psi =\begin{pmatrix}
\psi_\alpha \\ \overline{\psi}^{\dot{\alpha}}
\end{pmatrix}$$
Darstellung für Diracspinoren:
$$J_i=\begin{pmatrix}
\frac{\sigma^i}{2} & 0 \\ 0 & \frac{\sigma^i}{2}
\end{pmatrix}\qquad K_i =\begin{pmatrix}
-i\frac{\sigma^i}{2} & 0 \\ 0 & i\frac{\sigma^i}{2}
\end{pmatrix}$$
Diracspinoren: 4-komponentige komplexe Spinoren. Einfachste Darstellung mit $P$-Transformation.\par 

Lorentztransformationen und Darstellungen:
$${\Lambda^\mu}_\nu = {\delta^\mu}_\nu +{\omega^\mu}_\nu$$
(mit infinitesimalem und antisymmetrischem $\omega^{\mu\nu}$ (wenn beide Indizes oben!), z.B Drehung, Boost)

$$\Lambda = \vf{1} -\frac{i}{2}\omega^{\mu\nu}L_{\mu\nu}$$

mit $L_{ij} = -L_{ji} = \varepsilon_{ijk}\ell_k$ und $L_{i0}=-L_{0i}=k_i$

Für eine Darstellung $S$:
$$\boxed{S(\Lambda ) := \vf{1}-\frac{i}{2}\omega^{\mu\nu}L_{\mu\nu}}$$

\subsection{Diracspinoren und $\gamma$-Matrizen}
$\psi = (\psi_1,\psi_2,\psi_3,\psi_4)$ = komplexer Diracspinor.\par 

\paragraph{Def $\gamma$-Matrizen:} $\{\gamma^\mu ,\gamma^\nu\}=2g^{\mu\nu}\vf{1}$\par 

Weyl-Form:
$$\gamma^0 :=\begin{pmatrix}
0 & \vf{1}_2 \\ \vf{1}_2 & 0 
\end{pmatrix}\qquad\gamma^i \begin{pmatrix}
0 & \sigma^i \\ -\sigma^i & 0
\end{pmatrix}$$
Die Generatoren $\vf{J}$, $\vf{K}$ lassen sich so ausdrücken:
$$S_{\mu\nu}=\frac{i}{4}[\gamma_\mu,\gamma_\nu]$$
Dies reproduziert die Darstellungsmatrix $L_{\mu\nu}$ der Lorentztransformation.\par 

${\gamma^\mu}^\dagger$: ${\gamma^0}^\dagger = \gamma^0$, ${\gamma^i}^\dagger = -\gamma^i = \gamma^0\gamma^i\gamma^0$
$${S^\dagger}_{\mu\nu} = \gamma^0 S_{\mu\nu} \gamma^0$$
$$S^{-1}(\Lambda )=\vf{1}+\frac{i}{2}\omega^{\mu\nu}S_{\mu\nu}=\gamma^0 S^\dagger (\Lambda )\gamma^0$$
Def \textit{Adjungierter Spinor:} $\overline{\psi}:=\psi^\dagger\gamma^0$\par 
Lorentz:
$$\psi\longmapsto S(\Lambda )\psi$$
$$\overline{\psi}\longmapsto \overline{\psi}S^{-1}(\Lambda )$$
$$\overline{\psi}\psi\longmapsto\overline{\psi}\psi$$
$$\overline{\psi}\gamma^\mu\psi\longmapsto {\Lambda^\mu}_\nu\overline{\psi}\gamma^\nu\psi$$
$$S^{-1}(\Lambda )\gamma^\mu S(\Lambda )={\Lambda^\mu}_\nu\gamma^\nu$$

\section{Überblick über relativistische Wellengleichungen}

Welche Gleichungen wären erlaubt durch Lorentzinvarianz?\par 

Notation: \par 
\begin{itemize}
\item 4-Vektoren: $(x^\mu )=(t,\vf{x})$, $(p^\mu)=(E,\vf{p})$
\item Lorentzinvarianten sind Skalarprodukte, z.B. $p^\mu p_\mu = E^2-\vf{p}^2=:m^2$
\item Ableitungen: $\partial_\mu = \left(\frac{\partial}{\partial x^\mu}\right) = (\partial_t,\nabla )$, $\square = \partial_\mu\partial^\mu = \partial_t -\Delta$
\item Elektrodynamik: $j^\mu = (\rho ,\vf{j})$, $\partial_\mu j^\mu = 0$, $A^\mu = (\phi, \vf{A})$, $F^{\mu\nu}=\partial^\mu A^\nu -\partial^\nu A^\mu$\\
Maxwell: $\partial_\mu F^{\mu\nu} = \mu_0 j^\nu$, homogene Gleichung automatisch durch Potentiale erfüllt.\\
Lorentz-Transf.: $x'^\mu ={\Lambda^\mu}_\nu x^\nu$, $j'^\mu (x')={\Lambda^\mu}_\nu j^\nu (x)$
\end{itemize}

\subsection{Klein-Gordon-Gleichung}

$\phi (x)$ sei Skalarfeld ($\phi\mapsto \phi '$ mit $\phi '(x')=\phi (x)$).

$$\boxed{\square \phi (x) + m^2\phi (x) = 0}$$

\paragraph{Interpretation?}
\begin{itemize}
\item Einfachste relativistische Differentialgleichung
\item ``erraten aus QM'' (mit QM Ersetzungsregeln $E\rightarrow i\partial_t$, $\vf{p}\rightarrow -i\nabla$)
\item Nichtrelativistischer Limes: ein Teilchen, $E\approx m + \text{Korrektur}$. Ansatz:
$$\psi (\vf{x},t)=e^{-imt}\psi_{n.r.} (\vf{x},t)$$
$$\Rightarrow \partial_t^2\psi = (-2im\partial_t\psi_{n.r.}-m^2\psi_{n.r.}+\mathcal{O}(\ddot{\psi}))e^{-imt}$$
$$\Rightarrow 2im\partial_t\psi_{n.r.}=-\Delta\psi_{n.r.}$$
\item Klassische Feldgleichung:
$$\mathcal{L}_{KG}=(\partial^\mu \phi^*)(\partial_\mu\phi )-m^2\phi^*\phi$$
Euler-Lagrange:
$$0=\partial_\rho\frac{\partial\mathcal{L}}{\partial (\partial_\rho\phi^*)}-\frac{\partial\mathcal{L}}{\partial\phi^*}$$
\end{itemize}

\paragraph{Rolle als QM Wellengleichung für ein Teilchen in Ortsdarstellung:}
\mbox{}\par
Schrödinger-Gleichung nicht-relativistisch: $i\partial_t\psi = -\frac{\Delta}{2m}\psi$\\
Klein-Gordon-Gleichung: $-\partial_t^2\phi = (-\Delta +m^2)\phi$\par 

Aufenthaltwahrscheinlichkeitsdichte: Suche $(j^\mu )=(\rho ,\vf{j})$ mit Kontinuitätsgleichung $\partial_\mu j^\mu =0$:
$$\phi^* (\square + m^2)\phi -\phi (\square+m^2)\phi^* = 0$$
$$=\partial_\mu [\phi^*\partial^\mu \phi - \phi\partial^\mu\phi^*]$$
Definiere 4-Stromdichte:
$$j^\mu = \frac{i}{2m}\left[\phi^*\partial^\mu\phi -\phi\partial^\mu\phi^*\right]$$
$$\Rightarrow \vf{j}=-\frac{i}{2m}\left[\phi^*\nabla\phi -\phi\nabla\phi^*\right]$$
$$\Rightarrow \rho = \frac{i}{2m}\left[\phi^*\partial_t\phi - \phi\partial_t\phi^*\right]$$
\paragraph{Interpretation}
\begin{itemize}
\item $\rho$ ist nicht positiv definit! $\rho < 0$ möglich! Also kann $\rho$ nicht als Aufenthaltswahrscheinlichkeit interpretiert werden.
\item Lösungen: $\phi\sim e^{-iEt+i\vf{p}\cdot\vf{x}}$: $\rho = \frac{E}{m}>0$, $\rho\sim e^{+iEt-i\vf{p}\cdot\vf{x}}$: $\rho = -\frac{E}{m}<0$: negative Energie möglich!?
\item Idee: KG-Gl. beschreibt zwei Teilchentypen (Teilchen + Antiteilchen) mit entgegengesetzten Ladungen. Interpretiere $\rho$ als elektrische Ladungsdichte.
\end{itemize}

\subsection{Dirac-Gleichung}

$\psi (x)$ sein ``Dirac-Spinorfeld'' d.h. $\psi\mapsto \psi '$ mit $\psi'(x')=S(\Lambda ) \psi (x)$.
$$S(\Lambda ) = \vf{1}_4-\frac{i}{2}\omega^{\mu\nu}S_{\mu\nu}$$
$$S_{\mu\nu} = \frac{i}{4}[\gamma_\mu ,\gamma_\nu ]$$
$$\{\gamma^\mu ,\gamma^\nu\} = 2g^{\mu\nu}\vf{1}_4$$
Dirac-Gleichung:
$$\boxed{ (i\partial_\mu\gamma^\mu - m)\psi =0}$$

\paragraph{Interpretation:}
\begin{itemize}
\item nicht einfachste Differenzialgleichung
\item erraten von Dirac: gewünscht ``Wurzel aus KG-Gleichung'' (Herleitung $\curvearrowright$ Lit.)
\item $\mathcal{L} = \overline{\psi}(i\partial_\mu \gamma^\mu - m)\psi$
\item Adjungierte Dirac-Gl. $i\partial_\mu \overline{\psi}\gamma^\mu + m\overline{\psi} = 0$
$$\Rightarrow \partial_\mu (\overline{\psi}\gamma^\mu\psi )=0$$
\item Def. $j^\mu = \overline{\psi}\gamma^\mu\psi $, $\rho = \psi^\dagger\psi$ ist positiv-definit
\end{itemize}

\paragraph{Vollständige Darstellung der Lorentztransformationen}\mbox{}\par 

$$\psi '(x)=S(\Lambda )\psi (\Lambda^{-1}x) = (\vf{1}-\frac{i}{2}\omega^{\mu\nu}S_{\mu\nu})\psi (x-\omega x)$$
und
$$\psi '=(\vf{1}-\frac{i}{2}\omega^{\mu\nu}\hat{J}_{\mu\nu})\psi$$
($\hat{J}$ Generatoren der Darstellung der Lorentz-Algebra auf dem Fkt.-Raum der Spinorfelder)
$$\Longrightarrow \hat{J}_{\mu\nu}=i(x_\mu\partial_\nu -x_\nu\partial_\mu )+S_{\mu\nu}$$
$$\hat{J}_{\mu\nu}=\hat{L}_{\mu\nu}+S_{\mu\nu}$$
Analog zur KG-Gl. treten Inkonsistenzen auf, wenn man Diracgl. als 1-Teilchen-Theorie auffasst. Die Probleme sind ähnlich aber nicht gleich.

\section{Physik und Lösungen der Diracgleichung}

\subsection{Freie Lösungen, Impuls-/Spin-Eigenzustände}

\newcommand{\dslash}{\slashed{\partial}}
\newcommand{\pslash}{\slashed{p}}

Dirac-Gleichung: $(i\dslash - m)\psi = 0$\\
Gesamt-Drehimpuls: $\hat{J}_{ij}=\hat{L}_{ij}+S_{ij}$. Spin-EZ: $\pm\frac{1}{2}$\par 

Ansatz: $\psi (x)= w(p)e^{\mp ipx}$ (mit $px = p_\mu x^\mu$)
$$\Rightarrow (\pm \pslash - m)w(p)=0$$
Eigenwertgleichung für $\pslash$ !\par 

Beachte: $\pslash^2 = p^\mu\gamma_\mu p^\nu\gamma_\nu = p^\mu p^\nu \gamma_\mu\gamma_\nu = \frac{1}{2}p^\mu p^\nu \{\gamma_\mu ,\gamma_\nu\} = p^2\vf{1}$\par 
D.h. $\pslash$ hat EWe $\pm \sqrt{p^2}$ vermutlich je 2-fach entartet. Nicht-triviale Lösung der EW-Gl. für $p^2=m^2$ $\rightarrow$ Teilchen mit Ruhemasse $m$ beschrieben.

\paragraph{Bezeichnungen der Lösungen}
$$(\pslash - m)u(p,s)=0$$
$$(\pslash + m)v(p,s)=0$$
Beispiel: $p^2 = m^2$, $(p^\mu )=(E,0,0,p_z)$ in $z$-Richtung, $E^2=p_z^2+m^2$.
$$\pslash = p^\mu \gamma_\mu = E\gamma_0 + p_z\gamma_3 = E\gamma^0-p_z\gamma^3=\begin{pmatrix}
\mathbf{1}E & -p_z\sigma^3 \\ p_z\sigma^3 & -\mathbf{1}E
\end{pmatrix}$$
Es gilt $[\pslash , S_{12}]=0$, d.h. $\pslash$ und $S_z$ haben simultane Eigenzustände. (allg. $\pslash$ und $\frac{\vf{p}\cdot\vf{S}}{|\vf{p}|}$ = Helizitätsoperator simultan Diagonalisierbar).\par 
EW-Gleichung lösen:
$$u(p,+\sfrac{1}{2})=N\cdot\begin{pmatrix}
E+m \\ 0 \\ p_z \\ 0
\end{pmatrix}$$
$$u(p,-\sfrac{1}{2})=N\cdot\begin{pmatrix}
0 \\E+m \\ 0 \\ -p_z 
\end{pmatrix}$$
mit $N=\frac{1}{\sqrt{E+m}}$.
$$v(p, +\sfrac{1}{2})=N\cdot\begin{pmatrix}
p_z \\ 0 \\ E+m \\ 0
\end{pmatrix}$$
$$v(p, -\sfrac{1}{2})=N\cdot\begin{pmatrix}
0 \\ -p_z \\ 0 \\ E+m
\end{pmatrix}$$
Spinoren für andere $\vf{p}$: $\vf{p}=R\vf{p}_z = e^{-\frac{i}{2}\omega^{\mu\nu}L_{\mu\nu}}\vf{p}_z$:
$$u(p,s)=e^{-\frac{i}{2}\omega^{\mu\nu}S_{\mu\nu}}u(p_z,s)$$
\paragraph{Negative Energien}
$$\psi (x)=u(p,s)=e^{-iEt+i\vf{p}\cdot\vf{x}}$$
$$\psi (x)=v(p,s)=e^{+iEt-i\vf{p}\cdot\vf{x}}$$
D.h. Energie $(-E)<0$ für $v$-Lösungen.

\subsection{Mehr zum Drehimpuls}

Man betrachte die Diracgleichung als quantenmechanische 1-Teilchen-Gleichung. (sinnvoll, solange Antiteilchen und QFT Effekte vernachlässigbar sind).\par 

Formulierung analog zur Schrödingergleichung im Ortsraum:
$$(i\dslash - m)\psi = 0$$
Multiplikation mit $\gamma^0$ von links und nach Zeitableitung umstellen:
$$i\partial_t\psi = (-i\gamma^0\gamma^i\partial_i+m\gamma^0)\psi =:\hat{H}^{(0)}_D\psi$$
\paragraph{Drehimpuls} aus Darstellung der Lorentztransformation.
$$\hat{J}_{ij}=i(x_i\partial_j-x_j\partial_i)+\hat{S}_{ij} = \hat{L}_{ij}+\hat{S}_{ij}$$
$$\hat{\vf{J}}=\hat{\vf{L}}+\hat{\vf{S}}$$
Es gilt $[\hat{H}^{(0)}_D,\hat{\vf{J}}]=0$, d.h. Gesamtdrehimpuls erhalten. $[\hat{H}^{(0)}_D,\hat{\vf{L}}]=\gamma^0\gamma_1\partial_y-\gamma^0\gamma_2\partial_x$.
\paragraph{Helizität}
$$\frac{\hat{\vf{S}}\cdot\hat{\vf{p}}}{|\hat{\vf{p}}|}$$
$$[\hat{H}^{(0)}_D, \hat{\vf{S}}\cdot\hat{\vf{p}}]=[\hat{H}^{(0)}_D,\frac{1}{2}\epsilon_{ijk}S_{ij}\hat{p}^k]=\sim\frac{1}{2}\epsilon_{ijk}\gamma^0\gamma_i\partial_j\partial_k=0$$
Es gibt simultane Eigenzustände zu Energie, Impuls, Helizität.

\paragraph{Interpretation der 4 Komponenten von $\psi$}\mbox{}\par 
Zu gegebenem Impuls $\vf{p}$: 4 linear unabhängige Lösungen:
\begin{itemize}
\item $E>0$, Helizität $\pm\frac{1}{2}$
\item $E<0$, Helizität $\pm\frac{1}{2}$
\end{itemize}

\subsection{Kopplung ans elektromagnetische Feld}

Freie Diracgleichung: $(i\gamma^\mu\partial_\mu -m)\psi = 0$\par 
Freie Klein-Gordon-Gleichung: $(-\partial_\mu\partial^\mu - m^2)\phi = 0$\par 
Relativistisches klassisches Teilchen: $L=\frac{1}{2}m\frac{\dif x^\mu}{\dif\tau}\frac{\dif x_\mu}{\dif\tau}$

Kopplung and e.m. Feld soll relativistisch invariant und eichinvariant sein. (Eichung $A^\mu (x)\mapsto A^\mu (x)+\partial^\mu\theta (x)$).\par 

Klassisches Teilchen:
$$L=\frac{1}{2}m\frac{\dif x^\mu}{\dif\tau}\frac{\dif x_\mu}{\dif\tau} - e\frac{\dif x_\mu}{\dif\tau}A^\mu (x)$$
(Einfachse denkbare relativistische WW, Wirkung ist eichinvariant, reproduziert Coulomb- und Lorentzkraft)\par 
Kanonisch konjugierter Impuls:
$$\mathcal{P}^\mu = \frac{\partial L}{\partial\frac{\dif x_\mu}{\dif\tau}}=m\frac{\dif x^\mu}{\dif\tau}-eA^\mu$$
$$\Rightarrow H=\frac{1}{2m}(\mathcal{P}^\mu + eA^\mu )^2$$
Rezept: minimale Kopplung $\mathcal{P}^\mu\rightarrow\mathcal{P}^\mu + eA^\mu$, Klein-Gordon-Gleichung:
$$\boxed{\left[\left( i\partial^\mu + eA^\mu\right)\left( i\partial_\mu +eA_\mu \right) -m^2\right]\phi = 0}$$
Dirac-Gleichung:
$$\boxed{\left(i\dslash + e\slashed{A}-m\right)\psi = 0}$$
Elektromagnetische Stromdichte:
$$j^\mu = e\overline{\psi}\gamma^\mu\psi$$
Eichinvarianz:
\begin{align*}
A^\mu (x) &\longrightarrow A^\mu (x) +\partial^\mu\theta (x) \\
\psi (x) &\longrightarrow e^{ie\theta (x)}\psi(x)
\end{align*}
Eichkovariante Ableitung:
$D^\mu\psi :=(\partial^\mu -ieA^\mu )\psi$. Damit gilt $D^\mu\psi \longrightarrow e^{ie\theta (x)} D^\mu\psi$


\subsection{Nichtrelativistischer Limes}

Nichtrelativistische Schrödingergleichung mit e.m. Feld:
$$(i\partial_t + e\Phi )\psi = \frac{(\hat{\vf{p}}+e\vf{A})^2}{2m}\psi$$
Klein-Gordon-Gleichung:
$$\left[\left( i\partial^\mu + eA^\mu\right)\left( i\partial_\mu +eA_\mu \right) -m^2\right]\phi = 0$$
$(A^\mu ) = (\Phi ,\vf{A})$, $(i\partial^j) = (-i\partial_j ) = (p^j)$.\par 
Ansatz: 
\begin{itemize}
\item $\phi$ ist Energie-EZ, $i\partial_t\phi = E\phi$
\item $E=m+\textit{klein}$, $E>0$
\item $e|A^\mu |\ll m$
\item $|\partial_tA^\mu |\ll |mA^\mu |$
\item $|p|\ll m$
\end{itemize}
Einsetzen in KG-Gl.:
$$\left[ (i\partial_t+e\Phi)(E+e\Phi )-(\hat{\vf{p}}+e\vf{A})^2-m^2\right]\phi = 0$$
Vernachlässigen von $\partial_t\Phi$:
$$\left[ (E+e\Phi )^2 - (\hat{\vf{p}}+e\vf{A})^2-m^2\right]\phi = 0$$
Mit $E+e\Phi = m + (E-m+e\Phi )$ mit Vernachlässigung des Quadrates der letzten Klammer:
$$\left[ 2m(E-m+e\Phi ) - (\hat{\vf{p}}+e\vf{A})^2 \right]\phi = 0$$
Daraus folgt direkt die nichtrelativistische Schrödingergleichung.

\paragraph{Diracgleichung mit e.m. Feld}
$$(i\slashed{D} - m)\psi = 0$$
Ansatz wie oben. Aufteilung des Diracspinors in zwei Paulispinoren:
$$\psi = \begin{pmatrix}
\psi_A \\ \psi_B
\end{pmatrix}$$
$$\begin{pmatrix}
iD_0 -m & iD_i\sigma^i \\
-iD_i\sigma^i & -iD_0 - m
\end{pmatrix}\begin{pmatrix}
\psi_A \\ \psi_B
\end{pmatrix}=0$$
Nach Ansatz: $iD_0\rightarrow E+e\Phi $, $iD_i\sigma^i = -\vec{\sigma}(\hat{\vf{p}}+e\vf{A})$.
\begin{align*}
(E-m+e\Phi )\psi_A - \vec{\sigma}(\hat{\vf{p}}+e\vf{A})\psi_B &= 0 \\
(-E-m-e\Phi )\psi_B + \vec{\sigma}(\hat{\vf{p}}+e\vf{A})\psi_A &= 0
\end{align*}
Eliminiere 
$$\psi_B = \frac{\vec{\sigma}(\hat{\vf{p}}+e\vf{A})}{E+m+e\Phi}\psi_A \cong \left(\frac{1}{2m}+\mathcal{O}(m^{-2})\right)\vec{\sigma}(\hat{\vf{p}}+e\vf{A})$$
$$\Rightarrow\quad (E-m+e\Phi )\psi_A = \frac{1}{2m}\left(\vec{\sigma}(\hat{\vf{p}}+e\vf{A})\right)\left(\vec{\sigma}(\hat{\vf{p}}+e\vf{A})\right)\psi_A$$
\paragraph{Vereinfachung der $\sigma$-Anteile}
$$(\vec{\sigma}\cdot\hat{\vf{O}})(\vec{\sigma}\cdot\hat{\vf{O}}) = \sigma^i\hat{O}^i\sigma^j\hat{O}^j=\sigma^i\sigma^j\hat{O}^i\hat{O}^j$$
$$=\left(\frac{1}{2}\left\{\sigma^i,\sigma^j\right\} + \frac{1}{2}\left[\sigma^i,\sigma^j\right]\right)\hat{O}^i\hat{O}^j = \left(\delta^{ij}+i\epsilon^{ijk}\sigma^k\right)\hat{O}^i\hat{O}^j$$
$$=\hat{\vf{O}}^2+i\epsilon^{ijk}\sigma^k\frac{1}{2}[\hat{O}^i,\hat{O}^j]$$
Hier: $\hat{\vf{O}}=(\hat{\vf{p}}+e\vf{A})$:
$$\cdots = (\hat{\vf{p}}+e\vf{A})^2 + i\epsilon^{ijk}\sigma^k(-i\partial_ieA^j)$$
$$=(\hat{\vf{p}}+e\vf{A})^2 + e\vf{B}\cdot\vec{\sigma}$$
$$\boxed{(E-m+e\Phi )\psi_A=\left[\frac{(\hat{\vf{p}}+e\vf{A})^2}{2m}+\frac{e}{2m}\vec{\sigma}\cdot\vf{B}\right]\psi_A}$$
Pauli-Gleichung enthält Term $\vf{S}\cdot\vf{B}$ ($\vf{S}=\vec{\sigma}/2$) mit Vorfaktor:
$$\boxed{g_s\frac{e}{2m}\vf{S}\cdot\vf{B}\qquad ,\qquad g_s=2}$$

\paragraph{Bedeutung des $g_s$-Terms} Allg. Hamiltonian für magnetischen Dipol $\vec{\mu}$ im $\vf{B}$-Feld:
$$H = -\vec{\mu}\cdot\vf{B}_{ext}$$
Vergleich mit Pauli-Gleichung liefert $\vec{\mu}_s = -g_s\frac{e}{2m}\vf{S}$ mit $g_s = 2$. Das ist ein intrinsisches magnetisches Dipolmoment, proportional zum Spin.\par 

Vergleich mit klassischer Elektrodynamik (rotierende Ladungsverteilung, Ladung $Q$, Masse $M$, Drehimpuls $\vf{L}$) liefert $\vec{\mu}=\frac{Q}{M}\vf{L}$ $\Rightarrow$ Klassisches Ergebnis entspricht $g=1$.

\paragraph{Interpretation des ersten Terms} (identisch in der nicht-relativistischen Schrödingergleichung)
$$\frac{(\hat{\vf{p}}+e\vf{A})^2}{2m}=\underbrace{\frac{\hat{\vf{p}}^2}{2m}}_{E_{kin}}+\underbrace{\frac{e}{2m}(\hat{\vf{p}}\vf{A}+\vf{A}\hat{\vf{p}})+\frac{e^2}{2m}\vf{A}^2}_{\text{e.m. WW}}$$
Bsp. homogenes $\vf{B}$-Feld: setze $\vf{A}(x)=-\frac{1}{2}(\vf{x}\times\vf{B})$, dann $\vf{B}=\nabla\times\vf{A}$.
$$\hat{\vf{p}}\vf{A}+\vf{A}\hat{\vf{p}}=\vf{B}\cdot\hat{\vf{L}}$$
$$\Rightarrow \text{Erster Term } = \frac{\hat{\vf{p}}^2}{2m} +\frac{e}{2m}\vf{B}\cdot\hat{\vf{L}}+\frac{e^2}{2m}\vf{A}^2$$

\subsection{Weitere Konsequenzen: Spin-Bahn-Kopplung}

Höhere Ordnungen im nicht-relativistischen Limes:
\begin{itemize}
\item Spin-Bahn-Kopplung $\sim\vf{L}\cdot\vf{S}$ (Feinstrukturaufspaltung)
\item Darwin-Term
\item Korrektur E-kin. 
\end{itemize}

Saubere Herleitung durch systematische Entwicklung in Potenzen von $m$. $\frac{1}{m}$ sei eine kleine Größe. \\ $\rightarrow$ Foldy-Wouthuysen-Transformation/-Bild.

$$(i\slashed{D}-m)\psi = 0$$
$$\Leftrightarrow i\partial_t\psi = (-e\Phi + m\gamma^0 - iD_i\gamma^0\gamma^i)\psi = H_D\psi$$

Idee: Unitäre Transformation / neues "Bild", Zerlegung in 2-Spinoren.
$$\psi = e^{-iS}\psi '=e^{-iS}\begin{pmatrix}
\psi_A' \\ \psi_B'
\end{pmatrix}$$
$S$ hermitesch, eventuell $t$-abhängig.\par 

Neuer Hamiltonian:
$$i\partial_t\psi' = i\partial_t (e^{iS}\psi ) = (i\partial_t e^{iS})\psi + e^{iS}i\partial_t\psi$$
$$=\left[ (i\partial_te^{iS})e^{-iS}+e^{iS}H_De^{-iS}\right]\psi '$$
$$H_D'=i(i\dot{S}+\frac{i^2}{2}[S,\dot{S}]+\frac{i^3}{6}[S,[S,\dot{S}]]+\ldots )+ H_D + i[S,H_D]+\frac{i^2}{2}[S,[S,H_D]]+\ldots$$

Idee 2: $H_D'$ soll blockdiagonal sein in 2-Spinoren (bis zu bestimmter Ordnung) $\rightarrow$ Gleichung für $\psi_A '$ reicht aus.\par 

Konkret: 
$$H_D=m\gamma^0 + (-e\Phi ) + \begin{pmatrix}
0 & (\vf{p}+e\vf{A})\cdot\vec{\sigma} \\
(\vf{p}+e\vf{A})\cdot\vec{\sigma} & 0
\end{pmatrix}= \underbrace{m\gamma^0}_{\mathcal{O}(m^1)} + \underbrace{\mathcal{E}}_{{\mathcal{O}(m^0)}} + \underbrace{\mathcal{O}}_{{\mathcal{O}(m^0)}}$$
Häufige Umformung: $\gamma^0 O =-O\gamma^0$ mit ungeradem Operator $O$.\par 

1. Schritt: arbeite bis $\mathcal{O}(m^0)$: Setze $S=\mathcal{O}(m^{-1})$
$$H_D'=H_D + i[S,H_D] + \mathcal{O}(m^{-1}) = m\gamma^0+\mathcal{E}+\mathcal{O}+i[S,m\gamma^0+\mathcal{E}+\mathcal{O}] + \mathcal{O}(m^{-1})$$
$$=m\gamma^0 +\mathcal{E}+\mathcal{O} + i[S,m\gamma^0]$$
Lösung: $S=-\frac{i}{2m}\gamma^0\mathcal{O}$\par 
Damit $H_D'$ komplett ausrechnen bis $\mathcal{O}(m^{-2})$:
$$H_D'=H_D+i[S,H_D]-\dot{S}+\frac{i^2}{2}[S,[S,H_D]]-\frac{i}{2}[S,\dot{S}] + \frac{i^3}{6}[S,[S,[S,H_D]]]+\mathcal{O}(m^{-3})$$

Für die einzelnen Terme finden Wirkung
\begin{align*}
    i [S, H_D] &= i \left[-\frac{i}{2m} \gamma^0 \mathcal{O}, m \gamma^0 + \mathcal{E} + \mathcal{O} \right] = -\mathcal{O} + \frac{1}{2m} \gamma^0 [\mathcal{O}, \mathcal{E}] + \frac{1}{m} \gamma^0 \mathcal{O}^2 \\
    - \dot{S} &= \frac{i}{2m} \gamma^0 \dot{\mathcal{O}} \\
    \frac{i}{2} [S, \dot{S}] &= -\frac{i}{8 m^2} [\mathcal{O}, \dot{\mathcal{O}}] \\
    \frac{i^2}{2} [S, [S, H_D]] &= -\frac{1}{2m} \gamma^0 \mathcal{O}^2 - \frac{1}{8 m^2} [\mathcal{O}, [\mathcal{O}, \mathcal{E}]] - \frac{1}{2m^2} \mathcal{O}^3 \\
    \frac{i^3}{3!} [S, [S, [S, H_D]]] &= \frac{1}{6m^2} \mathcal{O}^3
\end{align*}
Der neue Hamiltonian ist nun
\begin{align*}
    H_D' &=\underbrace{m \gamma^0 + \mathcal{E} + \frac{1}{2m} \gamma^0 \mathcal{O}^2 - \frac{1}{8 m^2} [\mathcal{O}, i \dot{\mathcal{O}} + [\mathcal{E}, \mathcal{O}]]}_{\text{gerade} \;=: H_{D,\text{even}}'} + \\
    &= \underbrace{\frac{1}{2m} \gamma^0 (i \dot{\mathcal{O}} + [\mathcal{O}, \mathcal{E}]) - \frac{1}{6m^2} \mathcal{O}^3}_{\text{ungerade} \; =: \mathcal{O}'} \\
    &=: H_{D,\text{even}}' + \mathcal{O}'
\end{align*}

2. Schritt: arbeite bis $\mathcal{O}(m^-1)$: 

In Analogie setzen wir $\psi' = e^{i S'} \psi''$ mit $S' = -\frac{i}{2m} \gamma^0 \mathcal{O}'$ und erhalten
$$H_D'' = H_{D,\text{even}}' + i [S', \mathcal{E}] - \dot{S}' + \mathcal{O}(m^{-3}) := D_{D,\text{even}} + \mathcal{O}''$$

3. Schritt: arbeite bis $\mathcal{O}(m^{-2})$:

Wir setzen wieder $\psi'' = e^{i-i S''} \psi'''$ mit $S'' = -\frac{i}{2m} \gamma^0 \mathcal{O}'' = \mathcal{O}(m^{-3})$.

HIER FEHLT NOCH DIE GLEICHUNG FÜR $H_D'''$

Vollständig ausgerechnet:
$$H_D''' = \underbrace{m\gamma^0 + \mathcal{E} + \frac{1}{2m} \gamma^0 \mathcal{O}^2}_{\mathcal{O}(m^{-1})} - \underbrace{\frac{1}{8m^2} [\mathcal{O}, i \dot{\mathcal{O}} + [\mathcal{O}, \mathcal{E}]]}_{\mathcal{O}(m^{-2})}$$

\begin{itemize}
    \item Terme bis $\mathcal{O}(m^{-1})$ liefern genau den Limes aus 1.4.4 inkl. des $g-2$-Terms
    \item Zusätzliche Terme der relativistischen Korrektur bis $\mathcal{O}(m^{-2})$
\end{itemize}

Wir diskutieren diese Terme anhand des Zentralpotentials mit $\vf{A} = 0$ und $\Psi(\vf{x}, t) = \Psi(r)$ mit $r = |\vf{x}|$.
Es ergeben sich die Terme
\begin{align*}
    \nabla \Psi (r) &= \frac{\vf{x}}{r} \frac{\mathrm{d} \Psi}{\mathrm{d} r} \\
    \vf{E} &= - \nabla \Psi \\
    \mathcal{E} &= e \Psi \\
    \mathcal{O} &= \begin{pmatrix}
        0 & \vec{\sigma} \cdot\vf{p} \\ \vec{\sigma}\cdot\vf{p} & 0 \\
    \end{pmatrix} = -i \begin{pmatrix}
        0 & \vec{\sigma}\cdot\nabla \\ \vec{\sigma}\cdot \nabla & 0 \\
    \end{pmatrix} \\
    [\mathcal{O}, \mathcal{E}] &= -i e \begin{pmatrix}
        0 & \vec{\sigma}\cdot \vf{E} \\ \vec{\sigma}\cdot \vf{E} & 0 \\
    \end{pmatrix} \\
    [\mathcal{O}, [\mathcal{O}, \mathcal{E}]] &= (-i) (-i e) \begin{pmatrix}
        [\vec{\sigma}\cdot \nabla, \vec{\sigma}\cdot \vf{E}] & 0 \\ 0 & [\vec{\sigma}\cdot \nabla, \vec{\sigma}\cdot \vf{E}] \\
    \end{pmatrix}  \\
    [\vec{\sigma}\cdot \nabla, \vec{\sigma}\cdot \vf{E}]  &= \sigma^i \sigma^j (\partial_i E^j + E^j \partial_i) - \sigma^j \sigma^i E^j \partial_i \\
    &= \nabla\cdot \vf{E} + \underbrace{i \vec{\sigma}\cdot (\nabla \times \vf{E})}_{= 0} 
    + \underbrace{i^2 \epsilon^{i j k} \sigma^k E^j \partial_i}_{=2 \vec{\sigma}\cdot (\vf{E} \times \vec{p})} \\
    &= \nabla\cdot\vf{E} - \frac{2}{r} \frac{\mathrm{d}\Psi}{\mathrm{d}r} \vec{\sigma}\cdot \vf{L}
\end{align*}
Wir finden den nun bis zum $\mathcal{O}(m^{-2})$ Term blockdiagonalen Hamiltonian
$$H_D''' = \frac{e}{8 m^2} \nabla\cdot\vf{E} - \frac{e}{2 m^2 r} \frac{\mathrm{d} \Psi}{\mathrm{d}r} \vf{S}\cdot \vf{L}$$
Der obere Block ist 
\begin{align*}
    H_{\text{eff}} &= m + H_{\mathcal{O}(m^{-1})} + H_{\mathcal{O}(m^{-2})} + \ldots \\
    H_{\mathcal{O}(m^{-1})} &= H_{\text{Pauli}} = -e \Psi + \frac{(\vf{p} + e \vf{A})^2}{2 m} + \frac{e}{2m}\vec{\sigma}\cdot \vf{B} \\
    H_{\mathcal{O}(m^{-2})} &= \underbrace{\frac{e}{8m^2}\nabla\cdot \vf{E}}_{\text{Darwin-Term}} - \underbrace{\frac{e}{2m^2 r} \frac{\mathrm{d}\Psi}{\mathrm{d}r} \vf{S}\cdot \vf{L}}_{\text{Spin-Bahn-Kopplung}}
\end{align*}
Diskussion:
\begin{itemize}
    \item Darwin-Term: beim Atom $\nabla\cdot\vf{E} = 4\pi \rho_{\text{Kern}} \propto \delta^{(3)}(\vf{x})$ ergibt sich eine Korrektur für die s-Orbitale, die am Kern eine endliche Aufenthaltswahrscheinlichkeit haben
    \item Spin-Bahn-Koppluns: Wegen dieses Terms $[H_{\text{eff}}, \vf{S}] \neq 0$ und $[H_{\text{eff}}, \vf{L}] \neq 0$, aber $[H_{\text{eff}}, \vf{J}] = 0$.
\end{itemize}

\chapter{Ununterscheidbare Teilchen\\ \Large{Bosonen und Fermionen}}

Klassisch: jedes Teilchen hat eine eindeutige Bahnkurve $\rightarrow$ prinzipiell daran erkennbar.\par 

QM: keine eindeutige Bahnkurve

\paragraph{Fragen}
\begin{itemize}
\item Existieren ``ununterscheidbare Teilchen''? $\rightarrow$ Ja! (experimenteller Beweis)
\item Wie beschreibt man das? $\rightarrow$ Mehrteilchensysteme, Zustände, Hilberträume/Operatoren
\item Nützlicher Formalismus? $\rightarrow$ Erzeuger/Vernichter, Zweite Quantisierung, Quantenfeldtheorie
\end{itemize}

\section{Unterscheidbare Teilchen}

\subsection{Zustände}

\newcommand{\bra}[1]{\langle #1 |}
\newcommand{\sprod}[2]{\langle #1 | #2 \rangle}

Basiszustände für zwei Teilchen ohne Wechselwirkung:\par 
Basis für Teilchen 1: $\ket{n^{(1)}}$, $n=1,2,\ldots$\\
Basis für Teilchen 2: $\ket{m^{(2)}}$, $m=1,2,\ldots$\par 
$\Rightarrow$ vernünftige Annahme: Basiszustände für Teilchen 1+2:\\ $\ket{n^{(1)}}\ket{m^{(2)}}$, $n,m=1,2,\ldots$ ``Produktzustände''\par 

Hilbertraum: Teilchen 1 $\mathcal{H}_1^{(1)}$, Teilchen 2 $\mathcal{H}_1^{(2)}$. (Oberer Index Teilchenindex, Unterer Index Teilchenzahl)\\
Teilchen 1+2: $\mathcal{H}_2=\mathcal{H}_1^{(1)}\otimes\mathcal{H}_1^{(2)}$ (\textit{Produktraum})
\begin{itemize}
\item $\mathcal{H}_2$ enthält sowohl Produktzustände (\textit{separabel}), z.B. $$\ket{1^{(1)}}\ket{2^{(2)}}$$ oder $$\left(\ket{1^{(1)}}+\ket{{3}^{(1)}}\right)\left(\ket{5^{(2)}}+\ket{7^{(2)}}\right)$$\\
aber auch \textit{verschränkte Zustände} (``entangled''), z.B. $$\frac{\ket{1^{(1)}}\ket{1^{(2)}}-\ket{2^{(1)}}\ket{2^{(2)}}}{\sqrt{2}}$$
\end{itemize}

\paragraph{Skalarprodukte} ``offensichtlich'' übertragen
$$\left(\bra{\psi^{(1)}}\bra{\phi^{(2)}}\right)\left(\ket{\psi'^{(1)}}\ket{\phi'^{(2)}}\right):=\left(\sprod{\psi^{(1)}}{\psi'^{(1)}}\right)\cdot \left(\sprod{\phi^{(2)}}{\phi'^{(2)}}\right)$$
Schreibweise: $\ket{\psi^{(1)}}\ket{\phi^{(2)}}=\ket{\psi ,\phi}$, 
Ortsraum-Wellenfunktion: $\ket{x_1^{(1)}}\ket{x_2^{(2)}}=\ket{x_1,x_2}$\par 
$$\sprod{x_1,x_2}{\psi}=:\psi (x_1,x_2)$$

\subsection{Observablen/Operatoren}

Observable: $A_2$: hermitesche Operatoren auf $\mathcal{H}_2$
\begin{itemize}
\item Observablen, die nur ein Teilchen betreffen: entsprechen $A_1^{(1)}$:
$$\bra{\psi^{(1)}}\bra{\phi^{(2)}} A_2^{(1)} \ket{\psi'^{(1)}}\ket{\phi'^{(2)}} = \erwop{\psi^{(1)}}{A_1^{(1)}}{\psi'^{(1)}}\cdot\sprod{\phi^{(2)}}{\phi'^{(2)}}$$
$$A_2^{(1)}=A_1^{(1)}\otimes\mathbf{1}$$
\item Analog: Observable betrifft nur Teilchen 2:
$$B_2^{(2)}=\mathbf{1}\otimes B_1^{(2)}$$
\end{itemize}
Allgemeine Observable: keine Produktstruktur nötig! $\rightarrow$ WW zwischen Teilchen!\par 
Bsp. Coulomb-Potenzial zwischen Teilchen 1 und 2:
$$\erwop{\psi^{(1)},\phi^{(2)}}{V_2}{\psi^{(1)},\phi^{(2)}}=\int\dif^3x_1\,\dif^3x_2\frac{-\alpha}{|\vf{x}_1-\vf{x}_2|}|\psi (\vf{x}_1)|^2 |\phi (\vf{x}_2)|^2$$
$$\Longrightarrow V_2 = \int\dif^3x_1\,\dif^3x_2 (\ket{\vf{x}_1^{(1)}}\bra{\vf{x}_1^{(1)}}\otimes \ket{\vf{x}_2^{(2)}}\bra{\vf{x}_2^{(2)}})\frac{-\alpha}{|\vf{x}_1-\vf{x}_2|}$$

Hamiltonian:
$$H_2=H_1^{(1)}\otimes\mathbf{1} + \mathbf{1}\otimes H_1^{(2)}+H_{WW}^{(1,2)}$$

\section{Identische/Ununterscheidbare Teilchen}

\subsection{Prinzipien}

Exp: Pauliprinzip, Fermigas, Gibbs Paradoxon (keine Mischungsentropie wenn gleichatomige Gase gemischt werden)\par 

Bisheriger Formalismus reicht nicht aus, da die bisherigen Zustände zu detailliert sind (Zuordnung des Teilchenindexes ist überflüssig)\par 

\paragraph{Fundamentale Beobachtungstatsache / Postulat} Zustände eines Systems ununterscheidbarer Teilchen sind gegenüber Vertauschung der Teilchenindizes generell symmetrisch oder generell antisymmetrisch.

\paragraph{Bosonen} (Spin ganzzahlig) $\ket{...\psi, \phi ...} = +\ket{...\phi, \psi ...}$
\paragraph{Fermionen} (Spin halbzahlig) $\ket{...\psi, \phi ...} = -\ket{...\phi, \psi ...}$

\subsection{Zustände}

\newcommand{\hil}{\mathcal{H}}

$N$-Teilchen Hilbertraum $\hil_N=\hil_1\otimes\ldots\otimes\hil_N$\par 

Permutationsoperator $P_{ij}$:
$$P_{ij}\ket{...\psi^{(i)}...\phi^{(j)}...}=\ket{...\phi^{(i)}...\psi^{(j)}...}$$
$(P_{ij})^2=\mathbf{1}$, $(P_{ij})^\dagger = P_{ij}$\par 
(Anti-)symmetrischer Hilbertraum:
\begin{itemize}
\item $\hil_N^{(+)}$ Teilchenraum mit $P_{ij}\ket{\phi^{(+)}}=\ket{\phi^{(+)}}$
\item $\hil_N^{(-)}$ Teilchenraum mit $P_{ij}\ket{\phi^{(-)}}=-\ket{\phi^{(-)}}$
\end{itemize}

\paragraph{Bsp. 2 Bosonen} 
\begin{itemize}
\item Basis $\hil_1$: $\ket{n}$
\item Basis $\hil_2$: $\ket{n^{(1)},m^{(2)}}$
\item Basis $$\hil_2^{(+)}: \frac{\ket{n^{(1)}m^{(2)}}+\ket{m^{(1)}n^{(2)}}}{\sqrt{2}}=:\ket{n,m}^{(+)}$$
\end{itemize}
\paragraph{Bsp. 2 Fermionen} (Vernachlässige Spin)
\begin{itemize}
\item Basis $$\hil_2^{(-)}: \frac{\ket{n^{(1)}m^{(2)}}-\ket{m^{(1)}n^{(2)}}}{\sqrt{2}}=:\ket{n,m}^{(-)}$$
\end{itemize}
\paragraph{Bsp. 2 Fermionen} (Mit Spin)
\begin{itemize}
\item Basis $\hil_1$: $\ket{n^\uparrow}$, $\ket{n^\downarrow}$
\item Basis $\hil_2$: Vier Kombinationen von $n$ und $m$ für verschiedene Spineinstellungen oder äquivalent:
$$\ket{n^{(1)}m^{(2)}}\otimes\ket{\uparrow\uparrow}, \ket{n^{(1)}m^{(2)}}\otimes\left(\frac{\ket{\uparrow\downarrow}+\ket{\downarrow\uparrow}}{\sqrt{2}}\right), \ket{n^{(1)}m^{(2)}}\otimes\ket{\downarrow\downarrow}, \ket{n^{(1)}m^{(2)}}\otimes\left(\frac{\ket{\uparrow\downarrow}-\ket{\downarrow\uparrow}}{\sqrt{2}}\right)$$
\item $\hil_2^{(1)}$:
$$\frac{\ket{n^{(1)}m^{(2)}}-\ket{m^{(1)}n^{(2)}}}{\sqrt{2}} \otimes\left\{\begin{matrix}
\ket{\uparrow\uparrow} \\ \frac{\ket{\uparrow\downarrow}+\ket{\downarrow\uparrow}}{\sqrt{2}} \\ \ket{\downarrow\downarrow}
\end{matrix}\right.$$
$$ \frac{\ket{n^{(1)}m^{(2)}}+\ket{m^{(1)}n^{(2)}}}{\sqrt{2}}\otimes \frac{\ket{\uparrow\downarrow}-\ket{\downarrow\uparrow}}{\sqrt{2}}$$
Folgerung: Selber Ort unmöglich, wenn Spins gleich.
\end{itemize}

\paragraph{Frage:} Sind obige Zustände eine Basis? Wie konstruiert man allgemein eine Basis von $\hil_N^{(\pm )}$?

\paragraph{Antwort:} Nimm Basis aus Produktzuständen von $\hil_N$, symmetrisiere/antisymmetrisiere jedes Basiselement (wie für $N=2$ genutzt).\par 

Def. \textit{Symmetrisierungsoperator}
$$S_N^{(\pm )} := \frac{1}{N!}\sum_{\mathcal{P}} (\pm 1)^\mathcal{P}\mathcal{P}$$
mit Permutationsoperator $\mathcal{P}$ (beliebiges Produkt von $P_{ij}$-Operatoren).\par 
Es gilt: 
\begin{itemize}
\item[(a)] $$P_{ij}S_N^{(\pm )} = \frac{1}{N!}\sum_\mathcal{P}(\pm )^\mathcal{P} P_{ij}\mathcal{P} = \pm S_N^{(\pm )}=S_N^{(\pm )}P_{ij}$$
\item[(b)] $$\mathcal{P} S_N^{(\pm )} = (\pm 1)^\mathcal{P} S_N^{ (\pm )}$$
\item[(c)] $S_N^{(\pm )}$ ist hermitesch.
\item[(d)] $$S_N^{(\pm )} S_N^{(\pm )} = S_N^{(\pm )}$$
\end{itemize}
$S_N^{(\pm )}$ sind hermitesche Projektionsoperatoren auf $\hil_N^{(\pm )}$.

\paragraph{Konstruktion einer Basis}
\begin{itemize}
\item Nimm Basis von $\hil_N$ aus Produktzuständen: $\ket{n_1^{(1)}n_2^{(2)}\cdots n_N^{(N)}}$
\item Def. $S_N^{(\pm )}\ket{n_1^{(1)}n_2^{(2)}\cdots n_N^{(N)}} =: \ket{n_1\cdots n_N}^{(\pm )}$
\item Nimm beliebigen Zustand $\ket{\psi_N^{\pm}} \in \mathcal{H}_N^{\pm}$
\begin{align*}
    \implies \ket{\psi_N^{\pm}} &\in \mathcal{H}_N, \\
    P_{i j} \ket{\psi_N^{\pm}} &= \pm \ket{\psi_N^{\pm}} \implies S_N^{\pm} \ket{\psi_N^{\pm}} = + \ket{\psi_N^{\pm}} \\
    \implies \ket{\psi_N^{\pm}} &=  S_N^{\pm} \left(\int \ket{n_1^1 \ldots n_N^N}\bra{n_1^1 \ldots n_N^N}\right) \left(S_N^{\pm}\right)^\dagger \ket{\psi_N^{\pm}}\\
    &= \sum\int \underbrace{\ket{n_1 \ldots n_N}}_{\text{Basiszustände}} \underbrace{\sprod{n_1 \ldots n_N}{\psi_N^{\pm}}}_{\text{Koeffizienten}}
\end{align*}
In der Tat stimmt die obige Antwort und die Basis ist durch die obige Gleichung gegeben.
\item Normierung: per Konstruktion gilt die Vollständigkeitsrelation
$$\mathbf{1}_{\mathcal{H}^{\pm}_N} = \int \ket{n_1 \ldots n_N}^{\pm}\bra{n_1\ldots n_N}^{\pm}$$
wegen $S_N^{\pm} S_N^{\pm} = S_N^{\pm}$ aber anders normiert als im 2-Teilchen-Beispiel.
\end{itemize}

\subsubsection{Observablen, weitere Motivation für Symmetrisierungspostulate}

System aus $N$ identischen Teilchen, $A_N$ sei sinnvolle Observable, $\ket{\psi_N}$ und $\ket{\phi_N}$ seien sinnvolle Zustände.
\begin{itemize}
    \item $\ket{\psi_N}$ und $P_{i j} \ket{\psi_N}$ ``bedeuten das selbe''
    \item Sinnvolle Annahme für die Observablen
    \begin{align*}
        \bra{\psi_N} A_N \ket{\psi_N} &= \bra{\psi_N} P_{ i j} A_N P_{i j} \ket{\psi_N} \\
        \implies A_N &= P_{i j} A_N P_{i j} \implies [A_N, P_{i j}] = 0
    \end{align*}
    für jede sinnvolle Observable auf dem Raum der sinnvollen Zustände.
    \item Spezielle Observable $A_N := \ket{\psi_N} \bra{\psi_N}$ ergibt $$P_{i j} A_N \ket{\psi_N} = A_N P_{i j} \ket{\psi_N} \iff  (P_{i j} \ket{\phi_N}) \sprod{\phi_N}{\psi_N} = \ket{\phi_N} \bra{\phi_N} P_{i j} \ket{\psi_N}$$
    Woraus schließlich folgt dass $$\iff P_{i j} \ket{\phi_N} = \lambda \ket{\phi_N} \implies \lambda=\pm 1$$.
    \item Das Symmetrisierungspostulat wird hierdurch suggestiert.  Das Postulat selbst ist noch etwas stärker, denn es besagt, dass für jede Teilchensorte genau nur ein Vorzeichen erlaubt ist.
    \item Beispiele für Observablen
    \begin{table}[hb!]
        \centering
        \begin{tabular}{|l|l|}
            \hline
            2 Teilchen unterscheidbar & $\vec{x}_1, \vec{x}_2$; $\vec{p}_1, \vec{p}_2$; $H = \frac{\vec{p}_1^2}{2m} + \frac{\vec{p}_2^2}{2m}$; $\vec{L}_1, \vec{L}_2, \vec{L}_{\text{ges}}$ \\
            & $H$ sinnvoll, $\vec{x}_1, \vec{x}_2$ nicht sinnvoll \\\hline
            2 Teilchen ununterscheidbar & $\vec{x}_1 - \vec{x}_2$ nicht sinnvoll, \\
            & aber $\vec{x}_1 + \vec{x}_2$, $(\vec{x}_1 - \vec{x}_2)^2$, $|\vec{x}_1 - \vec{x}_2|^2$, $\vec{x}_1 \vec{x}_2$ sinnvoll \\
            \hline
        \end{tabular}
    \end{table}
    \item Vollständiges System kommutierender Observablen ist kompliziert.
    \item Oft möglich: Rechnen nicht direkt mit $\mathcal{H}_N^{\pm}$ sondern in $\mathcal{H}$ und mit einzelnen Observablen und am Ende: Spezialisieren/Einschränken auf symmetrische bzw. antisymmetrische Zustände.
\end{itemize}

\section{Einfache Anwendungen}

\subsection{Grund- und angeregte Zustände}

$N$ Teilchen ohne Wechselwirkung;
\begin{enumerate}
    \item  Unterscheidbar: z.B. die Elektronen im He-Atom
    \item Fermionen: z.B. Elektronen im Metall
    \item Bosonen: mehrere $H$-Atome
\end{enumerate}

Beispiel: Alle Teilchen im Potential mit möglichen Energien $e_1$, $e_2$, $e_3, \ldots$ und Eigenzuständen $\ket{1}$, $\ket{2}$, $\ket{3}, \ldots$.
\begin{enumerate}
    \item Grundzustand $\ket{1^1, 1^2}$, $E = 2 e_1$

    1. Angeregter Zustand $\ket{1^1 2^2}$ oder $\ket{2^1 1^2}$, $E = e_1 + e_2$ 2-fach entartet.

    \item Grundzustand N: $\ket{1, 2, \ldots, N}^{-}$, $E = e_1 + e_2 + \ldots + e_N$ nicht entartet.
    $e_N$ ist die maximale besetzte Energie im Grundzustand, genannt Fermienergie
    
    1. Angeregter Zustand: $\ket{1,2, \ldots, N-1, N+1}^{-}$ nicht entartet! $\Delta E = e_{N+1} - e_N$
    \item Grundzustand N: $\ket{1, 1, \ldots, 1}^+$, $E = N e_1$
    
    1. Angeregter Zustand: $\ket{2, 1, \ldots, 1}$ nicht entartet! $\Delta E = e_2 - e_1$
\end{enumerate}

\subsection{Direkter Prozess vs. Austauschterm}

Zwei Teilchen: $\ket{\psi}$, $\ket{\phi}$ $\longrightarrow$ Prozess $\longrightarrow$ $\ket{n}$, $\ket{m}$\par 

Anfangszustand $\ket{i}$ $\longrightarrow$ Endzustand $\ket{f}$. Frage: Was ist die Wahrscheinlichkeit?
$$P_{i\rightarrow f} = |A_{i\rightarrow f}|^2$$

Unterscheidbar: (entweder nur links oder nur rechts):
\begin{itemize}
\item ``direkt'': $$A^d_{i\rightarrow f} = \sprod{n^{(1)}m^{(2)}}{\psi^{(i)}\phi^{(2)}}=\sprod{n}{\psi}\sprod{m}{\phi}$$
\item ``Austauschterm'': $$A^a_{i\rightarrow f}=\sprod{m^{(1)}n^{(2)}}{\psi^{(1)}\phi^{(2)}}=\sprod{m}{\psi}\sprod{n}{\phi}$$
\item Gesamtwahrscheinlichkeit: ``entweder $\bra{nm}$ oder $\bra{mn}$''
$$P_{i\rightarrow f}=|A^d_{i\rightarrow f}|^2+|A^a_{i\rightarrow f}|^2$$
\end{itemize}

Bosonen:
$$\ket{i}=\frac{\ket{\psi\phi}+\ket{\phi\psi}}{\sqrt{2}}\qquad\qquad\ket{f}=\frac{\ket{nm}+\ket{mn}}{\sqrt{2}}$$
$$A_{i\rightarrow f} = \sprod{f}{i} = \frac{1}{2}\left(\sprod{\psi}{n}\sprod{\phi}{m}+\sprod{\phi}{n}\sprod{\psi}{m}\right)\cdot 2 = A^d_{i\rightarrow f}+A^a_{i\rightarrow f}$$
$$P_{i\rightarrow f} = \left|A^d_{i\rightarrow f}+A^a_{i\rightarrow f}\right|^2$$

Fermionen (Analog):
$$P_{i\rightarrow f} = \left|A^d_{i\rightarrow f}-A^a_{i\rightarrow f}\right|^2$$

Spezialfall $n=m$: (Beide Teilchen gehen in den selben Zustand über)
\begin{itemize}
\item Fermionen: $P_{i\rightarrow f}=0$
\item Bosonen: $P_{i\rightarrow f}=2|A^d_{i\rightarrow f}|^2$ (Doppelt so groß wie bei unterscheidbaren Teilchen)
\end{itemize}

\subsection{Wasserstoffmolekül $\mathrm{H}_2$}

Chemische Bindung, gewisser Atomabstand $R$ minimiert die Energie. Austauschwechselwirkung sehr wichtig $\rightarrow$ Orts-Wellenfunktion.\par 

Im Grundzustand: Orts-Wellenfunktion symmetrisch, Spin antisymmetrisch.\par 

Annahme/Näherung: Kerne fixiert im Abstand $R$, Positionen der Kerne $a$ und $b$, Elektronen $1$ und $2$

$$H=\frac{\vec{p}_1^2}{2m}+\frac{\vec{p}_2^2}{2m}-\alpha\left(\frac{1}{r_{1a}}+\frac{1}{r_{2a}}+\frac{1}{r_{1b}}+\frac{1}{r_{2b}}-\frac{1}{r_{12}}-\frac{1}{R}\right)$$

$$H=H_{1,a}+H_{2,b}-\alpha\left(\frac{1}{r_{1b}}+\frac{1}{r_{2a}}-\frac{1}{r_{12}}-\frac{1}{R}\right)$$

H-Atom-Zustände, Struktur der 2-Elektron-Zustände.\par 

\paragraph{Erinnerung H-Atom:}\mbox{}\par 
Quantenzahlen $n, l, m$: $\psi_{nlm}\sim R_{nl}(r)Y_{lm}(\theta ,\varphi )$\par 
Grundzustand:
$$\psi_{100}=\frac{2}{\sqrt{4\pi}}a_B^{-\frac{3}{2}}e^{-\frac{r}{a_B}}$$
(Bohrscher Radius $a_B=\frac{1}{\alpha m}$).\par 
Energien: $E_1 = -\frac{\alpha^2 m}{2}$, $E_n = \frac{E_1}{n^2}$ (Zusätzlich ungebundene Zustände mit $E>0$)\par 

\paragraph{H-Atom mit Proton im Punkt $\vec{R}_a$}\mbox{}\par 

Selbe Energie-EW, Eigenzustände: $\psi_a(\vec{x})=\psi_{\mathrm{Ursprung}}(\vec{x}-\vec{R}_a)$

\paragraph{2-Elektron-Zustände} 2 Basen von 1-T.-Zuständen um Proton $a$ $\ket{\psi_a,nlm}$ und um Proton $b$ $\ket{\psi_b,nlm}$\par 

Basis von 2-Teilchen-Zuständen (antisymmetrisch): $\hil_2^{(-)}$:
$$\ket{\psi_{a,nlm},\psi_{b,n'l'm'}}^{(-)}\otimes\left\{\begin{matrix}
\ket{\uparrow\uparrow} \\ \frac{\ket{\uparrow\downarrow}+\ket{\downarrow\uparrow}}{\sqrt{2}} \\ \ket{\downarrow\downarrow}
\end{matrix}\right.$$
$$\ket{\psi_{a,nlm},\psi_{b,n'l'm'}}^{(+)}\otimes \left(\frac{\ket{\uparrow\downarrow}-\ket{\downarrow\uparrow}}{\sqrt{2}}\right)$$

Spinoperator kommutiert mit Hamiltonian, des Weiteren: $[\vec{S}^2,S_z]=0$. Simultane Eigenzustände:
$$\ket{SM}\qquad \qquad \vec{S}^2\ket{SM}=S(S+1)\ket{SM}\qquad S_z\ket{SM}=M\ket{SM}$$
Spin-Notation:
\begin{align*}
\ket{1,1}&:=\ket{\uparrow\uparrow}\\
\ket{1,0}&:=\frac{\ket{\uparrow\downarrow}+\ket{\downarrow\uparrow}}{\sqrt{2}}\\
\ket{1, -1}&:=\ket{\downarrow\downarrow} \\
\ket{0, 0}&:= \frac{\ket{\uparrow\downarrow}-\ket{\downarrow\uparrow}}{\sqrt{2}}
\end{align*}

Damit Basis:\par 
Ort antisymmetrisch, Spin $S=1$: $\ket{\psi_{a,nlm},\psi_{b,n'l'm'}}^{(-)}\otimes\ket{1,M}$\\
Ort symmetrisch, Spin $S=0$: $\ket{\psi_{a,nlm},\psi_{b,n'l'm'}}^{(+)}\otimes\ket{0,0}$\par 

Idee zur Lösung des H$_2$-Moleküls:
\begin{itemize}
\item Ziel: Grundzustandsenergie? Optimaler Abstand R?
\item Annahme/Näherung: obigen Basiszustände sind Eigenzustände des vollen Moleküls (d.h. WW klein, H-Atome nur wenig beeinflusst)
\item Variationsprinzip: Ansatz sinnvoller Zustände $\ket{\psi_{\text{sinnvoll}}}$
$$E_{var}=\frac{\erwop{\psi_{\text{sinnvoll}}}{H}{\psi_{\text{sinnvoll}}}}{\sprod{\psi_{\text{sinnvoll}}}{\psi_{\text{sinnvoll}}}}$$
Auf jeden Fall: $E_{var}\geq E_{\text{Grundzustand}}$ (Gleichheit bei guter Wahl)
\end{itemize}
\paragraph{Heitler-London-Näherung}\mbox{}\par 

Wähle $\ket{\psi_{\text{sinnvoll}}}:=\ket{\psi}^{(\pm )}=\ket{\phi_a,\phi_b}^{(\pm )}\otimes\ket{SM}$, wobei $\phi_a$ und $\phi_b$ die Grundzustände bezüglich der einzelnen H-Atome sind. Bei folgenden Matrixelementen: $\sprod{SM}{SM}=1$ trägt nicht weiter bei $\rightarrow$ ab jetzt nur noch Ortsraum betrachten.

$$\sprod{\vec{x}}{\phi_{a,b}}=\frac{2}{\sqrt{4\pi}}a_B^{-\frac{3}{2}}e^{-\frac{|\vec{x}-\vec{R}_{a,b}|}{a_B}}$$

Längere Rechnung ($\psi^{(\pm )}$ einsetzen und bekannte Skalarproduktrelationen, Normierung ausnutzen und beim Matrixelement auf Eigenzustände von Teilen des Hamiltonians achten):
\begin{itemize}
\item[(a)] $$\sprod{\psi^{(\pm )}}{\psi^{(\pm )}} = 1\pm |L_{ab}|^2$$
mit $L_{ab}=\sprod{\phi_a}{\phi_b}=\int\dif^3x\:\phi_a(\vec{x})\phi_b(\vec{x})$ (Überlapp).
\item[(b)] $$\erwop{\psi^{(\pm )}}{H}{\psi^{(\pm )}} = \erwop{\phi_a^{(1)}\phi_b^{(2)}}{H}{\phi_a^{(1)}\phi_b^{(2)}} \pm \erwop{\phi_a^{(1)}\phi_b^{(2)}}{H}{\phi_b^{(1)}\phi_a^{(2)}}$$
Diagonalterm:
$$\erwop{\phi_a^{(1)}\phi_b^{(2)}}{H}{\phi_a^{(1)}\phi_b^{(2)}}=2E_1+C_{ab}$$
mit Coulomb-Zusatzenergie 
\begin{align*}
C_{ab}=\frac{\alpha}{R}&-\alpha\int\dif^3x\: |\phi_a(\vec{x})|^2\frac{1}{|\vec{x}-\vec{R}_b|} \\
&- \alpha\int\dif^3x\: |\phi_b(\vec{x})|^2\frac{1}{|\vec{x}-\vec{R}_a|} \\
&+\alpha\int\dif^3x_1\:\dif^3x_2\frac{|\phi_a(\vec{x}_1)|^2|\phi_b(\vec{x}_2)|^2}{|\vec{x}_1-\vec{x}_2|}
\end{align*}
Off-Diagonalterm:
$$\erwop{\phi_a^{(1)}\phi_b^{(2)}}{H}{\phi_b^{(1)}\phi_a^{(2)}}= 2E_1|L_{ab}|^2+A_{ab}$$
mit Austauschterm
\begin{align*}
A_{ab}=\frac{\alpha}{R}|L_{ab}|^2 &-\alpha L_{ab}^*\int\dif^3x\: \frac{\phi^*_a(\vec{x})\phi_b(\vec{x})}{|\vec{x}-\vec{R}_a|} \\
&-\alpha L_{ab}\int\dif^3x\: \frac{\phi_a(\vec{x})\phi_b^*(\vec{x})}{|\vec{x}-\vec{R}_b|} \\
&+\alpha \int\dif^3x_1\:\dif^3x_2\frac{\phi_a^*(\vec{x}_1)\phi_b(\vec{x}_1)\phi_b^*(\vec{x}_2)\phi_a(\vec{x}_2)}{|\vec{x}_1-\vec{x}_2|}
\end{align*}
Damit
$$\boxed{E_{var}^{(\pm )}=2E_1 + \frac{C_{ab}\pm A_{ab}}{1\pm |L_{ab}|^2}}$$
numerisch ausrechnen!\par 
Stabile Bindung? Für welches R?


\end{itemize}

\end{document}
































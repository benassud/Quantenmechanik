\documentclass[11pt,a4paper]{report}
\usepackage[utf8]{inputenc}
\usepackage[german]{babel}
\usepackage{amsmath}
\usepackage{amsfonts}
\usepackage{amssymb}
\usepackage{slashed}
\usepackage{xfrac}
\usepackage{amsbsy}
\usepackage[left=2cm,right=2cm,top=2cm,bottom=2cm]{geometry}
\usepackage{xcolor}
\usepackage{hyperref}
\hypersetup{hidelinks}

\setlength{\parindent}{0cm}
\setlength{\parskip}{0.2cm}

\renewcommand{\baselinestretch}{1.2}

\renewcommand{\vec}{\boldsymbol}

\newcommand{\norm}[1]{\left\lVert #1 \right\rVert}

\title{Skript QT2}

\setcounter{chapter}{-1}
\begin{document}
\maketitle

\tableofcontents

\chapter{Grundstruktur der Quantenmechanik}

\section{Postulate}

\textbf{Essenz:} Doppelspaltexperiment / Stern-Gerlach-Experiment\par 
\textbf{Zustand:} eindeutig / maximal präpariertes physikalisches System, reproduzierbares Verhalten, eindeutige Zeitentwicklung. Beschreibung durch $\left|\psi\right\rangle$ eines Hilbertraums. Linearkombinationen erlaubt!\par 
\textbf{Observablen:} Operatoren $\hat{A}$ (hermitesch, da reelle Eigenwerte $\leftrightarrow$ mögliche Messwerte)\par 
\textbf{Wahrscheinlichkeit:} Für ein Messergebnis $a_n$ ist die Wahrscheinlichkeit $\left|\left\langle a_n|\psi\right\rangle\right|^2$ (normierte Zustände).\par 
\textbf{Erwartungswert:} (Korrollar) $\langle\psi|\hat{A}|\psi\rangle$\par 
\textbf{Zeitentwicklung:} $\hat{H}$ (Hamiltonoperator), $\hat{H}$ sei nicht expl. zeitabh.
$$i\hbar\frac{\mathrm{d}}{\mathrm{d}t}\langle\psi_1|\hat{A}|\psi_2\rangle = \langle\psi_1|[\hat{A},\hat{H}]|\psi_2\rangle$$
\textbf{Schrödinger-Bild}
$$i\hbar\frac{\mathrm{d}}{\mathrm{d}t}|\psi (t)\rangle = \hat{H}|\psi (t)\rangle$$
\textbf{Heisenberg-Bild}
$$|\psi_H\rangle = e^{i\hat{H}t}|\psi (t)\rangle$$
$$\hat{A}_H(t)=e^{i\hat{H}t}\hat{A}e^{-i\hat{H}t}$$
$$i\hbar\frac{\mathrm{d}}{\mathrm{d}t}\hat{A}_H(t)=[\hat{A}_H(t),\hat{H}]$$

\section{Ortsraum, Teilchen in 1D}

Operatoren $\hat{x}$, $\hat{p}$, $[\hat{x},\hat{p}]=i\hbar$.\par 
EZe: $|x\rangle$, $|p\rangle$ (bilden jeweils Basis)\par 

Wellenfunktionen: $\psi (x) :=\langle x|\psi\rangle$, $\tilde{\psi}(p):=\langle p|\psi\rangle$

\chapter{Relativistische Quantenmechanik}

\newcommand{\dt}{\frac{\mathrm{d}}{\mathrm{d}t}}
\newcommand{\dif}{\mathrm{d}}
\newcommand{\vf}[1]{\mathbf{#1}}


\section{Kontinuierliche Symmetrien (Bsp. Rotationsinvarianz)}

Frage: Was ist Drehimpuls?

\subsection{Drehungen in 3D}

($\rightarrow$ Liegruppe $SO(3)$)\par 

Aktive Drehung: Bsp. $\vf{v}'=R_z(\theta )\vf{v}$ (Drehung um Winkel $\theta$ um $z$-Achse)\par 

Infinitesimale Drehungen, $\theta = \varepsilon\rightarrow 0$:
$$R_z(\varepsilon)=\mathbf{1} -i\varepsilon\begin{pmatrix}
0 & -i & 0 \\
i & 0 & 0 \\
0 & 0 & 0
\end{pmatrix} = \mathbf{1} - i\varepsilon \ell_z$$
$$\ell_z = \begin{pmatrix}
0 & -i & 0 \\
i & 0 & 0 \\
0 & 0 & 0  
\end{pmatrix}\qquad\ell_x = \begin{pmatrix}
0 & 0 & 0 \\
0 & 0 & -i \\
0 & i & 0  
\end{pmatrix}\qquad\ell_y = \begin{pmatrix}
0 & 0 & i \\
0 & 0 & 0 \\
-i & 0 & 0  
\end{pmatrix}\qquad (\ell_k)_{i,j}=-i\varepsilon_{ijk}$$

``Generatoren der zugehörigen Lie-Algebra"\par 

Charakteristische Kommutatorrelation: $[\ell_i, \ell_j]=i\varepsilon_{ijk}\ell_k$\par 

Endliche Drehungen: $R_z(\theta ) = \exp\:(-i\theta\ell_z)$

\subsection{Darstellungen}

Eine Darstellung einer Gruppe ist eine Zuordnung: $R\mapsto D(R)$ = Matrix / linearer Operator, mit 
$$D(R_1R_2) = D(R_1)D(R_2)$$
Physikalische Idee: Viele physikalische Größen $\rightarrow$ angeben, wie sie sich unter Drehungen verhält.
\begin{itemize}
\item Impuls: $\vf{p}\longmapsto \vf{p}'=R\vf{p}$
\item Energie: $E\longmapsto E' = E=D(R)E$ mit $\forall R: D(R)=1$
\item Ladung: $Q\longmapsto Q' = Q$
\item Dichte: $\rho \longmapsto \rho ' :\rho'(R\vf{x})=\rho (\vf{x})$
\item Quantenzustand $|\psi\rangle\longmapsto |\psi '\rangle = \hat{D}(R)|\psi\rangle$
\end{itemize}

Generatoren für Darstellungen: $\theta=\varepsilon\rightarrow 0$\par 
$$D(R_z(\varepsilon ))=\vf{1} - i\varepsilon J_z\qquad\text{(Analog für x, y)}$$
mit Operatoren $J_x, J_y, J_z$ wie $D(R_z(\varepsilon ))$, diese sind spezifisch für die Darstellung.
$$D(R_z(\theta ))=\exp\:(-i\theta J_z)$$
$$[J_i, J_j] = i\varepsilon_{ijk}J_k$$
\emph{Die Generatoren jeder Darstellung erfüllen dieselben Vertauschungsrelationen.}

\subsection{Drehungen in der Quantenmechanik}

\newcommand{\ket}[1]{|#1\rangle}
\newcommand{\braket}[2]{\langle #1|#2\rangle}
\newcommand{\erwop}[3]{\langle #1|#2|#3\rangle}

Darstellung von Drehungen:
$$\hat{D}(R_k(\theta )):\ket{\psi}\mapsto\ket{\psi'}=\hat{D}(R_k(\theta ))\ket{\psi}$$
Gruppenstruktur:
$$\hat{D}(R_1R_2)=\hat{D}(R_1)\hat{D}(R_2)$$
Falls Symmetrie:
$$\braket{\psi '}{\phi '} = \braket{\psi}{\phi} \Leftrightarrow\erwop{\psi}{\hat{D}^\dagger\hat{D}}{\phi}$$
$\hat{D}(R)$ ist ein \textit{unitärer} Operator. $[\hat{D}(R), H]=0$.\par 

Infinitesimale Drehung:
$$\hat{D}(R_k(\varepsilon ))=\vf{1}-i\varepsilon\hat{J}_k$$
Falls Symmetrie:
$$[\hat{J}_k, \hat{H}] = 0\qquad [\hat{J}_i, \hat{J}_j]=i\varepsilon_{ijk}\hat{J}_k$$
Per Definition: $\hat{\vf{J}}$ is Drehimpuls dieser Quantentheorie.\par 

Konsequenzen bei solchen $\hat{\vf{J}}$-Operatoren: (QT1)
$$[\hat{J}_z, \hat{\vf{J}}]=0\qquad \hat{J}_\pm =\hat{J}_x\pm i\hat{J}_y$$
Mögliche Eigenzustände: $\ket{j, m}$ mit $j = 0, \frac{1}{2}, 1, \frac{3}{2}, \ldots$ und $m=-j, \ldots , j$\par 

Einfachste nicht-triviale Darstellung: $j=\frac{1}{2}$, d.h. 2-Zustandssystem $\ket{\pm}:=\ket{j=\frac{1}{2},m=\pm\frac{1}{2}}$.
$$\ket{\psi}=\psi_+\ket{+}+\psi_-\ket{-}$$
$$\psi\overset{R_k(\theta )}{\longmapsto}\psi' = \left(\vf{1}-i\theta\frac{\sigma_k}{2}\right)\psi$$
mit Pauli-Matrizen $\sigma_k$.

\section{Lorentzinvarianz}

\subsection{Lorentzgruppe}

Drehungen: $(t,\vf{r})\longmapsto (t,R(\vf{r}))$\par 
Boosts in x-Richtung:
$$\begin{pmatrix}
t \\ x \\ y \\ z
\end{pmatrix}\longmapsto\begin{pmatrix}
\cosh\beta & \sinh\beta & 0 & 0 \\
\sinh\beta & \cosh\beta & 0 & 0 \\
0 & 0 & 1 & 0 \\
0 & 0 & 0 & 1
\end{pmatrix}\begin{pmatrix}
t \\ x \\ y \\ z
\end{pmatrix}$$
Generatoren: $\ell_x, \ell_y, \ell_z$ wie gehabt. Boosts: $\Lambda_x(\beta ) = \vf{1} - i\beta k_x+\mathcal{O}(\beta^2)$
$$k_x=i \begin{pmatrix}
0 & 1 & 0 & 0 \\
1 & 0 & 0 & 0 \\
0 & 0 & 0 & 0 \\
0 & 0 & 0 & 0 
\end{pmatrix}\qquad k_y=i \begin{pmatrix}
0 & 0 & 1 & 0 \\
0 & 0 & 0 & 0 \\
1 & 0 & 0 & 0 \\
0 & 0 & 0 & 0 
\end{pmatrix}\qquad k_z=i \begin{pmatrix}
0 & 0 & 0 & 1 \\
0 & 0 & 0 & 0 \\
0 & 0 & 0 & 0 \\
1 & 0 & 0 & 0 
\end{pmatrix}$$
6 Generatoren:
Vertauschungsrelationen (und zyklisch):
$$[\ell_x, \ell_y]=i\ell_z$$
$$[k_x,k_y]=-i\ell_z$$
$$[\ell_x,k_y]=ik_z$$

\subsection{Darstellungen}

\textbf{Def. Darstellung:} Matrizen/Operatoren $J_i$, $K_i$, mit $[J_x,J_y]=iJ_z$, $[K_x,K_y]=-iJ_z$, $[J_x,K_y]=iK_z$.\par 

Triviale Darstellung: $J_i=0$, $K_i=0$\par 

Spin $\frac{1}{2}$: $J_i=\sigma^i/2$, $K_i=-i\sigma^i/2$. Die Elemente des 2D Darstellungsraumes nennt man \textit{linkshändige Weyl-Spinoren}. (Andere Variante mit $K_i=+i\sigma^i/2$: Elemente sind \textit{rechtshändige Weyl-Spinoren})\par 

\paragraph{Partität/Raumspiegelung $P$:} $\vf{x}\mapsto -\vf{x}$, $\vf{p}\mapsto -\vf{p}$, $\vf{J}\mapsto\vf{J}$, $\vf{K}\mapsto -\vf{K}$. Falls $P$-Transformation genutzt werden soll, sind beide Darstellungen nötig $\Rightarrow$ 4D komplexer Spinorraum aus \textit{Dirac-Spinoren} notwendig.
$$\Psi =\begin{pmatrix}
\psi_\alpha \\ \overline{\psi}^{\dot{\alpha}}
\end{pmatrix}$$
Darstellung für Diracspinoren:
$$J_i=\begin{pmatrix}
\frac{\sigma^i}{2} & 0 \\ 0 & \frac{\sigma^i}{2}
\end{pmatrix}\qquad K_i =\begin{pmatrix}
-i\frac{\sigma^i}{2} & 0 \\ 0 & i\frac{\sigma^i}{2}
\end{pmatrix}$$
Diracspinoren: 4-komponentige komplexe Spinoren. Einfachste Darstellung mit $P$-Transformation.\par 

Lorentztransformationen und Darstellungen:
$${\Lambda^\mu}_\nu = {\delta^\mu}_\nu +{\omega^\mu}_\nu$$
(mit infinitesimalem und antisymmetrischem $\omega^{\mu\nu}$ (wenn beide Indizes oben!), z.B Drehung, Boost)

$$\Lambda = \vf{1} -\frac{i}{2}\omega^{\mu\nu}L_{\mu\nu}$$

mit $L_{ij} = -L_{ji} = \varepsilon_{ijk}\ell_k$ und $L_{i0}=-L_{0i}=k_i$

Für eine Darstellung $S$:
$$\boxed{S(\Lambda ) := \vf{1}-\frac{i}{2}\omega^{\mu\nu}L_{\mu\nu}}$$

\subsection{Diracspinoren und $\gamma$-Matrizen}
$\psi = (\psi_1,\psi_2,\psi_3,\psi_4)$ = komplexer Diracspinor.\par 

\paragraph{Def $\gamma$-Matrizen:} $\{\gamma^\mu ,\gamma^\nu\}=2g^{\mu\nu}\vf{1}$\par 

Weyl-Form:
$$\gamma^0 :=\begin{pmatrix}
0 & \vf{1}_2 \\ \vf{1}_2 & 0 
\end{pmatrix}\qquad\gamma^i \begin{pmatrix}
0 & \sigma^i \\ -\sigma^i & 0
\end{pmatrix}$$
Die Generatoren $\vf{J}$, $\vf{K}$ lassen sich so ausdrücken:
$$S_{\mu\nu}=\frac{i}{4}[\gamma_\mu,\gamma_\nu]$$
Dies reproduziert die Darstellungsmatrix $L_{\mu\nu}$ der Lorentztransformation.\par 

${\gamma^\mu}^\dagger$: ${\gamma^0}^\dagger = \gamma^0$, ${\gamma^i}^\dagger = -\gamma^i = \gamma^0\gamma^i\gamma^0$
$${S^\dagger}_{\mu\nu} = \gamma^0 S_{\mu\nu} \gamma^0$$
$$S^{-1}(\Lambda )=\vf{1}+\frac{i}{2}\omega^{\mu\nu}S_{\mu\nu}=\gamma^0 S^\dagger (\Lambda )\gamma^0$$
Def \textit{Adjungierter Spinor:} $\overline{\psi}:=\psi^\dagger\gamma^0$\par 
Lorentz:
$$\psi\longmapsto S(\Lambda )\psi$$
$$\overline{\psi}\longmapsto \overline{\psi}S^{-1}(\Lambda )$$
$$\overline{\psi}\psi\longmapsto\overline{\psi}\psi$$
$$\overline{\psi}\gamma^\mu\psi\longmapsto {\Lambda^\mu}_\nu\overline{\psi}\gamma^\nu\psi$$
$$S^{-1}(\Lambda )\gamma^\mu S(\Lambda )={\Lambda^\mu}_\nu\gamma^\nu$$

\section{Überblick über relativistische Wellengleichungen}

Welche Gleichungen wären erlaubt durch Lorentzinvarianz?\par 

Notation: \par 
\begin{itemize}
\item 4-Vektoren: $(x^\mu )=(t,\vf{x})$, $(p^\mu)=(E,\vf{p})$
\item Lorentzinvarianten sind Skalarprodukte, z.B. $p^\mu p_\mu = E^2-\vf{p}^2=:m^2$
\item Ableitungen: $\partial_\mu = \left(\frac{\partial}{\partial x^\mu}\right) = (\partial_t,\nabla )$, $\square = \partial_\mu\partial^\mu = \partial_t -\Delta$
\item Elektrodynamik: $j^\mu = (\rho ,\vf{j})$, $\partial_\mu j^\mu = 0$, $A^\mu = (\phi, \vf{A})$, $F^{\mu\nu}=\partial^\mu A^\nu -\partial^\nu A^\mu$\\
Maxwell: $\partial_\mu F^{\mu\nu} = \mu_0 j^\nu$, homogene Gleichung automatisch durch Potentiale erfüllt.\\
Lorentz-Transf.: $x'^\mu ={\Lambda^\mu}_\nu x^\nu$, $j'^\mu (x')={\Lambda^\mu}_\nu j^\nu (x)$
\end{itemize}

\subsection{Klein-Gordon-Gleichung}

$\phi (x)$ sei Skalarfeld ($\phi\mapsto \phi '$ mit $\phi '(x')=\phi (x)$).

$$\boxed{\square \phi (x) + m^2\phi (x) = 0}$$

\paragraph{Interpretation?}
\begin{itemize}
\item Einfachste relativistische Differentialgleichung
\item ``erraten aus QM'' (mit QM Ersetzungsregeln $E\rightarrow i\partial_t$, $\vf{p}\rightarrow -i\nabla$)
\item Nichtrelativistischer Limes: ein Teilchen, $E\approx m + \text{Korrektur}$. Ansatz:
$$\psi (\vf{x},t)=e^{-imt}\psi_{n.r.} (\vf{x},t)$$
$$\Rightarrow \partial_t^2\psi = (-2im\partial_t\psi_{n.r.}-m^2\psi_{n.r.}+\mathcal{O}(\ddot{\psi}))e^{-imt}$$
$$\Rightarrow 2im\partial_t\psi_{n.r.}=-\Delta\psi_{n.r.}$$
\item Klassische Feldgleichung:
$$\mathcal{L}_{KG}=(\partial^\mu \phi^*)(\partial_\mu\phi )-m^2\phi^*\phi$$
Euler-Lagrange:
$$0=\partial_\rho\frac{\partial\mathcal{L}}{\partial (\partial_\rho\phi^*)}-\frac{\partial\mathcal{L}}{\partial\phi^*}$$
\end{itemize}

\paragraph{Rolle als QM Wellengleichung für ein Teilchen in Ortsdarstellung:}
\mbox{}\par
Schrödinger-Gleichung nicht-relativistisch: $i\partial_t\psi = -\frac{\Delta}{2m}\psi$\\
Klein-Gordon-Gleichung: $-\partial_t^2\phi = (-\Delta +m^2)\phi$\par 

Aufenthaltwahrscheinlichkeitsdichte: Suche $(j^\mu )=(\rho ,\vf{j})$ mit Kontinuitätsgleichung $\partial_\mu j^\mu =0$:
$$\phi^* (\square + m^2)\phi -\phi (\square+m^2)\phi^* = 0$$
$$=\partial_\mu [\phi^*\partial^\mu \phi - \phi\partial^\mu\phi^*]$$
Definiere 4-Stromdichte:
$$j^\mu = \frac{i}{2m}\left[\phi^*\partial^\mu\phi -\phi\partial^\mu\phi^*\right]$$
$$\Rightarrow \vf{j}=-\frac{i}{2m}\left[\phi^*\nabla\phi -\phi\nabla\phi^*\right]$$
$$\Rightarrow \rho = \frac{i}{2m}\left[\phi^*\partial_t\phi - \phi\partial_t\phi^*\right]$$
\paragraph{Interpretation}
\begin{itemize}
\item $\rho$ ist nicht positiv definit! $\rho < 0$ möglich! Also kann $\rho$ nicht als Aufenthaltswahrscheinlichkeit interpretiert werden.
\item Lösungen: $\phi\sim e^{-iEt+i\vf{p}\cdot\vf{x}}$: $\rho = \frac{E}{m}>0$, $\rho\sim e^{+iEt-i\vf{p}\cdot\vf{x}}$: $\rho = -\frac{E}{m}<0$: negative Energie möglich!?
\item Idee: KG-Gl. beschreibt zwei Teilchentypen (Teilchen + Antiteilchen) mit entgegengesetzten Ladungen. Interpretiere $\rho$ als elektrische Ladungsdichte.
\end{itemize}

\subsection{Dirac-Gleichung}

$\psi (x)$ sein ``Dirac-Spinorfeld'' d.h. $\psi\mapsto \psi '$ mit $\psi'(x')=S(\Lambda ) \psi (x)$.
$$S(\Lambda ) = \vf{1}_4-\frac{i}{2}\omega^{\mu\nu}S_{\mu\nu}$$
$$S_{\mu\nu} = \frac{i}{4}[\gamma_\mu ,\gamma_\nu ]$$
$$\{\gamma^\mu ,\gamma^\nu\} = 2g^{\mu\nu}\vf{1}_4$$
Dirac-Gleichung:
$$\boxed{ (i\partial_\mu\gamma^\mu - m)\psi =0}$$

\paragraph{Interpretation:}
\begin{itemize}
\item nicht einfachste Differenzialgleichung
\item erraten von Dirac: gewünscht ``Wurzel aus KG-Gleichung'' (Herleitung $\curvearrowright$ Lit.)
\item $\mathcal{L} = \overline{\psi}(i\partial_\mu \gamma^\mu - m)\psi$
\item Adjungierte Dirac-Gl. $i\partial_\mu \overline{\psi}\gamma^\mu + m\overline{\psi} = 0$
$$\Rightarrow \partial_\mu (\overline{\psi}\gamma^\mu\psi )=0$$
\item Def. $j^\mu = \overline{\psi}\gamma^\mu\psi $, $\rho = \psi^\dagger\psi$ ist positiv-definit
\end{itemize}

\paragraph{Vollständige Darstellung der Lorentztransformationen}\mbox{}\par 

$$\psi '(x)=S(\Lambda )\psi (\Lambda^{-1}x) = (\vf{1}-\frac{i}{2}\omega^{\mu\nu}S_{\mu\nu})\psi (x-\omega x)$$
und
$$\psi '=(\vf{1}-\frac{i}{2}\omega^{\mu\nu}\hat{J}_{\mu\nu})\psi$$
($\hat{J}$ Generatoren der Darstellung der Lorentz-Algebra auf dem Fkt.-Raum der Spinorfelder)
$$\Longrightarrow \hat{J}_{\mu\nu}=i(x_\mu\partial_\nu -x_\nu\partial_\mu )+S_{\mu\nu}$$
$$\hat{J}_{\mu\nu}=\hat{L}_{\mu\nu}+S_{\mu\nu}$$
Analog zur KG-Gl. treten Inkonsistenzen auf, wenn man Diracgl. als 1-Teilchen-Theorie auffasst. Die Probleme sind ähnlich aber nicht gleich.

\section{Physik und Lösungen der Diracgleichung}

\subsection{Freie Lösungen, Impuls-/Spin-Eigenzustände}

\newcommand{\dslash}{\slashed{\partial}}
\newcommand{\pslash}{\slashed{p}}

Dirac-Gleichung: $(i\dslash - m)\psi = 0$\\
Gesamt-Drehimpuls: $\hat{J}_{ij}=\hat{L}_{ij}+S_{ij}$. Spin-EZ: $\pm\frac{1}{2}$\par 

Ansatz: $\psi (x)= w(p)e^{\mp ipx}$ (mit $px = p_\mu x^\mu$)
$$\Rightarrow (\pm \pslash - m)w(p)=0$$
Eigenwertgleichung für $\pslash$ !\par 

Beachte: $\pslash^2 = p^\mu\gamma_\mu p^\nu\gamma_\nu = p^\mu p^\nu \gamma_\mu\gamma_\nu = \frac{1}{2}p^\mu p^\nu \{\gamma_\mu ,\gamma_\nu\} = p^2\vf{1}$\par 
D.h. $\pslash$ hat EWe $\pm \sqrt{p^2}$ vermutlich je 2-fach entartet. Nicht-triviale Lösung der EW-Gl. für $p^2=m^2$ $\rightarrow$ Teilchen mit Ruhemasse $m$ beschrieben.

\paragraph{Bezeichnungen der Lösungen}
$$(\pslash - m)u(p,s)=0$$
$$(\pslash + m)v(p,s)=0$$
Beispiel: $p^2 = m^2$, $(p^\mu )=(E,0,0,p_z)$ in $z$-Richtung, $E^2=p_z^2+m^2$.
$$\pslash = p^\mu \gamma_\mu = E\gamma_0 + p_z\gamma_3 = E\gamma^0-p_z\gamma^3=\begin{pmatrix}
\mathbf{1}E & -p_z\sigma^3 \\ p_z\sigma^3 & -\mathbf{1}E
\end{pmatrix}$$
Es gilt $[\pslash , S_{12}]=0$, d.h. $\pslash$ und $S_z$ haben simultane Eigenzustände. (allg. $\pslash$ und $\frac{\vf{p}\cdot\vf{S}}{|\vf{p}|}$ = Helizitätsoperator simultan Diagonalisierbar).\par 
EW-Gleichung lösen:
$$u(p,+\sfrac{1}{2})=N\cdot\begin{pmatrix}
E+m \\ 0 \\ p_z \\ 0
\end{pmatrix}$$
$$u(p,-\sfrac{1}{2})=N\cdot\begin{pmatrix}
0 \\E+m \\ 0 \\ -p_z 
\end{pmatrix}$$
mit $N=\frac{1}{\sqrt{E+m}}$.
$$v(p, +\sfrac{1}{2})=N\cdot\begin{pmatrix}
p_z \\ 0 \\ E+m \\ 0
\end{pmatrix}$$
$$v(p, -\sfrac{1}{2})=N\cdot\begin{pmatrix}
0 \\ -p_z \\ 0 \\ E+m
\end{pmatrix}$$
Spinoren für andere $\vf{p}$: $\vf{p}=R\vf{p}_z = e^{-\frac{i}{2}\omega^{\mu\nu}L_{\mu\nu}}\vf{p}_z$:
$$u(p,s)=e^{-\frac{i}{2}\omega^{\mu\nu}S_{\mu\nu}}u(p_z,s)$$
\paragraph{Negative Energien}
$$\psi (x)=u(p,s)=e^{-iEt+i\vf{p}\cdot\vf{x}}$$
$$\psi (x)=v(p,s)=e^{+iEt-i\vf{p}\cdot\vf{x}}$$
D.h. Energie $(-E)<0$ für $v$-Lösungen.

\subsection{Mehr zum Drehimpuls}

Man betrachte die Diracgleichung als quantenmechanische 1-Teilchen-Gleichung. (sinnvoll, solange Antiteilchen und QFT Effekte vernachlässigbar sind).\par 

Formulierung analog zur Schrödingergleichung im Ortsraum:
$$(i\dslash - m)\psi = 0$$
Multiplikation mit $\gamma^0$ von links und nach Zeitableitung umstellen:
$$i\partial_t\psi = (-i\gamma^0\gamma^i\partial_i+m\gamma^0)\psi =:\hat{H}^{(0)}_D\psi$$
\paragraph{Drehimpuls} aus Darstellung der Lorentztransformation.
$$\hat{J}_{ij}=i(x_i\partial_j-x_j\partial_i)+\hat{S}_{ij} = \hat{L}_{ij}+\hat{S}_{ij}$$
$$\hat{\vf{J}}=\hat{\vf{L}}+\hat{\vf{S}}$$
Es gilt $[\hat{H}^{(0)}_D,\hat{\vf{J}}]=0$, d.h. Gesamtdrehimpuls erhalten. $[\hat{H}^{(0)}_D,\hat{\vf{L}}]=\gamma^0\gamma_1\partial_y-\gamma^0\gamma_2\partial_x$.
\paragraph{Helizität}
$$\frac{\hat{\vf{S}}\cdot\hat{\vf{p}}}{|\hat{\vf{p}}|}$$
$$[\hat{H}^{(0)}_D, \hat{\vf{S}}\cdot\hat{\vf{p}}]=[\hat{H}^{(0)}_D,\frac{1}{2}\epsilon_{ijk}S_{ij}\hat{p}^k]=\sim\frac{1}{2}\epsilon_{ijk}\gamma^0\gamma_i\partial_j\partial_k=0$$
Es gibt simultane Eigenzustände zu Energie, Impuls, Helizität.

\paragraph{Interpretation der 4 Komponenten von $\psi$}\mbox{}\par 
Zu gegebenem Impuls $\vf{p}$: 4 linear unabhängige Lösungen:
\begin{itemize}
\item $E>0$, Helizität $\pm\frac{1}{2}$
\item $E<0$, Helizität $\pm\frac{1}{2}$
\end{itemize}

\subsection{Kopplung ans elektromagnetische Feld}

Freie Diracgleichung: $(i\gamma^\mu\partial_\mu -m)\psi = 0$\par 
Freie Klein-Gordon-Gleichung: $(-\partial_\mu\partial^\mu - m^2)\phi = 0$\par 
Relativistisches klassisches Teilchen: $L=\frac{1}{2}m\frac{\dif x^\mu}{\dif\tau}\frac{\dif x_\mu}{\dif\tau}$

Kopplung and e.m. Feld soll relativistisch invariant und eichinvariant sein. (Eichung $A^\mu (x)\mapsto A^\mu (x)+\partial^\mu\theta (x)$).\par 

Klassisches Teilchen:
$$L=\frac{1}{2}m\frac{\dif x^\mu}{\dif\tau}\frac{\dif x_\mu}{\dif\tau} - e\frac{\dif x_\mu}{\dif\tau}A^\mu (x)$$
(Einfachse denkbare relativistische WW, Wirkung ist eichinvariant, reproduziert Coulomb- und Lorentzkraft)\par 
Kanonisch konjugierter Impuls:
$$\mathcal{P}^\mu = \frac{\partial L}{\partial\frac{\dif x_\mu}{\dif\tau}}=m\frac{\dif x^\mu}{\dif\tau}-eA^\mu$$
$$\Rightarrow H=\frac{1}{2m}(\mathcal{P}^\mu + eA^\mu )^2$$
Rezept: minimale Kopplung $\mathcal{P}^\mu\rightarrow\mathcal{P}^\mu + eA^\mu$, Klein-Gordon-Gleichung:
$$\boxed{\left[\left( i\partial^\mu + eA^\mu\right)\left( i\partial_\mu +eA_\mu \right) -m^2\right]\phi = 0}$$
Dirac-Gleichung:
$$\boxed{\left(i\dslash + e\slashed{A}-m\right)\psi = 0}$$
Elektromagnetische Stromdichte:
$$j^\mu = e\overline{\psi}\gamma^\mu\psi$$
Eichinvarianz:
\begin{align*}
A^\mu (x) &\longrightarrow A^\mu (x) +\partial^\mu\theta (x) \\
\psi (x) &\longrightarrow e^{ie\theta (x)}\psi(x)
\end{align*}
Eichkovariante Ableitung:
$D^\mu\psi :=(\partial^\mu -ieA^\mu )\psi$. Damit gilt $D^\mu\psi \longrightarrow e^{ie\theta (x)} D^\mu\psi$


\subsection{Nichtrelativistischer Limes}

Nichtrelativistische Schrödingergleichung mit e.m. Feld:
$$(i\partial_t + e\Phi )\psi = \frac{(\hat{\vf{p}}+e\vf{A})^2}{2m}\psi$$
Klein-Gordon-Gleichung:
$$\left[\left( i\partial^\mu + eA^\mu\right)\left( i\partial_\mu +eA_\mu \right) -m^2\right]\phi = 0$$
$(A^\mu ) = (\Phi ,\vf{A})$, $(i\partial^j) = (-i\partial_j ) = (p^j)$.\par 
Ansatz: 
\begin{itemize}
\item $\phi$ ist Energie-EZ, $i\partial_t\phi = E\phi$
\item $E=m+\textit{klein}$, $E>0$
\item $e|A^\mu |\ll m$
\item $|\partial_tA^\mu |\ll |mA^\mu |$
\item $|p|\ll m$
\end{itemize}
Einsetzen in KG-Gl.:
$$\left[ (i\partial_t+e\Phi)(E+e\Phi )-(\hat{\vf{p}}+e\vf{A})^2-m^2\right]\phi = 0$$
Vernachlässigen von $\partial_t\Phi$:
$$\left[ (E+e\Phi )^2 - (\hat{\vf{p}}+e\vf{A})^2-m^2\right]\phi = 0$$
Mit $E+e\Phi = m + (E-m+e\Phi )$ mit Vernachlässigung des Quadrates der letzten Klammer:
$$\left[ 2m(E-m+e\Phi ) - (\hat{\vf{p}}+e\vf{A})^2 \right]\phi = 0$$
Daraus folgt direkt die nichtrelativistische Schrödingergleichung.

\paragraph{Diracgleichung mit e.m. Feld}
$$(i\slashed{D} - m)\psi = 0$$
Ansatz wie oben. Aufteilung des Diracspinors in zwei Paulispinoren:
$$\psi = \begin{pmatrix}
\psi_A \\ \psi_B
\end{pmatrix}$$
$$\begin{pmatrix}
iD_0 -m & iD_i\sigma^i \\
-iD_i\sigma^i & -iD_0 - m
\end{pmatrix}\begin{pmatrix}
\psi_A \\ \psi_B
\end{pmatrix}=0$$
Nach Ansatz: $iD_0\rightarrow E+e\Phi $, $iD_i\sigma^i = -\vec{\sigma}(\hat{\vf{p}}+e\vf{A})$.
\begin{align*}
(E-m+e\Phi )\psi_A - \vec{\sigma}(\hat{\vf{p}}+e\vf{A})\psi_B &= 0 \\
(-E-m-e\Phi )\psi_B + \vec{\sigma}(\hat{\vf{p}}+e\vf{A})\psi_A &= 0
\end{align*}
Eliminiere 
$$\psi_B = \frac{\vec{\sigma}(\hat{\vf{p}}+e\vf{A})}{E+m+e\Phi}\psi_A \cong \left(\frac{1}{2m}+\mathcal{O}(m^{-2})\right)\vec{\sigma}(\hat{\vf{p}}+e\vf{A})$$
$$\Rightarrow\quad (E-m+e\Phi )\psi_A = \frac{1}{2m}\left(\vec{\sigma}(\hat{\vf{p}}+e\vf{A})\right)\left(\vec{\sigma}(\hat{\vf{p}}+e\vf{A})\right)\psi_A$$
\paragraph{Vereinfachung der $\sigma$-Anteile}
$$(\vec{\sigma}\cdot\hat{\vf{O}})(\vec{\sigma}\cdot\hat{\vf{O}}) = \sigma^i\hat{O}^i\sigma^j\hat{O}^j=\sigma^i\sigma^j\hat{O}^i\hat{O}^j$$
$$=\left(\frac{1}{2}\left\{\sigma^i,\sigma^j\right\} + \frac{1}{2}\left[\sigma^i,\sigma^j\right]\right)\hat{O}^i\hat{O}^j = \left(\delta^{ij}+i\epsilon^{ijk}\sigma^k\right)\hat{O}^i\hat{O}^j$$
$$=\hat{\vf{O}}^2+i\epsilon^{ijk}\sigma^k\frac{1}{2}[\hat{O}^i,\hat{O}^j]$$
Hier: $\hat{\vf{O}}=(\hat{\vf{p}}+e\vf{A})$:
$$\cdots = (\hat{\vf{p}}+e\vf{A})^2 + i\epsilon^{ijk}\sigma^k(-i\partial_ieA^j)$$
$$=(\hat{\vf{p}}+e\vf{A})^2 + e\vf{B}\cdot\vec{\sigma}$$
$$\boxed{(E-m+e\Phi )\psi_A=\left[\frac{(\hat{\vf{p}}+e\vf{A})^2}{2m}+\frac{e}{2m}\vec{\sigma}\cdot\vf{B}\right]\psi_A}$$
Pauli-Gleichung enthält Term $\vf{S}\cdot\vf{B}$ ($\vf{S}=\vec{\sigma}/2$) mit Vorfaktor:
$$\boxed{g_s\frac{e}{2m}\vf{S}\cdot\vf{B}\qquad ,\qquad g_s=2}$$

\paragraph{Bedeutung des $g_s$-Terms} Allg. Hamiltonian für magnetischen Dipol $\vec{\mu}$ im $\vf{B}$-Feld:
$$H = -\vec{\mu}\cdot\vf{B}_{ext}$$
Vergleich mit Pauli-Gleichung liefert $\vec{\mu}_s = -g_s\frac{e}{2m}\vf{S}$ mit $g_s = 2$. Das ist ein intrinsisches magnetisches Dipolmoment, proportional zum Spin.\par 

Vergleich mit klassischer Elektrodynamik (rotierende Ladungsverteilung, Ladung $Q$, Masse $M$, Drehimpuls $\vf{L}$) liefert $\vec{\mu}=\frac{Q}{M}\vf{L}$ $\Rightarrow$ Klassisches Ergebnis entspricht $g=1$.

\paragraph{Interpretation des ersten Terms} (identisch in der nicht-relativistischen Schrödingergleichung)
$$\frac{(\hat{\vf{p}}+e\vf{A})^2}{2m}=\underbrace{\frac{\hat{\vf{p}}^2}{2m}}_{E_{kin}}+\underbrace{\frac{e}{2m}(\hat{\vf{p}}\vf{A}+\vf{A}\hat{\vf{p}})+\frac{e^2}{2m}\vf{A}^2}_{\text{e.m. WW}}$$
Bsp. homogenes $\vf{B}$-Feld: setze $\vf{A}(x)=-\frac{1}{2}(\vf{x}\times\vf{B})$, dann $\vf{B}=\nabla\times\vf{A}$.
$$\hat{\vf{p}}\vf{A}+\vf{A}\hat{\vf{p}}=\vf{B}\cdot\hat{\vf{L}}$$
$$\Rightarrow \text{Erster Term } = \frac{\hat{\vf{p}}^2}{2m} +\frac{e}{2m}\vf{B}\cdot\hat{\vf{L}}+\frac{e^2}{2m}\vf{A}^2$$

\subsection{Weitere Konsequenzen: Spin-Bahn-Kopplung}

Höhere Ordnungen im nicht-relativistischen Limes:
\begin{itemize}
\item Spin-Bahn-Kopplung $\sim\vf{L}\cdot\vf{S}$ (Feinstrukturaufspaltung)
\item Darwin-Term
\item Korrektur E-kin. 
\end{itemize}

Saubere Herleitung durch systematische Entwicklung in Potenzen von $m$. $\frac{1}{m}$ sei eine kleine Größe. \\ $\rightarrow$ Foldy-Wouthuysen-Transformation/-Bild.

$$(i\slashed{D}-m)\psi = 0$$
$$\Leftrightarrow i\partial_t\psi = (-e\Phi + m\gamma^0 - iD_i\gamma^0\gamma^i)\psi = H_D\psi$$

Idee: Unitäre Transformation / neues "Bild", Zerlegung in 2-Spinoren.
$$\psi = e^{-iS}\psi '=e^{-iS}\begin{pmatrix}
\psi_A' \\ \psi_B'
\end{pmatrix}$$
$S$ hermitesch, eventuell $t$-abhängig.\par 

Neuer Hamiltonian:
$$i\partial_t\psi' = i\partial_t (e^{iS}\psi ) = (i\partial_t e^{iS})\psi + e^{iS}i\partial_t\psi$$
$$=\left[ (i\partial_te^{iS})e^{-iS}+e^{iS}H_De^{-iS}\right]\psi '$$
$$H_D'=i(i\dot{S}+\frac{i^2}{2}[S,\dot{S}]+\frac{i^3}{6}[S,[S,\dot{S}]]+\ldots )+ H_D + i[S,H_D]+\frac{i^2}{2}[S,[S,H_D]]+\ldots$$

Idee 2: $H_D'$ soll blockdiagonal sein in 2-Spinoren (bis zu bestimmter Ordnung) $\rightarrow$ Gleichung für $\psi_A '$ reicht aus.\par 

Konkret: 
$$H_D=m\gamma^0 + (-e\Phi ) + \begin{pmatrix}
0 & (\vf{p}+e\vf{A})\cdot\vec{\sigma} \\
(\vf{p}+e\vf{A})\cdot\vec{\sigma} & 0
\end{pmatrix}= \underbrace{m\gamma^0}_{\mathcal{O}(m^1)} + \underbrace{\mathcal{E}}_{{\mathcal{O}(m^0)}} + \underbrace{\mathcal{O}}_{{\mathcal{O}(m^0)}}$$
Häufige Umformung: $\gamma^0 O =-O\gamma^0$ mit ungeradem Operator $O$.\par 

1. Schritt: arbeite bis $\mathcal{O}(m^0)$: Setze $S=\mathcal{O}(m^{-1})$
$$H_D'=H_D + i[S,H_D] + \mathcal{O}(m^{-1}) = m\gamma^0+\mathcal{E}+\mathcal{O}+i[S,m\gamma^0+\mathcal{E}+\mathcal{O}] + \mathcal{O}(m^{-1})$$
$$=m\gamma^0 +\mathcal{E}+\mathcal{O} + i[S,m\gamma^0]$$
Lösung: $S=-\frac{i}{2m}\gamma^0\mathcal{O}$\par 
Damit $H_D'$ komplett ausrechnen bis $\mathcal{O}(m^{-2})$:
$$H_D'=H_D+i[S,H_D]-\dot{S}+\frac{i^2}{2}[S,[S,H_D]]-\frac{i}{2}[S,\dot{S}] + \frac{i^3}{6}[S,[S,[S,H_D]]]+\mathcal{O}(m^{-3})$$

Für die einzelnen Terme finden Wirkung
\begin{align*}
    i [S, H_D] &= i \left[-\frac{i}{2m} \gamma^0 \mathcal{O}, m \gamma^0 + \mathcal{E} + \mathcal{O} \right] = -\mathcal{O} + \frac{1}{2m} \gamma^0 [\mathcal{O}, \mathcal{E}] + \frac{1}{m} \gamma^0 \mathcal{O}^2 \\
    - \dot{S} &= \frac{i}{2m} \gamma^0 \dot{\mathcal{O}} \\
    \frac{i}{2} [S, \dot{S}] &= -\frac{i}{8 m^2} [\mathcal{O}, \dot{\mathcal{O}}] \\
    \frac{i^2}{2} [S, [S, H_D]] &= -\frac{1}{2m} \gamma^0 \mathcal{O}^2 - \frac{1}{8 m^2} [\mathcal{O}, [\mathcal{O}, \mathcal{E}]] - \frac{1}{2m^2} \mathcal{O}^3 \\
    \frac{i^3}{3!} [S, [S, [S, H_D]]] &= \frac{1}{6m^2} \mathcal{O}^3
\end{align*}
Der neue Hamiltonian ist nun
\begin{align*}
    H_D' &=\underbrace{m \gamma^0 + \mathcal{E} + \frac{1}{2m} \gamma^0 \mathcal{O}^2 - \frac{1}{8 m^2} [\mathcal{O}, i \dot{\mathcal{O}} + [\mathcal{E}, \mathcal{O}]]}_{\text{gerade} \;=: H_{D,\text{even}}'} + \\
    &= \underbrace{\frac{1}{2m} \gamma^0 (i \dot{\mathcal{O}} + [\mathcal{O}, \mathcal{E}]) - \frac{1}{6m^2} \mathcal{O}^3}_{\text{ungerade} \; =: \mathcal{O}'} \\
    &=: H_{D,\text{even}}' + \mathcal{O}'
\end{align*}

2. Schritt: arbeite bis $\mathcal{O}(m^-1)$: 

In Analogie setzen wir $\psi' = e^{i S'} \psi''$ mit $S' = -\frac{i}{2m} \gamma^0 \mathcal{O}'$ und erhalten
$$H_D'' = H_{D,\text{even}}' + i [S', \mathcal{E}] - \dot{S}' + \mathcal{O}(m^{-3}) := D_{D,\text{even}} + \mathcal{O}''$$

3. Schritt: arbeite bis $\mathcal{O}(m^{-2})$:

Wir setzen wieder $\psi'' = e^{i-i S''} \psi'''$ mit $S'' = -\frac{i}{2m} \gamma^0 \mathcal{O}'' = \mathcal{O}(m^{-3})$.

HIER FEHLT NOCH DIE GLEICHUNG FÜR $H_D'''$

Vollständig ausgerechnet:
$$H_D''' = \underbrace{m\gamma^0 + \mathcal{E} + \frac{1}{2m} \gamma^0 \mathcal{O}^2}_{\mathcal{O}(m^{-1})} - \underbrace{\frac{1}{8m^2} [\mathcal{O}, i \dot{\mathcal{O}} + [\mathcal{O}, \mathcal{E}]]}_{\mathcal{O}(m^{-2})}$$

\begin{itemize}
    \item Terme bis $\mathcal{O}(m^{-1})$ liefern genau den Limes aus 1.4.4 inkl. des $g-2$-Terms
    \item Zusätzliche Terme der relativistischen Korrektur bis $\mathcal{O}(m^{-2})$
\end{itemize}

Wir diskutieren diese Terme anhand des Zentralpotentials mit $\vf{A} = 0$ und $\Psi(\vf{x}, t) = \Psi(r)$ mit $r = |\vf{x}|$.
Es ergeben sich die Terme
\begin{align*}
    \nabla \Psi (r) &= \frac{\vf{x}}{r} \frac{\mathrm{d} \Psi}{\mathrm{d} r} \\
    \vf{E} &= - \nabla \Psi \\
    \mathcal{E} &= e \Psi \\
    \mathcal{O} &= \begin{pmatrix}
        0 & \vec{\sigma} \cdot\vf{p} \\ \vec{\sigma}\cdot\vf{p} & 0 \\
    \end{pmatrix} = -i \begin{pmatrix}
        0 & \vec{\sigma}\cdot\nabla \\ \vec{\sigma}\cdot \nabla & 0 \\
    \end{pmatrix} \\
    [\mathcal{O}, \mathcal{E}] &= -i e \begin{pmatrix}
        0 & \vec{\sigma}\cdot \vf{E} \\ \vec{\sigma}\cdot \vf{E} & 0 \\
    \end{pmatrix} \\
    [\mathcal{O}, [\mathcal{O}, \mathcal{E}]] &= (-i) (-i e) \begin{pmatrix}
        [\vec{\sigma}\cdot \nabla, \vec{\sigma}\cdot \vf{E}] & 0 \\ 0 & [\vec{\sigma}\cdot \nabla, \vec{\sigma}\cdot \vf{E}] \\
    \end{pmatrix}  \\
    [\vec{\sigma}\cdot \nabla, \vec{\sigma}\cdot \vf{E}]  &= \sigma^i \sigma^j (\partial_i E^j + E^j \partial_i) - \sigma^j \sigma^i E^j \partial_i \\
    &= \nabla\cdot \vf{E} + \underbrace{i \vec{\sigma}\cdot (\nabla \times \vf{E})}_{= 0} 
    + \underbrace{i^2 \epsilon^{i j k} \sigma^k E^j \partial_i}_{=2 \vec{\sigma}\cdot (\vf{E} \times \vec{p})} \\
    &= \nabla\cdot\vf{E} - \frac{2}{r} \frac{\mathrm{d}\Psi}{\mathrm{d}r} \vec{\sigma}\cdot \vf{L}
\end{align*}
Wir finden den nun bis zum $\mathcal{O}(m^{-2})$ Term blockdiagonalen Hamiltonian
$$H_D''' = \frac{e}{8 m^2} \nabla\cdot\vf{E} - \frac{e}{2 m^2 r} \frac{\mathrm{d} \Psi}{\mathrm{d}r} \vf{S}\cdot \vf{L}$$
Der obere Block ist 
\begin{align*}
    H_{\text{eff}} &= m + H_{\mathcal{O}(m^{-1})} + H_{\mathcal{O}(m^{-2})} + \ldots \\
    H_{\mathcal{O}(m^{-1})} &= H_{\text{Pauli}} = -e \Psi + \frac{(\vf{p} + e \vf{A})^2}{2 m} + \frac{e}{2m}\vec{\sigma}\cdot \vf{B} \\
    H_{\mathcal{O}(m^{-2})} &= \underbrace{\frac{e}{8m^2}\nabla\cdot \vf{E}}_{\text{Darwin-Term}} - \underbrace{\frac{e}{2m^2 r} \frac{\mathrm{d}\Psi}{\mathrm{d}r} \vf{S}\cdot \vf{L}}_{\text{Spin-Bahn-Kopplung}}
\end{align*}
Diskussion:
\begin{itemize}
    \item Darwin-Term: beim Atom $\nabla\cdot\vf{E} = 4\pi \rho_{\text{Kern}} \propto \delta^{(3)}(\vf{x})$ ergibt sich eine Korrektur für die s-Orbitale, die am Kern eine endliche Aufenthaltswahrscheinlichkeit haben
    \item Spin-Bahn-Koppluns: Wegen dieses Terms $[H_{\text{eff}}, \vf{S}] \neq 0$ und $[H_{\text{eff}}, \vf{L}] \neq 0$, aber $[H_{\text{eff}}, \vf{J}] = 0$.
\end{itemize}

\chapter[Ununterscheidbare Teilchen - Bosonen und Fermionen]{Ununterscheidbare Teilchen\\ \Large{Bosonen und Fermionen}}

Klassisch: jedes Teilchen hat eine eindeutige Bahnkurve $\rightarrow$ prinzipiell daran erkennbar.\par 

QM: keine eindeutige Bahnkurve

\paragraph{Fragen}
\begin{itemize}
\item Existieren ``ununterscheidbare Teilchen''? $\rightarrow$ Ja! (experimenteller Beweis)
\item Wie beschreibt man das? $\rightarrow$ Mehrteilchensysteme, Zustände, Hilberträume/Operatoren
\item Nützlicher Formalismus? $\rightarrow$ Erzeuger/Vernichter, Zweite Quantisierung, Quantenfeldtheorie
\end{itemize}

\section{Unterscheidbare Teilchen}

\subsection{Zustände}

\newcommand{\bra}[1]{\langle #1 |}
\newcommand{\sprod}[2]{\langle #1 | #2 \rangle}

Basiszustände für zwei Teilchen ohne Wechselwirkung:\par 
Basis für Teilchen 1: $\ket{n^{(1)}}$, $n=1,2,\ldots$\\
Basis für Teilchen 2: $\ket{m^{(2)}}$, $m=1,2,\ldots$\par 
$\Rightarrow$ vernünftige Annahme: Basiszustände für Teilchen 1+2:\\ $\ket{n^{(1)}}\ket{m^{(2)}}$, $n,m=1,2,\ldots$ ``Produktzustände''\par 

Hilbertraum: Teilchen 1 $\mathcal{H}_1^{(1)}$, Teilchen 2 $\mathcal{H}_1^{(2)}$. (Oberer Index Teilchenindex, Unterer Index Teilchenzahl)\\
Teilchen 1+2: $\mathcal{H}_2=\mathcal{H}_1^{(1)}\otimes\mathcal{H}_1^{(2)}$ (\textit{Produktraum})
\begin{itemize}
\item $\mathcal{H}_2$ enthält sowohl Produktzustände (\textit{separabel}), z.B. $$\ket{1^{(1)}}\ket{2^{(2)}}$$ oder $$\left(\ket{1^{(1)}}+\ket{{3}^{(1)}}\right)\left(\ket{5^{(2)}}+\ket{7^{(2)}}\right)$$\\
aber auch \textit{verschränkte Zustände} (``entangled''), z.B. $$\frac{\ket{1^{(1)}}\ket{1^{(2)}}-\ket{2^{(1)}}\ket{2^{(2)}}}{\sqrt{2}}$$
\end{itemize}

\paragraph{Skalarprodukte} ``offensichtlich'' übertragen
$$\left(\bra{\psi^{(1)}}\bra{\phi^{(2)}}\right)\left(\ket{\psi'^{(1)}}\ket{\phi'^{(2)}}\right):=\left(\sprod{\psi^{(1)}}{\psi'^{(1)}}\right)\cdot \left(\sprod{\phi^{(2)}}{\phi'^{(2)}}\right)$$
Schreibweise: $\ket{\psi^{(1)}}\ket{\phi^{(2)}}=\ket{\psi ,\phi}$, 
Ortsraum-Wellenfunktion: $\ket{x_1^{(1)}}\ket{x_2^{(2)}}=\ket{x_1,x_2}$\par 
$$\sprod{x_1,x_2}{\psi}=:\psi (x_1,x_2)$$

\subsection{Observablen/Operatoren}

Observable: $A_2$: hermitesche Operatoren auf $\mathcal{H}_2$
\begin{itemize}
\item Observablen, die nur ein Teilchen betreffen: entsprechen $A_1^{(1)}$:
$$\bra{\psi^{(1)}}\bra{\phi^{(2)}} A_2^{(1)} \ket{\psi'^{(1)}}\ket{\phi'^{(2)}} = \erwop{\psi^{(1)}}{A_1^{(1)}}{\psi'^{(1)}}\cdot\sprod{\phi^{(2)}}{\phi'^{(2)}}$$
$$A_2^{(1)}=A_1^{(1)}\otimes\mathbf{1}$$
\item Analog: Observable betrifft nur Teilchen 2:
$$B_2^{(2)}=\mathbf{1}\otimes B_1^{(2)}$$
\end{itemize}
Allgemeine Observable: keine Produktstruktur nötig! $\rightarrow$ WW zwischen Teilchen!\par 
Bsp. Coulomb-Potenzial zwischen Teilchen 1 und 2:
$$\erwop{\psi^{(1)},\phi^{(2)}}{V_2}{\psi^{(1)},\phi^{(2)}}=\int\dif^3x_1\,\dif^3x_2\frac{-\alpha}{|\vf{x}_1-\vf{x}_2|}|\psi (\vf{x}_1)|^2 |\phi (\vf{x}_2)|^2$$
$$\Longrightarrow V_2 = \int\dif^3x_1\,\dif^3x_2 (\ket{\vf{x}_1^{(1)}}\bra{\vf{x}_1^{(1)}}\otimes \ket{\vf{x}_2^{(2)}}\bra{\vf{x}_2^{(2)}})\frac{-\alpha}{|\vf{x}_1-\vf{x}_2|}$$

Hamiltonian:
$$H_2=H_1^{(1)}\otimes\mathbf{1} + \mathbf{1}\otimes H_1^{(2)}+H_{WW}^{(1,2)}$$

\section{Identische/Ununterscheidbare Teilchen}

\subsection{Prinzipien}

Exp: Pauliprinzip, Fermigas, Gibbs Paradoxon (keine Mischungsentropie wenn gleichatomige Gase gemischt werden)\par 

Bisheriger Formalismus reicht nicht aus, da die bisherigen Zustände zu detailliert sind (Zuordnung des Teilchenindexes ist überflüssig)\par 

\paragraph{Fundamentale Beobachtungstatsache / Postulat} Zustände eines Systems ununterscheidbarer Teilchen sind gegenüber Vertauschung der Teilchenindizes generell symmetrisch oder generell antisymmetrisch.

\paragraph{Bosonen} (Spin ganzzahlig) $\ket{...\psi, \phi ...} = +\ket{...\phi, \psi ...}$
\paragraph{Fermionen} (Spin halbzahlig) $\ket{...\psi, \phi ...} = -\ket{...\phi, \psi ...}$

\subsection{Zustände}

\newcommand{\hil}{\mathcal{H}}

$N$-Teilchen Hilbertraum $\hil_N=\hil_1\otimes\ldots\otimes\hil_N$\par 

Permutationsoperator $P_{ij}$:
$$P_{ij}\ket{...\psi^{(i)}...\phi^{(j)}...}=\ket{...\phi^{(i)}...\psi^{(j)}...}$$
$(P_{ij})^2=\mathbf{1}$, $(P_{ij})^\dagger = P_{ij}$\par 
(Anti-)symmetrischer Hilbertraum:
\begin{itemize}
\item $\hil_N^{(+)}$ Teilchenraum mit $P_{ij}\ket{\phi^{(+)}}=\ket{\phi^{(+)}}$
\item $\hil_N^{(-)}$ Teilchenraum mit $P_{ij}\ket{\phi^{(-)}}=-\ket{\phi^{(-)}}$
\end{itemize}

\paragraph{Bsp. 2 Bosonen} 
\begin{itemize}
\item Basis $\hil_1$: $\ket{n}$
\item Basis $\hil_2$: $\ket{n^{(1)},m^{(2)}}$
\item Basis $$\hil_2^{(+)}: \frac{\ket{n^{(1)}m^{(2)}}+\ket{m^{(1)}n^{(2)}}}{\sqrt{2}}=:\ket{n,m}^{(+)}$$
\end{itemize}
\paragraph{Bsp. 2 Fermionen} (Vernachlässige Spin)
\begin{itemize}
\item Basis $$\hil_2^{(-)}: \frac{\ket{n^{(1)}m^{(2)}}-\ket{m^{(1)}n^{(2)}}}{\sqrt{2}}=:\ket{n,m}^{(-)}$$
\end{itemize}
\paragraph{Bsp. 2 Fermionen} (Mit Spin)
\begin{itemize}
\item Basis $\hil_1$: $\ket{n^\uparrow}$, $\ket{n^\downarrow}$
\item Basis $\hil_2$: Vier Kombinationen von $n$ und $m$ für verschiedene Spineinstellungen oder äquivalent:
$$\ket{n^{(1)}m^{(2)}}\otimes\ket{\uparrow\uparrow}, \ket{n^{(1)}m^{(2)}}\otimes\left(\frac{\ket{\uparrow\downarrow}+\ket{\downarrow\uparrow}}{\sqrt{2}}\right), \ket{n^{(1)}m^{(2)}}\otimes\ket{\downarrow\downarrow}, \ket{n^{(1)}m^{(2)}}\otimes\left(\frac{\ket{\uparrow\downarrow}-\ket{\downarrow\uparrow}}{\sqrt{2}}\right)$$
\item $\hil_2^{(1)}$:
$$\frac{\ket{n^{(1)}m^{(2)}}-\ket{m^{(1)}n^{(2)}}}{\sqrt{2}} \otimes\left\{\begin{matrix}
\ket{\uparrow\uparrow} \\ \frac{\ket{\uparrow\downarrow}+\ket{\downarrow\uparrow}}{\sqrt{2}} \\ \ket{\downarrow\downarrow}
\end{matrix}\right.$$
$$ \frac{\ket{n^{(1)}m^{(2)}}+\ket{m^{(1)}n^{(2)}}}{\sqrt{2}}\otimes \frac{\ket{\uparrow\downarrow}-\ket{\downarrow\uparrow}}{\sqrt{2}}$$
Folgerung: Selber Ort unmöglich, wenn Spins gleich.
\end{itemize}

\paragraph{Frage:} Sind obige Zustände eine Basis? Wie konstruiert man allgemein eine Basis von $\hil_N^{(\pm )}$?

\paragraph{Antwort:} Nimm Basis aus Produktzuständen von $\hil_N$, symmetrisiere/antisymmetrisiere jedes Basiselement (wie für $N=2$ genutzt).\par 

Def. \textit{Symmetrisierungsoperator}
$$S_N^{(\pm )} := \frac{1}{N!}\sum_{\mathcal{P}} (\pm 1)^\mathcal{P}\mathcal{P}$$
mit Permutationsoperator $\mathcal{P}$ (beliebiges Produkt von $P_{ij}$-Operatoren).\par 
Es gilt: 
\begin{itemize}
\item[(a)] $$P_{ij}S_N^{(\pm )} = \frac{1}{N!}\sum_\mathcal{P}(\pm )^\mathcal{P} P_{ij}\mathcal{P} = \pm S_N^{(\pm )}=S_N^{(\pm )}P_{ij}$$
\item[(b)] $$\mathcal{P} S_N^{(\pm )} = (\pm 1)^\mathcal{P} S_N^{ (\pm )}$$
\item[(c)] $S_N^{(\pm )}$ ist hermitesch.
\item[(d)] $$S_N^{(\pm )} S_N^{(\pm )} = S_N^{(\pm )}$$
\end{itemize}
$S_N^{(\pm )}$ sind hermitesche Projektionsoperatoren auf $\hil_N^{(\pm )}$.

\paragraph{Konstruktion einer Basis}
\begin{itemize}
\item Nimm Basis von $\hil_N$ aus Produktzuständen: $\ket{n_1^{(1)}n_2^{(2)}\cdots n_N^{(N)}}$
\item Def. $S_N^{(\pm )}\ket{n_1^{(1)}n_2^{(2)}\cdots n_N^{(N)}} =: \ket{n_1\cdots n_N}^{(\pm )}$
\item Nimm beliebigen Zustand $\ket{\psi_N^{\pm}} \in \mathcal{H}_N^{\pm}$
\begin{align*}
    \implies \ket{\psi_N^{\pm}} &\in \mathcal{H}_N, \\
    P_{i j} \ket{\psi_N^{\pm}} &= \pm \ket{\psi_N^{\pm}} \implies S_N^{\pm} \ket{\psi_N^{\pm}} = + \ket{\psi_N^{\pm}} \\
    \implies \ket{\psi_N^{\pm}} &=  S_N^{\pm} \left(\int \ket{n_1^1 \ldots n_N^N}\bra{n_1^1 \ldots n_N^N}\right) \left(S_N^{\pm}\right)^\dagger \ket{\psi_N^{\pm}}\\
    &= \sum\int \underbrace{\ket{n_1 \ldots n_N}}_{\text{Basiszustände}} \underbrace{\sprod{n_1 \ldots n_N}{\psi_N^{\pm}}}_{\text{Koeffizienten}}
\end{align*}
In der Tat stimmt die obige Antwort und die Basis ist durch die obige Gleichung gegeben.
\item Normierung: per Konstruktion gilt die Vollständigkeitsrelation
$$\mathbf{1}_{\mathcal{H}^{\pm}_N} = \int \ket{n_1 \ldots n_N}^{\pm}\bra{n_1\ldots n_N}^{\pm}$$
wegen $S_N^{\pm} S_N^{\pm} = S_N^{\pm}$ aber anders normiert als im 2-Teilchen-Beispiel.
\end{itemize}

\subsubsection{Observablen, weitere Motivation für Symmetrisierungspostulate}

System aus $N$ identischen Teilchen, $A_N$ sei sinnvolle Observable, $\ket{\psi_N}$ und $\ket{\phi_N}$ seien sinnvolle Zustände.
\begin{itemize}
    \item $\ket{\psi_N}$ und $P_{i j} \ket{\psi_N}$ ``bedeuten das selbe''
    \item Sinnvolle Annahme für die Observablen
    \begin{align*}
        \bra{\psi_N} A_N \ket{\psi_N} &= \bra{\psi_N} P_{ i j} A_N P_{i j} \ket{\psi_N} \\
        \implies A_N &= P_{i j} A_N P_{i j} \implies [A_N, P_{i j}] = 0
    \end{align*}
    für jede sinnvolle Observable auf dem Raum der sinnvollen Zustände.
    \item Spezielle Observable $A_N := \ket{\psi_N} \bra{\psi_N}$ ergibt $$P_{i j} A_N \ket{\psi_N} = A_N P_{i j} \ket{\psi_N} \iff  (P_{i j} \ket{\phi_N}) \sprod{\phi_N}{\psi_N} = \ket{\phi_N} \bra{\phi_N} P_{i j} \ket{\psi_N}$$
    Woraus schließlich folgt dass $$\iff P_{i j} \ket{\phi_N} = \lambda \ket{\phi_N} \implies \lambda=\pm 1$$.
    \item Das Symmetrisierungspostulat wird hierdurch suggestiert.  Das Postulat selbst ist noch etwas stärker, denn es besagt, dass für jede Teilchensorte genau nur ein Vorzeichen erlaubt ist.
    \item Beispiele für Observablen
    \begin{table}[hb!]
        \centering
        \begin{tabular}{|l|l|}
            \hline
            2 Teilchen unterscheidbar & $\vec{x}_1, \vec{x}_2$; $\vec{p}_1, \vec{p}_2$; $H = \frac{\vec{p}_1^2}{2m} + \frac{\vec{p}_2^2}{2m}$; $\vec{L}_1, \vec{L}_2, \vec{L}_{\text{ges}}$ \\
            & $H$ sinnvoll, $\vec{x}_1, \vec{x}_2$ nicht sinnvoll \\\hline
            2 Teilchen ununterscheidbar & $\vec{x}_1 - \vec{x}_2$ nicht sinnvoll, \\
            & aber $\vec{x}_1 + \vec{x}_2$, $(\vec{x}_1 - \vec{x}_2)^2$, $|\vec{x}_1 - \vec{x}_2|^2$, $\vec{x}_1 \vec{x}_2$ sinnvoll \\
            \hline
        \end{tabular}
    \end{table}
    \item Vollständiges System kommutierender Observablen ist kompliziert.
    \item Oft möglich: Rechnen nicht direkt mit $\mathcal{H}_N^{\pm}$ sondern in $\mathcal{H}$ und mit einzelnen Observablen und am Ende: Spezialisieren/Einschränken auf symmetrische bzw. antisymmetrische Zustände.
\end{itemize}

\section{Einfache Anwendungen}

\subsection{Grund- und angeregte Zustände}

$N$ Teilchen ohne Wechselwirkung;
\begin{enumerate}
    \item  Unterscheidbar: z.B. die Elektronen im He-Atom
    \item Fermionen: z.B. Elektronen im Metall
    \item Bosonen: mehrere $H$-Atome
\end{enumerate}

Beispiel: Alle Teilchen im Potential mit möglichen Energien $e_1$, $e_2$, $e_3, \ldots$ und Eigenzuständen $\ket{1}$, $\ket{2}$, $\ket{3}, \ldots$.
\begin{enumerate}
    \item Grundzustand $\ket{1^1, 1^2}$, $E = 2 e_1$

    1. Angeregter Zustand $\ket{1^1 2^2}$ oder $\ket{2^1 1^2}$, $E = e_1 + e_2$ 2-fach entartet.

    \item Grundzustand N: $\ket{1, 2, \ldots, N}^{-}$, $E = e_1 + e_2 + \ldots + e_N$ nicht entartet.
    $e_N$ ist die maximale besetzte Energie im Grundzustand, genannt Fermienergie
    
    1. Angeregter Zustand: $\ket{1,2, \ldots, N-1, N+1}^{-}$ nicht entartet! $\Delta E = e_{N+1} - e_N$
    \item Grundzustand N: $\ket{1, 1, \ldots, 1}^+$, $E = N e_1$
    
    1. Angeregter Zustand: $\ket{2, 1, \ldots, 1}$ nicht entartet! $\Delta E = e_2 - e_1$
\end{enumerate}

\subsection{Direkter Prozess vs. Austauschterm}

Zwei Teilchen: $\ket{\psi}$, $\ket{\phi}$ $\longrightarrow$ Prozess $\longrightarrow$ $\ket{n}$, $\ket{m}$\par 

Anfangszustand $\ket{i}$ $\longrightarrow$ Endzustand $\ket{f}$. Frage: Was ist die Wahrscheinlichkeit?
$$P_{i\rightarrow f} = |A_{i\rightarrow f}|^2$$

Unterscheidbar: (entweder nur links oder nur rechts):
\begin{itemize}
\item ``direkt'': $$A^d_{i\rightarrow f} = \sprod{n^{(1)}m^{(2)}}{\psi^{(i)}\phi^{(2)}}=\sprod{n}{\psi}\sprod{m}{\phi}$$
\item ``Austauschterm'': $$A^a_{i\rightarrow f}=\sprod{m^{(1)}n^{(2)}}{\psi^{(1)}\phi^{(2)}}=\sprod{m}{\psi}\sprod{n}{\phi}$$
\item Gesamtwahrscheinlichkeit: ``entweder $\bra{nm}$ oder $\bra{mn}$''
$$P_{i\rightarrow f}=|A^d_{i\rightarrow f}|^2+|A^a_{i\rightarrow f}|^2$$
\end{itemize}

Bosonen:
$$\ket{i}=\frac{\ket{\psi\phi}+\ket{\phi\psi}}{\sqrt{2}}\qquad\qquad\ket{f}=\frac{\ket{nm}+\ket{mn}}{\sqrt{2}}$$
$$A_{i\rightarrow f} = \sprod{f}{i} = \frac{1}{2}\left(\sprod{\psi}{n}\sprod{\phi}{m}+\sprod{\phi}{n}\sprod{\psi}{m}\right)\cdot 2 = A^d_{i\rightarrow f}+A^a_{i\rightarrow f}$$
$$P_{i\rightarrow f} = \left|A^d_{i\rightarrow f}+A^a_{i\rightarrow f}\right|^2$$

Fermionen (Analog):
$$P_{i\rightarrow f} = \left|A^d_{i\rightarrow f}-A^a_{i\rightarrow f}\right|^2$$

Spezialfall $n=m$: (Beide Teilchen gehen in den selben Zustand über)
\begin{itemize}
\item Fermionen: $P_{i\rightarrow f}=0$
\item Bosonen: $P_{i\rightarrow f}=2|A^d_{i\rightarrow f}|^2$ (Doppelt so groß wie bei unterscheidbaren Teilchen)
\end{itemize}

\subsection{Wasserstoffmolekül $\mathrm{H}_2$}

Chemische Bindung, gewisser Atomabstand $R$ minimiert die Energie. Austauschwechselwirkung sehr wichtig $\rightarrow$ Orts-Wellenfunktion.\par 

Im Grundzustand: Orts-Wellenfunktion symmetrisch, Spin antisymmetrisch.\par 

Annahme/Näherung: Kerne fixiert im Abstand $R$, Positionen der Kerne $a$ und $b$, Elektronen $1$ und $2$

$$H=\frac{\vec{p}_1^2}{2m}+\frac{\vec{p}_2^2}{2m}-\alpha\left(\frac{1}{r_{1a}}+\frac{1}{r_{2a}}+\frac{1}{r_{1b}}+\frac{1}{r_{2b}}-\frac{1}{r_{12}}-\frac{1}{R}\right)$$

$$H=H_{1,a}+H_{2,b}-\alpha\left(\frac{1}{r_{1b}}+\frac{1}{r_{2a}}-\frac{1}{r_{12}}-\frac{1}{R}\right)$$

H-Atom-Zustände, Struktur der 2-Elektron-Zustände.\par 

\paragraph{Erinnerung H-Atom:}\mbox{}\par 
Quantenzahlen $n, l, m$: $\psi_{nlm}\sim R_{nl}(r)Y_{lm}(\theta ,\varphi )$\par 
Grundzustand:
$$\psi_{100}=\frac{2}{\sqrt{4\pi}}a_B^{-\frac{3}{2}}e^{-\frac{r}{a_B}}$$
(Bohrscher Radius $a_B=\frac{1}{\alpha m}$).\par 
Energien: $E_1 = -\frac{\alpha^2 m}{2}$, $E_n = \frac{E_1}{n^2}$ (Zusätzlich ungebundene Zustände mit $E>0$)\par 

\paragraph{H-Atom mit Proton im Punkt $\vec{R}_a$}\mbox{}\par 

Selbe Energie-EW, Eigenzustände: $\psi_a(\vec{x})=\psi_{\mathrm{Ursprung}}(\vec{x}-\vec{R}_a)$

\paragraph{2-Elektron-Zustände} 2 Basen von 1-T.-Zuständen um Proton $a$ $\ket{\psi_a,nlm}$ und um Proton $b$ $\ket{\psi_b,nlm}$\par 

Basis von 2-Teilchen-Zuständen (antisymmetrisch): $\hil_2^{(-)}$:
$$\ket{\psi_{a,nlm},\psi_{b,n'l'm'}}^{(-)}\otimes\left\{\begin{matrix}
\ket{\uparrow\uparrow} \\ \frac{\ket{\uparrow\downarrow}+\ket{\downarrow\uparrow}}{\sqrt{2}} \\ \ket{\downarrow\downarrow}
\end{matrix}\right.$$
$$\ket{\psi_{a,nlm},\psi_{b,n'l'm'}}^{(+)}\otimes \left(\frac{\ket{\uparrow\downarrow}-\ket{\downarrow\uparrow}}{\sqrt{2}}\right)$$

Spinoperator kommutiert mit Hamiltonian, des Weiteren: $[\vec{S}^2,S_z]=0$. Simultane Eigenzustände:
$$\ket{SM}\qquad \qquad \vec{S}^2\ket{SM}=S(S+1)\ket{SM}\qquad S_z\ket{SM}=M\ket{SM}$$
Spin-Notation:
\begin{align*}
\ket{1,1}&:=\ket{\uparrow\uparrow}\\
\ket{1,0}&:=\frac{\ket{\uparrow\downarrow}+\ket{\downarrow\uparrow}}{\sqrt{2}}\\
\ket{1, -1}&:=\ket{\downarrow\downarrow} \\
\ket{0, 0}&:= \frac{\ket{\uparrow\downarrow}-\ket{\downarrow\uparrow}}{\sqrt{2}}
\end{align*}

Damit Basis:\par 
Ort antisymmetrisch, Spin $S=1$: $\ket{\psi_{a,nlm},\psi_{b,n'l'm'}}^{(-)}\otimes\ket{1,M}$\\
Ort symmetrisch, Spin $S=0$: $\ket{\psi_{a,nlm},\psi_{b,n'l'm'}}^{(+)}\otimes\ket{0,0}$\par 

Idee zur Lösung des H$_2$-Moleküls:
\begin{itemize}
\item Ziel: Grundzustandsenergie? Optimaler Abstand R?
\item Annahme/Näherung: obigen Basiszustände sind Eigenzustände des vollen Moleküls (d.h. WW klein, H-Atome nur wenig beeinflusst)
\item Variationsprinzip: Ansatz sinnvoller Zustände $\ket{\psi_{\text{sinnvoll}}}$
$$E_{var}=\frac{\erwop{\psi_{\text{sinnvoll}}}{H}{\psi_{\text{sinnvoll}}}}{\sprod{\psi_{\text{sinnvoll}}}{\psi_{\text{sinnvoll}}}}$$
Auf jeden Fall: $E_{var}\geq E_{\text{Grundzustand}}$ (Gleichheit bei guter Wahl)
\end{itemize}
\paragraph{Heitler-London-Näherung}\mbox{}\par 

Wähle $\ket{\psi_{\text{sinnvoll}}}:=\ket{\psi}^{(\pm )}=\ket{\phi_a,\phi_b}^{(\pm )}\otimes\ket{SM}$, wobei $\phi_a$ und $\phi_b$ die Grundzustände bezüglich der einzelnen H-Atome sind. Bei folgenden Matrixelementen: $\sprod{SM}{SM}=1$ trägt nicht weiter bei $\rightarrow$ ab jetzt nur noch Ortsraum betrachten.

$$\sprod{\vec{x}}{\phi_{a,b}}=\frac{2}{\sqrt{4\pi}}a_B^{-\frac{3}{2}}e^{-\frac{|\vec{x}-\vec{R}_{a,b}|}{a_B}}$$

Längere Rechnung ($\psi^{(\pm )}$ einsetzen und bekannte Skalarproduktrelationen, Normierung ausnutzen und beim Matrixelement auf Eigenzustände von Teilen des Hamiltonians achten):
\begin{itemize}
\item[(a)] $$\sprod{\psi^{(\pm )}}{\psi^{(\pm )}} = 1\pm |L_{ab}|^2$$
mit $L_{ab}=\sprod{\phi_a}{\phi_b}=\int\dif^3x\:\phi_a(\vec{x})\phi_b(\vec{x})$ (Überlapp).
\item[(b)] $$\erwop{\psi^{(\pm )}}{H}{\psi^{(\pm )}} = \erwop{\phi_a^{(1)}\phi_b^{(2)}}{H}{\phi_a^{(1)}\phi_b^{(2)}} \pm \erwop{\phi_a^{(1)}\phi_b^{(2)}}{H}{\phi_b^{(1)}\phi_a^{(2)}}$$
Diagonalterm:
$$\erwop{\phi_a^{(1)}\phi_b^{(2)}}{H}{\phi_a^{(1)}\phi_b^{(2)}}=2E_1+C_{ab}$$
mit Coulomb-Zusatzenergie 
\begin{align*}
C_{ab}=\frac{\alpha}{R}&-\alpha\int\dif^3x\: |\phi_a(\vec{x})|^2\frac{1}{|\vec{x}-\vec{R}_b|} \\
&- \alpha\int\dif^3x\: |\phi_b(\vec{x})|^2\frac{1}{|\vec{x}-\vec{R}_a|} \\
&+\alpha\int\dif^3x_1\:\dif^3x_2\frac{|\phi_a(\vec{x}_1)|^2|\phi_b(\vec{x}_2)|^2}{|\vec{x}_1-\vec{x}_2|}
\end{align*}
Off-Diagonalterm:
$$\erwop{\phi_a^{(1)}\phi_b^{(2)}}{H}{\phi_b^{(1)}\phi_a^{(2)}}= 2E_1|L_{ab}|^2+A_{ab}$$
mit Austauschterm
\begin{align*}
A_{ab}=\frac{\alpha}{R}|L_{ab}|^2 &-\alpha L_{ab}^*\int\dif^3x\: \frac{\phi^*_a(\vec{x})\phi_b(\vec{x})}{|\vec{x}-\vec{R}_a|} \\
&-\alpha L_{ab}\int\dif^3x\: \frac{\phi_a(\vec{x})\phi_b^*(\vec{x})}{|\vec{x}-\vec{R}_b|} \\
&+\alpha \int\dif^3x_1\:\dif^3x_2\frac{\phi_a^*(\vec{x}_1)\phi_b(\vec{x}_1)\phi_b^*(\vec{x}_2)\phi_a(\vec{x}_2)}{|\vec{x}_1-\vec{x}_2|}
\end{align*}
Damit
$$\boxed{E_{var}^{(\pm )}=2E_1 + \frac{C_{ab}\pm A_{ab}}{1\pm |L_{ab}|^2}}$$
numerisch ausrechnen!\par 
Stabile Bindung? Für welches R?


\end{itemize}

\section{Erzeugungs- und Vernichtungsoperatoren}

\subsection{Fock-Raum}

Der \textit{Fock-Raum} ist ein Zustandsraum, der sowohl 1-Teilchen-, also auch Mehr-Teilchen-Zustände enthält.\par 

Start wie bisher: 1-Teilchenraum $\hil_1$, Basis $\ket{n}$ sei gegeben.\par 

\newcommand{\fock}{\mathcal{F}}

Nun füge hinzu:
\begin{itemize}
\item ``Vakuumzustand'' $\ket{0}$ (Nicht Nullvektor!), $\sprod{0}{0} = 1$\\
$\rightarrow$ Vakuum-Hilbertraum $\hil_0 = \{ c\ket{0} , c\in\mathbb{C}\}$
\item ``Fockraum''\\
Bosonen: $\fock:=\hil_0 \oplus \hil_1 \oplus \hil_2^{(+)} \oplus \hil_3^{(+)} \oplus ... $\\
Fermionen: $\fock:=\hil_0 \oplus \hil_1 \oplus \hil_2^{(-)} \oplus \hil_3^{(-)} \oplus ... $\par 

Basis von $\fock$:
\begin{itemize}
\item Vakuum: $\ket{0}$
\item 1-Teilchen: $\ket{n}$
\item 2-Teilchen: $\ket{n_1n_2}^{(\pm )}$
\item 3-Teilchen: $\ket{n_1n_2n_3}^{(\pm )}$
\end{itemize}

Skalarprodukte:
$$\sprod{N\text{-Teilchen-Zustand}}{M\text{-Teilchen-Zustand}} = \left\{ \begin{matrix}
0 & N\neq M \\
\text{wie gehabt} & N=M
\end{matrix}\right. $$

\end{itemize}

\subsection{Erzeuger/Vernichter für Bosonen}

Erzeugungsoperator $a_n^\dagger$ ``erzeugt ein zusätzliches Teilchen im Basiszustand $\ket{n}$''

$$a_n^\dagger : \hil_N^{(+)}\rightarrow \hil_{N+1}^{(+)} \qquad a_n^\dagger\ket{0} = \ket{n}\qquad a_n^\dagger\ket{m} = \sqrt{2}\ket{nm}^{(+)} \qquad a_n^\dagger\ket{mk}^{(+)}=\sqrt{3}\ket{nmk}^{(+)}, \qquad ...$$

Jeder Basiszustand des Fockraums lässt sich durch mehrfache Anwendung des Erzeugers auf das Vakuum gewinnen.
$$\boxed{\ket{n_1,...,n_N}^{(+)} = \frac{1}{\sqrt{N!}}a_{n_1}^\dagger\cdots a_{n_N}^\dagger \ket{0}}$$

\paragraph{Vertauschungsrelationen} Nimm einen beliebigen Basiszustand aus $\fock$
$$a_{n_1}^\dagger a_{n_2}^\dagger \ket{m_1,...,m_N}^{(+)} = \sqrt{(N+1)(N+2)}\ket{n_1n_2m_1,...,m_N}^{(+)}$$
$$a_{n_2}^\dagger a_{n_1}^\dagger \ket{m_1,...,m_N}^{(+)} = \sqrt{(N+1)(N+2)}\ket{n_2n_1m_1,...,m_N}^{(+)}$$

\newcommand{\create}[1]{a_{#1}^\dagger}
\newcommand{\destroy}[1]{a_{#1}}

$$\Rightarrow [\create{n_1},\create{n_2}] = 0\qquad \text{(Für Bosonen)}$$

\paragraph{Vernichtungsoperator} $a_n := (\create{n})^\dagger$

$$\destroy{n}\ket{0} = 0$$
$$\destroy{n}\ket{m} = \left\{\begin{matrix}
0 & n\neq m \\
\ket{0} &n=m \text{ (und Zustände normiert)}
\end{matrix}\right.$$
$$\destroy{n}\ket{m_1,...,m_N}^{(+)} = \frac{1}{\sqrt{N}}\sum_{i=1}^N \delta_{nm_i}\ket{m_1...m_{i-1}m_{i+1}...m_N}^{(+)}$$

\paragraph{Vertauschungsrelation}
$$\destroy{n}\create{m}\ket{m_1...m_N}^{(+)} = \destroy{n}\sqrt{N+1}\ket{mm_1...m_N}^{(+)} =  \delta_{nm}\ket{m_1...m_N} + \sum \delta_{nm_i}\ket{mm_1...m_{i-1}m_{i+1}...m_N}^{(+)}$$
$$\create{m}\destroy{n}\ket{m_1...m_N}^{(+)}= \sum \delta_{nm_i}\ket{mm_1...m_{i-1}m_{i+1}...m_N}^{(+)}$$
Differenz enthält nur $\delta_{nm}$-Term.
\begin{align*}
[\destroy{n},\create{m}]&=\delta_{nm} :=\sprod{n}{m} \\
[\create{n},\create{m}] &= 0 \\
[\destroy{n},\destroy{m}] &= 0
\end{align*}
Diese Vertauschungsrelationen beschreiben die Bose-Natur der Teilchen. 

\subsection{Erzeuger/Vernichter für Fermionen}

\newcommand{\createf}[1]{c^\dagger_{#1}}
\newcommand{\destroyf}[1]{c_{#1}}

Erzeugungsoperator
$$\createf{n}:\hil_N^{(-)}\rightarrow \hil_{N+1}^{-}\qquad \createf{n}\ket{0} = \ket{n}\qquad \createf{n}\ket{m} = \sqrt{2}\ket{nm}^{(-)}\qquad ...$$

$$\boxed{\ket{n_1,...,n_N}^{(-)} = \frac{1}{\sqrt{N!}}\createf{n_1}\cdots \createf{n_N} \ket{0}}$$

Vernichter: $\destroyf{n} := (\createf{n})^\dagger$\par 

\paragraph{Vertauschungsrelationen} Vorgehen analog zu Bosonen.

$$\{\createf{n_1}\createf{n_2}\} = 0$$

$$c_n\ket{m_1...m_N}^{(-)} = \frac{1}{\sqrt{N}}\sum (-1)^{i-1}\delta_{nm_i}\ket{m_1...m_{i-1}m_{i+1}...m_N}^{(-)}$$

\begin{align*}
\{\destroyf{n},\createf{m}\}&=\delta_{nm} :=\sprod{n}{m} \\
\{\createf{n},\createf{m}\} &= 0 \\
\{\destroyf{n},\destroyf{m}\} &= 0
\end{align*}
Diese Vertauschungsrelationen beschreiben die Fermi-Statistik.

\subsection{Besetzungszahldarstellung}

1.-T.-Basiszustände $\ket{\psi_n}$, Mehr-T.-Zustände z.B. $\create{\psi_{n_1}}\create{\psi_{n_2}}\create{\psi_{n_3}}\ket{0} = \sqrt{3!}\ket{\psi_{n_1}\psi_{n_2}\psi_{n_3}}$\par 
Äquivalente Charakterisierung: ``... Teilchen mit Zustand $\psi_i$'' $\rightarrow$ \textit{Besetzungszahldarstellung} (Nur sinnvoll für sym./antisym. Zustände, mit abzählbarer Basis)
$$\ket{\psi_1\psi_3\psi_6}^{(\pm)} \quad \hat{=} \quad \ket{1,0,1,0,0,1,0,0,...}$$
Häufig andere Normierung genutzt:\par 
Bsp: 
$$\create{\psi_{1}}\create{\psi_{2}}\create{\psi_{3}}\ket{0} = \sqrt{3!}\ket{\psi_1\psi_3\psi_6}^{(\pm)}$$
$$\ket{\psi_1\psi_3\psi_6}^{(\pm)}=\frac{1}{3!}\sum_{\mathcal{P}}(\pm 1)^\mathcal{P}\ket{\psi_1\psi_3\psi_6}$$
$$\sprod{\psi_1\psi_3\psi_6}{\psi_1\psi_3\psi_6}^{(\pm )} = \frac{1}{3!}$$

$$\create{\psi_{1}}\create{\psi_{1}}\create{\psi_{5}}\ket{0} = \sqrt{3!}\ket{\psi_1\psi_1\psi_5}^{(+)}$$
$$\ket{\psi_1\psi_1\psi_5}^{(+)}=\frac{1}{3!}\sum_{\mathcal{P}}\ket{\psi_1\psi_1\psi_5}$$
$$\sprod{\psi_1\psi_1\psi_5}{\psi_1\psi_1\psi_5}^{(+ )} = \frac{2!}{3!}$$

Allgemein: 
\begin{itemize}
\item Falls jeder Zustand maximal einfach besetzt ist, dann Umnormierung mit $\sqrt{N!}$
\item Falls Zustände Besetzungszahlen $n_1, n_2,...,$ haben, dann Umnormierung mit
$$\sim \sqrt{\frac{N!}{n_1!n_2!...}}$$
\end{itemize}
\paragraph{Besetzungszahldarstellung normiert}
$$\ket{n_1,n_2,...}=\pm \cdots \frac{\left(\create{\psi_2}\right)^{n_2}}{\sqrt{n_2 !}} \frac{\left(\create{\psi_1}\right)^{n_1}}{\sqrt{n_1 !}}\ket{0}$$
Vorzeichen für Bosonen immer $+$.

\subsection{Formulierung von Observablen}

Immer entweder Bose/Fermi aber immer mit $a^\dagger$/$a$\par 

Nehme direkte normierte Basis $\ket{\psi_n}$\par

Besetzungszahloperator:
\newcommand{\anzahl}[1]{\hat{n}_{\psi_{#1}}}
\newcommand{\vac}{\ket{0}}

$$\hat{n}_{\psi_k} := \create{\psi_k}\destroy{\psi_k}$$
$$\anzahl{k}\vac =  0$$

Vertauschungsrelation:
$$\anzahl{k}\create{\psi_l}=\create{\psi_k}\destroy{\psi_k}\create{\psi_l} = \create{\psi_l}\anzahl{k}+\delta_{kl}\create{\psi_l}$$
$$[\anzahl{k},\create{\psi_l}]=\delta_{kl}\create{\psi_l} \qquad [\anzahl{k},\left(\create{\psi_l}\right)^{n_l}]=n_l\delta_{kl}\left(\create{\psi_l}\right)^{n_l}$$

$$\anzahl{k}\ket{n_1,n_2,...,n_k,...} = n_k\ket{n_1,n_2,...,n_k,...}$$

Teilchenzahloperator:
$$N :=\sum_k\anzahl{k}$$

\newcommand{\cre}[1]{\create{\psi_{#1}}}
\newcommand{\des}[1]{\destroy{\psi_{#1}}}

Nebenrechnung:
$$\cre{k}\des{l}\ket{\psi_{n_1},...,\psi_{n_N}}^{(\pm )}=\sum_{m=1}^N \delta_{ln_m}\ket{\psi_{n_1},...,\psi_k,...,\psi_{n_N}}^{(\pm )}$$
(wobei $\psi_k$ den Zustand $\psi_{n_m}$ ersetzt)\par 

Einteilchenobservablen: z.B. kinetische Energie:

$$T_1 = \frac{\vec{p}^2}{2m}$$
$$T_N = \sum_{m=1}^N T_1^{(m)}$$

Matrixelement zw. 1-T.-Zuständen
$$\erwop{\psi_k}{T_1}{\psi_l} =: T_{kl}$$
$$T_1\ket{\psi_l} = \sum_k T_{kl}\ket{\psi_k}$$

Wirkung auf $N$-T.-Zustand:
$$T_N\ket{\psi_{n_1}...\psi_{n_N}}^{(\pm)} = \sum_{m=1}^N\sum_k T_{kn_m}\ket{\psi_{n_1},...,\psi_k,...,\psi_{n_N}}^{(\pm )}$$
Vergleich mit Nebenrechnung:
$$\boxed{T_N = \sum_{k,l}T_{kl}\:\cre{k}\des{l}}$$

\paragraph{Zwei-Teilchen-Observablen} (z.B. Coulomb-Potential zwischen Teilchen $i$,$j$)
$$V_2^{(ij)} : \erwop{\psi_{k_1}\psi_{k_2}}{V_2^{(12)}}{\psi_{l_1}\psi_{l_2}}=: V_{k_1,k_2,l_1,l_2}$$

$$V = \frac{1}{2}\sum_{i\neq j} V_2^{(ij)}$$

analog

$$V = \frac{1}{2}\sum_{k_1k_2l_1l_2} V_{k_1,k_2,l_1,l_2}\cre{k_1}\cre{k_2}\des{l_2}\des{l_1}$$

Beispiel Impulsbasis:\par 

Nicht diskret, sondern kontinuierlich. Impuls-EZ für 1 Teilchen $\ket{\vf{p}}$, W.fkt $\sprod{\vf{x}}{\vf{p}}=\frac{1}{\sqrt{2\pi}^3}e^{i\vf{p}\cdot\vf{x}}$
$$\sprod{\vf{p}'}{\vf{p}} = \delta^{(3)}(\vf{p}-\vf{p}')$$
Erzeuger/Vernichter, kontinuierlicher Index:
$$[\destroy{\vf{p}},\create{\vf{p}'}]=\delta^{(3)}(\vf{p}-\vf{p}')$$
Alles analog mit $\sum_{k,l} \rightarrow \int\dif^3p\:\dif^3p'$

$$T = \int \dif^3p\:\dif^3p' \: \frac{\vf{p}^2}{2m}\: \delta^{(3)}(\vf{p}-\vf{p}') \create{\vf{p}'}\destroy{\vf{p}}$$

$$\boxed{T=\int\dif^3p \: \frac{\vf{p}^2}{2m} \:\create{\vf{p}}\destroy{\vf{p}}}$$

2-Teilchen-Potentiale in Impulsbasis:\par 

im Ortsraum: $V_2^{(12)}=V(\vf{x}_1-\vf{x}_2)$:

$$V = \frac{1}{2}\sum\int V_{\vf{p}_1'\vf{p}_2'\vf{p}_1\vf{p}_2}\create{\vf{p}_1'}\create{\vf{p}_2'}\destroy{\vf{p}_2}\destroy{\vf{p}_1}$$

\begin{align*}
V_{\vf{p}_1'\vf{p}_2'\vf{p}_1\vf{p}_2} &= \erwop{\vf{p}_1'\vf{p}_2'}{V_2}{\vf{p}_1\vf{p}_2} \\
&=\frac{1}{(2\pi )^6}\int\dif^3x_1\:\dif^3x_2\: e^{i(\vf{p}_1\vf{x}_1 + \vf{p}_2\vf{x}_2 - \vf{p}_1'\vf{x}_1 - \vf{p}_2'\vf{x}_2)}\: V(\vf{x}_1-\vf{x}_2) \\
&= \delta^{(3)}(\vf{p}_1+\vf{p}_2 - \vf{p}_1' -\vf{p}_2')\cdot \frac{1}{(2\pi )^3}\int d^3z\:e^{i\vf{z}\vf{q}}V(\vf{z})
\end{align*}
mit $\vf{z}=\vf{x}_1-\vf{x}_2$ und $\vf{q}=\vf{p}_2'-\vf{p}_2$

$$\boxed{V = \frac{1}{2}\int \dif^3p_1\:\dif^3p_2\:d^3q\:\frac{1}{(2\pi )^3}\tilde{V}(\vf{q})\create{\vf{p}_1+\vf{q}}\create{\vf{p}_2-\vf{q}}\destroy{\vf{p}_2}\destroy{\vf{p}_1}}$$

\subsection{Kurz-Überblick über Anwendungen}

System identischer Teilchen mit 2.-T.-WW, endl. Volumen

$$H=\underbrace{T+V_{ext}}_{H_0}+V_2$$

Wähle 1-T-Basis aus $H_0$ Eigenzuständen:
$$H_0\ket{\psi_n}=E_n\ket{\psi_n}$$
Zugehöriger Erzeuger: $\create{n}$
$$T+V_{ext}=\sum_n E_n\create{n}\destroy{n}$$
$V_2$ in Impulsbasis:
$$V_2 = \frac{2}{L^3}\sum_{\vf{p},\vf{p}',\vf{q}}\tilde{V}(\vf{q})\create{\vf{p}-\vf{q}}\create{\vf{p}'+\vf{q}}\destroy{\vf{p}'}\destroy{\vf{p}}$$

Genereller Hamiltonian für Festkörperelektronen mit spinunabhängigem $V_2$

$$H = \sum_{\vf{k},\sigma}\xi_{\vf{k}}\createf{\vf{k}\sigma}\destroyf{\vf{k}\sigma} + \frac{1}{2L_3}\sum_{\vf{k}_1\vf{k}_2\vf{q}\sigma_1\sigma_2}\tilde{V}(\vf{q})\createf{\vf{k}_1-\vf{q},\sigma_1}\createf{\vf{k}_2+\vf{q},\sigma_2}\destroyf{\vf{k}_2\sigma_2}\destroyf{\vf{k}_1\sigma_1}$$

Genereller Hamiltonian für Bosegas mit Wechselwirkung:

$$H=\sum_\vf{p}\frac{\vf{p}^2}{2m}\create{\vf{p}}\destroy{\vf{p}}+\frac{1}{2L^3}\sum_{\vf{p}_1\vf{p}_2\vf{q}}\tilde{V}(\vf{q})\create{\vf{p}_1-\vf{q}}\create{\vf{p}_2+\vf{q}}\destroy{\vf{p}_2}\destroy{\vf{p}_1}$$

\paragraph{Anwendung: Hartree-Fock-Näherung}\mbox{}\par 

$\createf{A}\createf{B}\destroyf{C}\destroyf{D}$ = Produkt aus Paar $AB$ oder aus Paar $A'B'$

$$A=\createf{A}\destroyf{D}\qquad B = \createf{B}\destroyf{C}\qquad A'=\createf{A}\destroyf{C}\qquad B'=\createf{B}\destroyf{D}$$

\newcommand{\erw}[1]{\langle #1\rangle}

\textit{Mean-field-Näherung}:
$$A=\erw{A}+\delta A\qquad B=\erw{B}+\delta B$$
$$AB = (\erw{A}+\delta A)(\erw{B}+\delta B) \approx \erw{A}B + A\erw{B} - \erw{A}\erw{B}$$
H.F.-Näherung:
$$\createf{}\createf{}cc \approx A\erw{B}+B\erw{A}-\erw{A}\erw{B}-A'\erw{B'}-B'\erw{A'}+\erw{A'}\erw{B'}$$
(s. Wick-Theorem)
$$\Rightarrow\quad H^{\text{genähert}}\approx \text{Summe von Termen }\sim c^\dagger c$$

\paragraph{Weitere Anwendungsbeispiele:} Supraleitung, Suprafluidität

\section{Ortsraum, Impulsraum, QFT (Spin=0)}

\subsection{Zur Interpretation der letzten Ergebnisse}

$$H=T+V$$

In Impulsbasis:
$$H=\int\dif^3p\:\frac{\vf{p}^2}{2m}\create{\vf{p}}\destroy{\vf{p}}+\frac{1}{2}\int\dif^3p_1\:\dif^3p_2\:\dif^3q\: \tilde{V}(\vf{q})\create{\vf{p}_1-\vf{q}}\create{\vf{p}_2+\vf{q}}\destroy{\vf{p}_2}\destroy{\vf{p}_1}$$

\begin{itemize}
\item freier Anteil: Summe über harmonische Oszillatoren mit $\omega_\vf{p} = \frac{\vf{p}^2}{2m}$.
\item Zahl von Teilchen mit $\vf{p}$: Anregungszahl des entsprechenden Oszillators $n_\vf{p} =\create{\vf{p}}\destroy{\vf{p}}$
\item Bedeutung der Oszillatoren: de Broglie-Wellen der Impuls-Eigenzustände
\item Bedeutung von $\sum$ bzw. $\int\dif^3p$: Oszillatoren sind unabhängig, $T$ beschreibt keine Wechselwirkung zwischen den Teilchen.
\item Wechselwirkungsanteil als Feynmandiagramm. (Siehe Vorlesung)\par 
\small{\textit{Das gegebene Feynmandiagramm enthält zwei Teilchen mit Impulsen $\vf{p}_1$ und $\vf{p}_2$, die von links kommen und in der Mitte wechselwirken mit $\tilde{V}(\vf{q})$ (Wie ein bosonisches Austauschteilchen gezeichnet). Danach kommen Teilchen mit veränderten Impulsen $\vf{p}_1+\vf{q}$ und $\vf{p}_2-\vf{q}$ rechts raus.}}
\end{itemize}

\subsection{Ortsraum}

Ortsraum-EZ für ein Teilchen: $\ket{\vf{x}}$.\par 

\newcommand{\ccreate}[1]{\hat{\Psi}^\dagger (\vf{#1})}
\newcommand{\ddestroy}[1]{\hat{\Psi} (\vf{#1})}

Erzeuger/Vernichter kontinuierlicher ``Index'': $\create{\vf{x}}$. Andere Bezeichnung $\hat{\Psi}^\dagger (\vf{x})$.

$$[\ddestroy{x},\ddestroy{y}]^{(\pm )}=0$$
$$[\ddestroy{x},\ccreate{y}]^{(\pm )}=\delta^{(3)}(\vf{x}-\vf{y})$$
$$\ccreate{x}\vac = \ket{\vf{x}}$$

Zusammenhang mit Impulsdarstellung:

$$\ccreate{x}=\frac{1}{\sqrt{2\pi}^3}\int\dif^3p\: \create{\vf{p}} e^{-i\vf{x}\cdot\vf{p}}$$
$$\ddestroy{x}=\frac{1}{\sqrt{2\pi}^3}\int\dif^3p\: \destroy{\vf{p}} e^{i\vf{x}\cdot\vf{p}}$$

Dann mit 2-Teilchen-Potential:

$$V = \frac{1}{2}\int\dif^3x_1\:\dif^3x_2\:\ccreate{x_1}\ccreate{x_2}V(\vf{x}_1-\vf{x}_2)\ddestroy{x_2}\ddestroy{x_1}$$

Dann freier 1-Teilchen-Hamiltonian:

\begin{align*}
T&=\int\dif^3x_1\:\dif^3x_2\:\erwop{\vf{x}_2}{\frac{\vf{p}^2}{2m}}{\vf{x}_1}\ccreate{x_2}\ddestroy{x_1} \\
&= \int\dif^3x_1\:\dif^3x_2\:\dif^3p\:\frac{\vf{p}^2}{2m}\frac{e^{i(\vf{p}\vf{x}_2-\vf{p}x_1)}}{(2\pi )^3}\ccreate{x_2}\ddestroy{x_1}
\end{align*}

$\vf{p}^2$-Term durch Laplaceoperator ersetzen und partielle Integration.

\begin{align*}
&= \int\dif^3x_1\:\dif^3x_2\:\dif^3p \frac{e^{i(\vf{p}\vf{x}_2-\vf{p}x_1)}}{(2\pi )^3}\ccreate{x_2}\cdot\frac{-\Delta}{2m}\ddestroy{x_1}
\end{align*}

Integrieren nach $\vf{p}$ ergibt $\delta$-Funktion.

$$\boxed{T=\int\dif^3x\:\ccreate{x}\frac{-\Delta}{2m}\ddestroy{x}}$$

Bedeutung und Vergleich:\par 

$\ddestroy{x}$:
\begin{itemize}
\item Vernichter für Teilchen bei $\vf{x}$. 
\item Definiert auf Fockraum.
\item Auch \textit{Quantenfeldoperator} genannt.
\item Kann auf beliebige Zustände wirken.
\end{itemize}

$\psi (x)$:
\begin{itemize}
\item 1-Teilchen-QM
\item Wellenfunktion $\psi (x) = \sprod{\vf{x}}{\psi}$
\item Charakterisiert einen bestimmten Zustand $\ket{\psi}$
\item Nicht sinnvoll in Mehrteilchentheorie.
\end{itemize}

Die Ähnlichkeit motivierte den historischen Begriff \textit{zweite Quantisierung}.\par 

Relation: im Fockraum gibt es einen 1-Teilchen-Unterraum und 1-Teilchen-Zustände. Präpariere einen 1-Teilchen-Zustand $\ket{\psi}$. Dann:

$$\erwop{0}{\ddestroy{x}}{\psi} = \psi (\vf{x})$$

\subsection{Quantenfeldtheorie und Ortsraum}

Betrachte nur freien Hamiltonian

$$H = \int\dif^3x\:\ccreate{x}\frac{-\Delta}{2m}\ddestroy{x} = \int\dif^3x\:\frac{1}{2m}(\nabla\ccreate{x})(\nabla\ddestroy{x})$$

Umgekehrte Sichtweise: starte von anderem Startpunkt $\longrightarrow$ liefert Fockraum.\par 

Starte mit einer klassischen Feldtheorie mit klassischem Feld $\psi (\vf{x},t)$. (Bekannte klassische Feldtheorien sind die Elektrodynamik und die allgemeine Relativitätstheorie)

\paragraph{Lagrangedichte}

$$\mathcal{L}=i\psi^*\dot{\psi}-\frac{1}{2m}|\nabla\psi |^2$$

\newcommand{\lag}{\mathcal{L}}

\paragraph{Euler-Lagrange-Gleichungen} (Subtilität: $\psi^*$ und $\psi$ als unabhängig betrachten)

$$\frac{\partial\lag}{\partial\psi^*}=\frac{\partial}{\partial x^\mu}\frac{\partial\lag}{\partial\left(\frac{\partial\psi^*}{\partial x^\mu}\right)}$$

$$\Longleftrightarrow\qquad i\dot{\psi}=\frac{-\Delta}{2m}\psi$$

Die Form ist äquivalent zur Schrödingergleichung, die Bedeutung ist hier aber nur die einer klassischen Feldgleichung.\par 

\paragraph{Kanonisch konjugierter Impuls}
$$\pi = \frac{\partial\lag}{\partial\dot{\psi}}=i\psi^*$$

\paragraph{Hamiltonian} $H=\int\dif^3x\:\mathcal{H}$
$$\mathcal{H}=\pi\cdot\dot{\psi}-\lag=\frac{1}{2m}|\nabla\psi |^2$$

\paragraph{Poissonklammern}
$$\{A,B\}_{PK}:=\int\left(\frac{\partial A}{\partial\psi (\vf{x})}\frac{\partial B}{\partial\pi (\vf{x})}-\frac{\partial B}{\partial\psi (\vf{x})}\frac{\partial A}{\partial\pi (\vf{x})}\right)\:\dif^3x$$

$$\frac{\partial\psi (\vf{x})}{\partial\psi (\vf{y})}=\delta^{(3)}(\vf{x}-\vf{y})$$

$$\{\psi (\vf{x}),\pi (\vf{y})\}_{PK} = \delta^{(3)}(\vf{x}-\vf{y})$$
$$\{\psi (\vf{x}),\psi (\vf{y})\}_{PK}=0$$

\paragraph{Quantisierung} Rezept: ``kanonische Quantisierung''

Ersetze $ih\{\; ,\:\}_{PK}\longrightarrow [\; ,\:]$. Die kanonische Quantisierung liefert Operatoren $\ddestroy{x}$, $\hat{\pi}(\vf{x})=i\ccreate{x}$, $\hat{\hil}$.

$$[\ddestroy{x},\ddestroy{y}]^{(\pm )}=0\qquad [\ddestroy{x},\ccreate{y}]^{(\pm )}=\delta^{(3)}(\vf{x}-\vf{y})$$
$$\hat{H}=\int\dif^3x\:\hat{\hil}(\vf{x}) = \int\dif^3x\:\frac{1}{2m}(\nabla\ccreate{x})(\nabla\ddestroy{x})$$

\subsection*{2.5.3. Anhang: Kanonische Quantisierung}

Rezept, um eine sinnvolle Quantentheorie zu definieren.\par 

\begin{tabular}{|p{8cm}|p{8cm}|}
\hline 
Klassische Theorie & Quantentheorie \\
\hline 
ein Paar von ``kanonischen Variablen'' $q$, $\dot{q}$ ``ein Freiheitsgrad'' & Forderung: es existieren Operatoren auf dem Zustandsraum mit $\hat{q}$, $\hat{p}$ und \\
$L(q,\dot{q})\rightarrow q, p=\frac{\partial L}{\partial\dot{q}}$ kanon. konj. Impuls & $\hat{H}=H(\hat{q},\hat{p})$\\
$\rightarrow H(q,p)=p\dot{q}-L$ &  mit $[\hat{A},\hat{B}]=i\hbar \{ A,B\}_{PK}$\\
Bewegungsgleichung $$\frac{\dif}{\dif t}\frac{\partial L}{\partial\dot{q}}=\frac{\partial L}{\partial q}$$ & \\
oder äquivalent $$\dot{q}=\frac{\partial H}{\partial p}\quad ,\quad \dot{p}=-\frac{\partial H}{\partial q}$$ & \\
oder äquivalent $$\frac{\dif}{\dif t}A = \{ A,H\}_{PK}$$ mit & \\
$$\{A ,B\}_{PK} = \frac{\partial A}{\partial q}\frac{\partial B}{\partial p} - \frac{\partial A}{\partial p}\frac{\partial B}{\partial q}$$ & \\
\hline
\end{tabular}

Verallgemeinerung: viele Variablen $q_1(t)$, $q_2(t)$, $q_3(t)$,... bzw. unendlich viele Variablen und auch kontinuierliche Variablen $q_x(t)=: q(t,x)$ (``Feld'')

\paragraph{Beispiel: Harmonischer Oszillator}

$$L=\frac{m}{2}\dot{q}^2 - \frac{m\omega^2}{2}q^2$$
$$p = \frac{\partial L}{\partial \dot{q}} = m\dot{q}$$
$$H=p\dot{q} -L = \frac{p^2}{2m}+\frac{m\omega^2}{q}q^2$$
Bewegungsgleichung aus Lagrange
$$m\ddot{q}=-m\omega^2q$$
Bewegungsgleichung aus Hamilton
$$\dot{p}=-m\omega^2q\quad , \quad \dot{q}=\frac{p}{m}$$
QT: Operatoren $\hat{q}$, $\hat{p}$ mit $[\hat{q},\hat{p}]=i\hbar$
$$\hat{H}=\frac{\hat{p}^2}{2m}+\frac{m\omega^2}{2}\hat{q}^2$$
Aus dem Kommutator $[\hat{q},\hat{p}]=i\hbar$ folgt
$$\sprod{x}{p}=N\cdot e^{ipx/\hbar }$$
In Ortsdarstellung folgt
$$\hat{p}=-i\hbar\frac{\partial}{\partial x}$$
Warum ist das Rezept sinnvoll? $\rightarrow$ Die so erzeugte QT reproduziert die ursprüngliche klassische Theorie im klassischen Limes!\par 

Beispiel:
$$\hbar\frac{\dif}{\dif t}\erwop{\psi}{\hat{p}}{\psi}=\erwop{\psi}{i[\hat{H},\hat{p}]}{\psi}$$
$$[\hat{H},\hat{p}]=m\omega^2\hat{q}i\hbar$$
$$\Rightarrow \frac{\dif}{\dif t}\erw{\hat{p}} = -m\omega^2\erw{\hat{q}}$$
D.h. die QT liefert, dass die Erwartungswerte die klassischen Bewegungsgleichungen erfüllen!\par 

\paragraph{Ehrenfest-Theorem}
$$\frac{\dif}{\dif t}\hat{A}=\frac{i}{\hbar}[\hat{H},\hat{A}]$$
(Entspricht der klassischen Bewegungsgleichung mit Poissonklammern)

\subsection{Quantenfeldtheorie und Impulsraum}

Startpunkt: $\lag = i\psi^*\psi - \frac{1}{2m}|\nabla\psi |^2$.\par 

Wie kann man die Struktur des Zustandsraums ermitteln?\par 
Ansatz: neue Operatoren $a_{\vf{p}}$:
$$\hat{\Psi} (\vf{x})=\frac{1}{\sqrt{2\pi}^3}\int \dif^3p \: a_{\vf{p}} \: e^{i\vf{p}\cdot\vf{x}}$$
simple Rechnung erzeugt Vertauschungsrelationen:
$$[a_{\vf{p}},a_{\vf{p}'}]=0\qquad ,\qquad [a_{\vf{p}},a^\dagger_{\vf{p}'}]=\delta^{(3)}(\vf{p}-\vf{p}')$$
Hamiltonian wird zu
$$\hat{H}\int\dif^3p\:\frac{\vf{p}^2}{2m}\create{\vf{p}}\destroy{\vf{p}}$$

\textcolor{red}{
[HIER FEHLT EIN GANZES STÜCK ZUM ERZEUGTEN FOCKRAUM UND ZUR INTERPRETATION DER ZUSTÄNDE - Vorlesung 16.06.2021]}

\subsection{Relativistische Quantenfeldtheorie}

Frage: Kausalität? (\textcolor{red}{Hier wurde viel gezeichnet... - Vorlesung 17.06.2021})\par 

Man hat gezeigt, dass bei der Zeitentwicklung eines Orts-EZ zu einem späteren Zeitpunkt das Teilchen mit einer Wahrscheinlichkeit $\neq 0$ an einem raumartig getrennten Ort auftauchen kann $\rightarrow$ nicht kausal!\par 

\paragraph{Umformulierung mit Feldoperatoren}

$$\ket{\vf{x}}=\ccreate{x}\vac$$
Heisenberg-Bild:
$$\hat{\Psi}_H(\vf{x},t)=e^{i\hat{H}t}\ddestroy{x}e^{-i\hat{H}t} = \frac{1}{\sqrt{2\pi}^3}\int\dif^3p\:\destroy{\vf{p}}\: e^{-iE_{\vf{p}}t+i\vf{p}\vf{x}}$$

$$\hat{\Psi}^\dagger_H(\vf{x},t)\vac = e^{i\hat{H}t}\ccreate{x}\vac = e^{i\hat{H}t}\ket{\vf{x}}$$
Kausalität: Übergang von Ereignis am Koordinatenursprung zu Ereignis am Raumzeitpunkt $\vf{x},t$ (Ortseigenzustände mit Zeitentwicklung)
$$\sprod{\vf{x},t}{0,0}=\erwop{0}{\hat{\Psi}_H(\vf{x},t)\hat{\Psi}^\dagger_H(0,0)}{0}=\erwop{0}{[\hat{\Psi}_H(\vf{x},t),\hat{\Psi}^\dagger_H(0,0)]}{0}$$
Problem:
$$[\hat{\Psi}_H(\vf{x},t),\hat{\Psi}^\dagger_H(0,0)]=\frac{1}{\sqrt{2\pi}^2}\int\dif^3p\:e^{-iE_\vf{p}t+i\vf{p}\vf{x}}=:\Delta (\vf{x},t)\neq 0\quad \text{(sogar für }|\vf{x}|>ct\text{)}$$
Lösungsmöglichkeiten:
\begin{itemize}
\item $E_\vf{p}=\sqrt{\vf{p}^2+m^2}$ korrekt relativistisch
\item Lorenzinvariantes Integralmaß
$$\int\dif^3p\longrightarrow\int\dif^4p\:\delta ((p^0)^2-\vf{p}^2-m^2)\theta (p^0)$$
Damit ist $\hat{\Psi}_H(\vf{x},t)$ ein Skalarfeldoperator (unter Lorentztransformation). Vertauschungsrelation nicht wesentlich geändert, aber $\Delta (\vf{x},t)$ ist jetzt lorentzinvariant.
$$\Delta (x^\mu )=\Delta ({\Lambda^\mu}_\nu x^\nu)$$
\item Es muss einen zweiten Teilchentyp geben, welcher die selbe Energie-Impuls-Relation haben sollte (selbe Ruhemasse), aber neue Erzeuger $b^\dagger_\vf{p}$, $\hat{\Phi}_H (\vf{x},t)$.\par 

Kausaler Feldoperator:
\begin{align*}
\hat{\Psi}_{\text{kausal}}(\vf{x},t) &= \hat{\Psi}_H(\vf{x},t)+\hat{\Psi}_H^\dagger (\vf{x},t) \\
 &= \int\dif^4p\:\delta (p^2-m^2)\theta (p^0)\left[ a_{\vf{p}}e^{-iE_{\vf{p}}t+i\vf{p}\vf{x}} + b^\dagger_{\vf{p}}e^{iE_{\vf{p}}t-i\vf{p}\vf{x}}\right]
\end{align*}

$$[\hat{\Psi}_{\text{kausal}}(\vf{x},t),\hat{\Psi}^\dagger_{\text{kausal}}(0,0)=\Delta (x^\mu )\mp \Delta (-x^\mu )$$

Falls $|\vf{x}|>ct$: $-x^\mu$ und $x^\mu$ gehen durch Lorentztransformation ineinander über.\par 
$\Rightarrow$ Aufgrund der Lorentzinvariant von $\Delta$ ist der Kommutator für raumartig getrennte Bosonen $=0$.

\paragraph{Bedeutung}

\begin{itemize}
\item relativistische QT muss QFT sein, um Kausalität zu ermöglichen
\item Antiteilchen müssen existieren mit selber Ruhemasse $\rightarrow$ Fundamentale Vorhersage $\rightarrow$ bestätigt!
\item Theorie aufgebaut aus kausalen Feldoperatoren, d.h. $a_\vf{p}\leftrightarrow b^\dagger_\vf{p}$ tauchen immer gemeinsam auf, z.B. auch im Hamiltonian.\par 
$\Rightarrow$ Teilchenvernichtung $\leftrightarrow$ Antiteilchenerzeugung\par 
Teilchenzahl kann nicht konstant bleiben $\rightarrow$ bestätigt!
\item Das Vorzeichen in $[\: ,\:]^{(\pm )}$ muss $-$ sein $\rightarrow$ Bosonen!\par 
$\Rightarrow$ fundamentale Vorhersage: Bosonen haben ganzzahligen Spin, Fermionen halbzahligen Spin $\rightarrow$ bestätigt!
\end{itemize}

\end{itemize}

\subsection{Ausblick auf QFT für Vielteilchensysteme}

Lineare Kette mit Orten $q_1,\ldots ,q_N$. Bsp. nur nächste-Nachbar-WW.

$$L=\sum_i \frac{m}{2}\dot{q}_i^2 - \sum_i\kappa \frac{(q_{i+i}-q_i)^2}{2}$$

$\Rightarrow$ Vibrationswellen/Schallwellen mit Dispersionsrelation $\omega (k)$\par 

Quantisierung liefert Schallwellenquanten (``Phononen'')


\chapter{Streutheorie}

\section{Grundbegriffe}

\subsection{Motivation}

Interessante Fragen:

\begin{itemize}
\item zeitabhängige Phänomene $\rightarrow$ Prozesse
\item Wie findet man $\hat{H}$ aus gegebenen Energie-EW?
\end{itemize}

\paragraph{Streuung:} Ein Teilchen bewegt sich auf ein Potential zu und wird abgelenkt. Die Ablenkung hängt mit der Struktur des Potentials zusammen.

\paragraph{Beispiel:}
\begin{itemize}
\item elastische Streuung: Natur/innere Struktur der Teilchen ändert sich nicht, $E$, $|\vf{p}|$ bleiben gleich. (z.B. Rutherford, Compton, Bhabha, Rayleigh)
\item inelastische Streuung: Innere Struktur der Teilchen kann sich ändern. (z.B. Photoeffekt,\linebreak $e^- + H^{1s} \rightarrow e^- + H^{2s}$, $pp\rightarrow pp + \pi^0$)
\end{itemize}

\subsection{Übersicht}

\textcolor{red}{[Hier gab es einige Zeichnungen]}\par 

Wichtige Parameter:
\begin{itemize}
\item Streuwahrscheinlichkeit in gewisse Winkel $\rightarrow$ $\frac{\dif\sigma (\theta , \varphi )}{\dif\Omega}$
\item Größe des Streuteilchens / ``Ausdehnung'' des Potentials $\rightarrow$ de-Broglie-Wellenlänge $|\vf{p}|=\frac{h}{\lambda}$ $\leftrightarrow$ Reichweite des Potentials.\par 
Falls $\lambda \ll$ Reichweite $\Rightarrow$ innere Struktur des Potentials auflösbar, sonst sehr einfache Winkelverteilung
\item Stärke des Potentials $\rightarrow$ falls $|V|\ll E$, evtl ``Taylorentwicklung'' in $V$ möglich
\end{itemize}

\paragraph{Themenübersicht}
\begin{itemize}
\item Allgemeine Formulierung:\par 
$S$- $T$-Matrix, $\sprod{\sim\text{freie Teilchen für }t\rightarrow\infty}{\sim\text{freie Teilchen bei }t\rightarrow -\infty , p_i}$\par 
Relation $S$-Matrix $\leftrightarrow$ Wirkungsquerschnitt\par 
optisches Theorem\par 
exakte Gleichungen: Lippmann-Schwinger-Gleichung, Greensche Funktionen $\rightarrow$ Näherungen\par 
Näherungen: zeitabhängige Störungstheorie ($\rightarrow$ Feynmandiagramme der Teilchenphysik)
\item nichtrelativistische elastische Streuung an Potential $V(\vf{x})$\par 
Spezialfall, in obigem enthalten\par 
Streuwelle $\sim f(\theta ,\varphi )\frac{e^{ikr}}{r}$\par 
Relation $f \leftrightarrow $ Wirkungsquerschnitt\par 
Störungstheorie für kleines $V$ $\rightarrow$ Borusche Näherung\par 
Partialwellenentwicklung $\rightarrow$ insbesondere für kleine Reichweite
\item (Themenreihenfolge von unten nach oben)
\end{itemize}

\subsection{Grundstruktur der 3-dimensionalen Streuung (nichtrelativisch, elastisch)}

\newcommand{\ham}{\hat{H}}
\newcommand{\mom}{\hat{\vf{p}}}
\newcommand{\pot}{\hat{V}}
\newcommand{\vx}{\vf{x}}
\newcommand{\vp}{\vf{p}}
\newcommand{\vk}{\vf{k}}

$$\ham = \frac{\mom^2}{2m} + \pot = \ham_0 + \pot$$

Annahme: $|\vx |\cdot V(\vx )\longrightarrow 0$ für $|\vx |\rightarrow\infty$.\par 

\newcommand{\dep}{(\vx ,t)}

Suche Lösung $\psi\dep$ für einfallenden Zustand $E,\vp$.
$$\ham\psi = E\psi\qquad ,\qquad E=\frac{\hbar^2\vf{k}^2}{2m} > 0$$

Vgl. 1D Potentialstufe $V(x)\sim \Theta (a-|x|)$: Einfallende Welle resultiert in einer reflektierten und transmittierten Welle mit Koeffizienten $r$ und $t$.
$$\psi (x) = \left\{ \begin{matrix}
\text{links} & e^{ipx} + re^{-ipx} \\
\text{rechts} & te^{ipx} \\
\text{mitte} & \text{irgendwas}
\end{matrix}\right.$$

\paragraph{Ansatz} 
\begin{itemize}
\item einlaufend: $\phi_\vk(\vx )=e^{i\vk\vx}$
\item auslaufend: $ f(\theta ,\varphi ) \frac{e^{ikr}}{r}$
$$\psi_\text{gesamt} \cong e^{i\vk\vx} + f(\theta ,\varphi ) \frac{e^{ikr}}{r}$$
\end{itemize}
Der Ansatz löst die Schrödingergleichung für $x\rightarrow\infty$.\par 
\textit{Beweis:}
$$\ham\psi = \left[\frac{\hbar^2}{2m} (-i\nabla )^2 + V(\vf{x})\right]\psi = E\psi=\frac{\hbar^2\vk^2}{2m}\psi$$
$$\Leftrightarrow \frac{\hbar^2}{2m}(-\Delta -\vk^2 )\psi (\vf{x}) = -V(\vf{x})\psi (\vf{x})$$
$x\rightarrow\infty$:
$$(\Delta + \vk^2 )\Psi (\vx )=0$$
Das gilt trivialerweise für die einfallende Welle.\par 
Gestreute Welle (Längere Rechnung$\rightarrow$ schreibe $\Delta$ in Kugelkoordinaten)
$$\Delta\psi_\text{streu} \overset{r\rightarrow\infty }{=} (ik)^2f(\theta ,\varphi ) \frac{e^{ikr}}{r}+\mathcal{O}(r^{-2}) = -k^2\psi_\text{streu} $$
Damit haben wir die asymptotische Lösung durch einlaufendes $\phi$ und auslaufende Kugelwelle bestimmt, die interessante Größe ist die \textit{Streuamplitude} $f(\theta ,\varphi )$.\par 

\newcommand{\vj}{\vf{j}}

\paragraph{Messbare Stromdichten} Gegeben sei ein Teilchenstrahl mit einer gegebenen Anzahl an Teilchen pro Zeit pro Fläche. $:=|\vj_\text{ein}|$ (Einlaufende Stromdichte) \par 
Nach Streuung betrachten wir Teilchen in einem infinitesimalen Raumwinkel $\dif\Omega$ und zählen darin die Teilchen pro Zeit. $:=dI_\text{aus}$ (auslaufender Strom)

Das Verhältnis der beiden Größen wird als \textit{differenzieller Wirkungsquerschnitt} $\dif\sigma$ definiert:
$$\dif\sigma\cdot |\vj_\text{ein}| = \dif I_\text{aus}$$
und
$$\dif I_\text{aus} = |\vj_\text{streu}|\cdot r^2\dif\Omega$$
Konkret mit obiger Streulösung:
$$\vf{j}_\text{ein} = \frac{\hbar\vk}{m}\qquad \vf{j}_\text{aus} = \frac{\hbar k \vf{e}_r}{m}\frac{1}{r^2}\cdot |f(\theta ,\phi )|^2 + \mathcal{O}(r^{-3})$$
Damit:
$$\boxed{\dif\sigma = |f(\theta ,\varphi )|^2\:\dif\Omega}$$

\section{Detail-Analyse der Potentialstreuung}

\subsection{Differentialgleichung, Greenfunktion, Integralgleichung}

Gewünscht: $\ham\psi_\vk = E\psi_\vk$, $E=\frac{\hbar^2\vk^2}{2m}$

$$(\Delta +\vk^2)\psi_\vk (\vx ) = v(\vx )\psi_\vk (\vx )\qquad \text{mit}\qquad v(\vx ):=\frac{2m}{\hbar^2}V(\vx )$$

Außerdem gewünscht ist die Randbedingung

$$\psi_\vk (\vx ) \cong e^{i\vk\cdot\vx }+f(\theta ,\varphi )\frac{e^{i\vk\cdot\vx}}{r}$$

Greensche-Funktion obigen Form der Schrödingergleichung:
$$(\Delta +k^2)G(\vx )=\delta^{(3)}(\vx )\qquad\text{mit dem ``Yukawapotential''}\qquad G(\vx )=-\frac{e^{ik|\vx |}}{4\pi |\vx |}$$

Damit lässt sich die Schrödingergleichung zu einer Integralgleichung umformen:
$$\psi_\vk (\vx )=e^{i\vk\cdot\vx}+\int\dif^3x'\: G(\vx - \vx ')v(\vx ')\psi_\vk (\vx ')$$
Bemerkungen:
\begin{itemize}
\item Das ist eine Integralgleichung für $\psi_\vk$ (immer noch nichttrivial)
\item Ist äquivalent zur Schrödingergleichung (Beweis durch Einsetzen)
\item Randbedingung ist auch erfüllt:\par 
Ebene Welle steht schon da, das Integral werten wir für $|\vx |\gg$ Reichweite des Potentials, $|\vx |\gg |\vx '|$ aus:
$$|\vx - \vx '| = |\vx |(1 - \vf{e}_x\cdot\frac{\vx '}{|\vx |}+...) \overset{r\rightarrow\infty}{=} |\vf{x}|-\vf{e}_x\cdot\vx '$$
$$\frac{e^{ik|\vf{x}-\vf{x}'|}}{|\vx -\vx'|}\cong \frac{e^{ikx}e^{-ik'x'}}{x}\qquad\text{mit}\qquad k'=\vf{e}_x'\cdot \vf{k}$$
Damit
$$\psi_k (\vx ) \cong e^{i\vk\cdot\vx} + \frac{e^{ikr}}{r}\cdot\underbrace{\left(-\frac{1}{4\pi}\right)\int\dif^3x'\:e^{-i\vf{k}'\vf{x}'}v(\vx ')\psi_\vk (\vx ')}_{=f(\vf{e}_{\vk'})=f(\theta ,\varphi )}$$
\end{itemize}

\subsection{Bornsche Näherung}

Näherung für schwache Potentiale, $V$=``klein''
$$\psi_{\vk \vx }=\phi_{\vk\vx}+\int_{x'}G_{xx'}v_{x'}{\psi_\vk}_{x'}$$
Variablen umbenennen:
$$\psi_{\vk \vx' }=\phi_{\vk\vx'}+\int_{x''}G_{x'x''}v_{x''}{\psi_\vk}_{x''}$$
Einsetzen der zweiten Gleichung in das erste Integral:
$$\psi_{\vk \vx }=\phi_{\vk\vx} + \int_{x'}G_{xx'}v_{x'}{\phi_\vk}_{x'} +\int_{x'}\int_{x''}G_{xx'}v_{x'}G_{x'x''}v_{x''}{\psi_\vk}_{x''}$$
Die ersten beiden Terme sind bekannt, der letzte enthält noch $\psi$. Durch Iteration dieses Verfahrens verschiebt sich der unbekannte Term weiter nach hinten. Es entsteht eine Potenzreihe in $v(\vx )$.
$$\psi_\vk (\vf{x})=\phi_\vk (\vx )+\psi_\vk^{(1)}(\vx )+...$$
Für alle Ordnungen gibt es eine explizite Form. Oft reichen wenige Ordnungen aus.\par 
Allerdings: es ist nicht klar, ob $\psi_\vk$ als Potenzreihe darstellbar bzw. ob die Potenzreihe konvergent ist.
\paragraph{Ergebnis für Streuamplitude}

$$f=f^{(1)}+f^{(2)}+...$$

$$f^{(n)}=-\frac{m}{2\pi}\int\dif^3x'e^{-i\vk\cdot\vx } V(\vx')\psi_\vk^{(n-1)}(\vx')$$

Besonders interessant: 1. Bornsche Näherung
$$f^{(1)}=-\frac{m}{2\pi}\int\dif^3x'\:e^{i(\vk -\vk')\vx'}V(\vx )$$

\section{Mathematische Methoden - Funktionen in drei oder weniger dimensionaler Physik}

\subsection{Komplexe und reelle Analysis}

\subsubsection{Funktionenräume}
Besonders interessant sind die $p$ Normen und $L^p$ Räume
\begin{align*}
    \norm{f}_p &:= (\int |f(x)|^p \mathrm{d}x)^{1/p} \\
    L^p & := \left\{ f \middle| \norm{f}_p < \infty \right\}
\end{align*}
mit gewissen Definitionsbereichen.

Interessant:
\begin{align*}
    L^1 &&&\int |f(x)| \mathrm{d}x < \infty \\
    L^2 &&&\int |f(x)|^2 \mathrm{d}x < \infty
\end{align*}

\subsubsection{Distributionen}

Die Distributionen $\delta(x), \theta(x), \text{etc.}$ als Element aus dem Dualraum der Testfunktionen $f \mapsto D(f) = ``\int \mathrm{d}x D(x) f(x)''$.
Eine Testfunktion ist nur in einem kompakten Bereich ungleich $0$ und $\infty$ oft differenzierbar.

\subsubsection{Fourier Transformation}

\begin{align*}
    f \in L^1 \implies \tilde{f}(k) := \int \mathrm{d}x e^{-i k x} f(x) \text{ existiert } \forall k 
\end{align*}
Es sind $\tilde{f} \in L^1$ oder $\tilde{f} \notin L^1$ möglich.
Wenn $f$, $\tilde{f} \in L^1$ gilt $f(x) = \frac{1}{2\pi} \int \mathrm{d}k e^{+ i k x} \tilde{f}(k)$ fast überall.

Für quadratintegrable Funktionen ist die Fouriertransformation nicht unbedingt konvergent.
Es gilt das \textbf{Planchorel Theorem} nach dem sich die Fouriertrasformation als Abbildung $f \in L^2 \mapsto \tilde{f} \in L^2$ mit 
\begin{itemize}
    \item falls $f \in L^2$ \textbf{und} $f \in L^1$ enspricht die Fouriertransformation $\tilde{f}(k) = \int \mathrm{d}x e^{-i k x} f(x)$
    \item $\tilde{f}$ ist aber sonst auch definiert, mit selber Schreibweise und $\norm{f}_2 = \norm{\frac{1}{\sqrt{2\pi}} \tilde{f}}_2$
    \item Es gilt die Verallgemeinerung des Inversionstheorems
\end{itemize}

Die Anwendung in der Quantenmehanik ist mit der Wellenfunktion $\Psi \in L^2$ im Hilbertraum der Quantenmechanik $L^2$ und den entsprechenden Fouriertransformationen
\begin{align*}
    \tilde{\Psi}(p) &= \int \mathrm{d}x e^{-i p x} \Psi(x) \\
    \Psi(x) &= \int \mathrm{d}p e^{i p x} \tilde{\Psi}(p)
\end{align*}

Für Distributionen ist die Fouriertransformation über die Testfunktionen definiert.
\subsubsection{Residuensatz für komplexe Analysis}

Sei $f$ holomorph in einem Gebiet $G$, bis auf Pole an Punkten $\{z_n\}$ (haben keinen Häufungswert in $G$)
\begin{align*}
    \int_{\text{Geschl. Weg} T \text{in} G} f(z) \mathrm{d}z = 2\pi i \sum_{z_n} \mathrm{ind}_T(z_n) \mathrm{Res}(f; z_n) 
\end{align*}

Eine wichtige, typische Anwendung ist die Integration im Reellen.

\subsubsection{Beispiel}

Herleitung der Fouriertransformation der Greenfunktion
\begin{align*}
    - \frac{1}{4\pi} \frac{e^{i k r}}{r} = \lim_{\varepsilon \rightarrow 0_+} \int \frac{\mathrm{d}^3 q}{(2\pi)^3} \frac{e^{i \mathbf{q} \mathbf{x}}}{k^2 - q^2 + i \varepsilon}
\end{align*}

Wir beginnen mit $\mathbf{q} \mathbf{x} = q r \cos \theta$ und finden 
\begin{align*}
    \int \frac{\mathrm{d}^3 q}{(2\pi)^3} \frac{e^{i \mathbf{q} \mathbf{x}}}{k^2 - q^2 + i \varepsilon} &= \int \frac{q^2 \mathrm{d}q \mathrm{d}\cos \theta \mathrm{d}\phi}{(2\pi)^3} \frac{e^{i q r \cos \theta}}{k^2 - q^2 + i \varepsilon} \\ 
    &= \frac{2\pi}{(2\pi)^3} \int \frac{q^2 \mathrm{d}q}{k^2 - q^2 + i \epsilon} \frac{1}{i q r} \left(e^{i q r} - e^{-i q r}\right) \\ 
    &= \frac{1}{i (2\pi)^2 r} \int_0^\infty \frac{q \mathrm{d}q}{k^2 - q^2 + i \epsilon} \left(e^{i q r} - e^{-i q r}\right) \\ 
    &= \frac{1}{i (2\pi)^2 r} \int_{-\infty}^\infty \frac{q \mathrm{d}q}{k^2 - q^2 + i \epsilon} e^{i q r} \\ 
\end{align*}

\subsection{dreidimensionale Funktionen, Kugelkoordinaten}

In Kugelkoordinaten:
\begin{align*}
    \Delta = \frac{1}{r^2} \partial_r r^2 \partial_r - \frac{\mathbf{L}^2}{r^2} = \frac{1}{r} \partial_r^2 r - \frac{\mathbf{L}^2}{r^2}
\end{align*}

Kugelflächenfunktionen dienen zur Beschreibung von Funktionen $f(\theta, \phi)$ auf Kugeloberflächen.
Beispielsweise kann die Temperaturverteilung gut durch Kugelflächenfunktionen mit $l=1$ oder $l=2$ oder die Kosmische Hintergrundstrahlung mit $l \approx 200$ modelliert werden.
Die Entwicklung ist 
\begin{align*}
    f(\theta, \phi) = \sum_{l m} c_{l m} Y_{l m}(\theta, \phi)
\end{align*}

Es gilt $Y_{l m} = e^{i m \phi} P_l^{|m|}(\cos \theta)$ mit den Legendrepolynomen $P_l(x) = \frac{1}{2^l l!} \left(\frac{\mathrm{d}}{\mathrm{d}x}\right)^l (x^2 - 1)^l$.
Die Normierung ist dabei $\int_{-1}^1 P_{l}(x) P_{l'}(x) \mathrm{d}x = \frac{2}{2 l + 1} \delta_{l l'}$.

Aus der \textbf{Freien Schrödingergleichung} $(k^2 + \Delta) \Psi = 0$ folgt mit $\Psi(r, \theta, \phi) = R(r) Y_{l m} (\theta, \phi)$
\begin{align*}
    \frac{1}{r} \partial_r^2 r R + \left(k^2 - \frac{l (l+1)}{r^2}\right) R = 0
\end{align*}
Wir verwenden $k r = \rho$, $\partial_r^2 = \partial_\rho^2 k^2$, $R(r) = \xi(kr) = \xi(\rho)$ woraus die sphärische Bessel-Differentialgleichung folgt
\begin{align*}
    \frac{1}{\rho} \partial_\rho^2 \rho \xi + \left(1 - \frac{l (l +1)}{\rho^2} \right) \chi = 0
\end{align*}
Die Lösung sind die 
\begin{align*}
    \xi(\rho) &= j_l(\rho) && \text{sphärischen Besselfunktionen} \\
    \xi(\rho) &= n_l(\rho) && \text{sphärischen Neumannfunktionen, nicht regulär für } \rho\rightarrow 0 \\
\end{align*}

In der asymptotischen Form ($\rho \rightarrow \infty$) vernachlässigen wir $l(l+1)/\rho^2$ und erhalten 
\begin{align*}
    j_l(\rho) &= \frac{\sin (\rho - l \pi/2)}{\rho} && \text{allgemein} \\
    j_l(\rho) &= (-1)^l \rho^l \left(\frac{\mathrm{d}}{\mathrm{d}\rho}\right)^l\frac{\sin \rho}{\rho} && \text{exakt} \\
\end{align*}

Damit ergibt sich die allgemeine reguläre Lösung der freien Schrödingergleichung
\begin{align*}
    \Psi(r, \theta, \phi) = \sum_{l m} c_{l m} j_l (k r) Y_{l m}(\theta, \phi)
\end{align*}
Beispiel: ebene Welle
\begin{align*}
    e^{i \mathbf{k} \mathbf{x}} = e^{i k r \cos \theta} = \sum_{l = 0}^\infty i^l (2l +1) j_l(k r) P_l (\cos \theta)
\end{align*}

\section{Partialwellenmethode, Streuphasen}

\subsection{Partialwellenentwicklung}

Nun: Zentralpotential $V = V(r) \implies f(\theta, \phi) = f(\theta)$

Schrödingergleichung: $H \psi = E \psi$, $E = \hbar^2 k^2 / 2 m > 0$ bzw. $(\Delta + k^2) \psi = v(r) \psi$ mit $v(r) = \frac{2m}{\hbar^2} V(r)$.

Randbedingungen: $\psi(\mathbf{x}) = e^{i \mathbf{k} \mathbf{x}} + f(\theta) \frac{e^{i k r}}{r}$.

Ansatz: Entwicklung durch $Y_{l m}$ aber nur $m=0$ trägt bei.
\begin{align*}
    \psi(r, \theta, \phi) &= \sum_l \frac{u_l(r)}{r} P_l(\cos \theta) \\
    f(\theta) &= \sum_l b_l P_l
\end{align*}

Schrödingergleichung damit $\partial_r^2 u_l + k^2 u_l - \frac{l (l+1)}{r^2} u_l = v(r) u_l$ bzw. mit dem effektiven Potential $v_{\mathrm{eff}} = v(r) + \frac{l (l+1)}{r^2}$ 
\begin{align*}
    (\partial_r^2 + k^2) u_l (r) = v_{\mathrm{eff}}(r) u_l(r)
\end{align*}

Allgemeine Lösung für $r \rightarrow \infty$ mit $v_{\mathrm{eff}} \approx 0$ (kleine Reichweite) ist mir $u_l = \sin kr$ oder $u_l = \cos k r$ über die Besselfunktionen
\begin{align*}
    u_l(r) = c_l \sin \left(k r - l \frac{\pi}{2} + \delta_l \right)
\end{align*}
mit der Streuphase $\delta_l$.

Zur Berechnung:
\begin{itemize}
    \item explizite Lösung der Radialgleichung gegeben
    \item $j_l$ und eventuell $u_l$ tauchen Auf
    \item Randbedingungen einsetzen
    \item Lösung eindeutig bis auf Normierung
    \item Kann asymptotisches Verhalten auswerten und mit $A j_l + B u_l \leftrightarrow \sin \left( k r - l \frac{\pi}{2} + \delta_l\right)$ vergleichen.
\end{itemize}

Vergleich / Auswertung der Randbedingungen:

Die allgemeine Lösung war
\begin{align*}
    \psi &= \sum c_l \frac{\sin \left(k r - l \frac{\pi}{2} + \delta_l \right)}{r} P_l(\cos \theta) \\
    &= \sum \frac{1}{r} \left(\frac{c_l}{2 i} e^{i k r} e^{-i l \frac{\pi}{2}} e^{i \delta} - \frac{c_l}{2 i} e^{-i k r} e^{i l \frac{\pi}{2}} e^{-i \delta}\right) P_l (\cos \theta)     
\end{align*}

Wir haben zusätzlich gefordert
\begin{align*}
    \psi &= e^{i \mathbf{k} \mathbf{x}} + f(\theta) \frac{e^{i k r}}{r} \\
    &= \sum \frac{1}{r} \left(\left[ \frac{2 l + 1}{2 i k} + b_l \right] e^{i k r} - e^{-i k r} (-1)^l \frac{2 l + 1} {2 i k}\right) P_l(\cos \theta)
\end{align*}

Ein Koeffizientenvergleich liefert 
\begin{align*}
    c_l &= 2 i (-1)^l e^{-i l \frac{\pi}{2}} e^{i \delta_l} \frac{2 l + 1}{2 i k} = e^{i l \frac{\pi}{2}} e^{i \delta_l} \frac{2 l + 1}{k} \\
    b_l &= \frac{2 l + 1}{k} e^{i \delta_l} \sin \delta_l
\end{align*}

\subsubsection{Zusammenfassung}

Falls $\delta_l$ bekannt gilt für $r \rightarrow \infty$:
\begin{align*}
    \psi(r, \theta, \phi) &= \sum_l \frac{2 l + 1}{2 k} \left( \left[ -i + 2 e^{i \delta_l} \sin \delta_l \right] \frac{e^{i k r}}{r} + i (-1)^l \frac{e^{-i k r}}{r}\right) P_l(\cos \theta) \\
    e^{i \mathbf{k} \mathbf{x}} &= \sum_l \frac{2 l + 1}{2 k} \left( -i \frac{e^{i k r}}{r} + i (-1)^l \frac{e^{-i k r}}{r}\right) P_l(\cos \theta) \\
    f(\theta) &= \sum_l \frac{2 l + 1}{ k}  e^{i \delta_l} \sin \delta_l  P_l(\cos \theta) 
\end{align*}

\subsection{Optisches Theorem und Wirkungsquerschnitt}

\begin{align*}
    \text{Differentieller Wirkungsquerschnitt} && \frac{\mathrm{d} \sigma}{\mathrm{d} \Sigma} &= |f(\theta)|^2 = \sum_{l l'} \frac{(2 l + 1)(2 l' + 1)}{k^2} e^{i \delta_l - i \delta_{l'}} \sin \delta_l \sin \delta_{l'} P_l P_{l'} \\
    \text{Totaler Wirkungsquerschnitt} && \sigma &= \int \mathrm{d}\sigma = \int \mathrm{d} \Sigma |f(\theta)|^2 = 2\pi \int_{-1}^1 \mathrm{d}\cos \theta |f(\theta)|^2 \\
    &&&=\frac{2\pi}{k^2} \sum_l (2 l + 1) 2 \sin^2 \delta_l 
\end{align*}
mit der Orthogonalität und Normalisierung der Legendrepolynome.
Aufschlüsselung in $\sigma = \sum_l \sigma_l$ mit $\sigma_l = \frac{4\pi}{k^2} (2 l + 1) \sin^2  \delta_l \leq \frac{4\pi}{k^2} (2 l + 1)$ (Unitaritätsschranke).

Nutzen: 
\begin{itemize}
    \item falls $\delta_l$ bekannt $\implies$ $\sigma, \mathrm{d}\sigma$ einfach erhaltbar
    \item $\delta_l$-Bestimmung = Hauptarbeit
    \item oft ausreichend: nur kleine $l$ betrachten, z.B. nur $l=0$ (``s-Wellenstreuung'')
    \item Ungleichung liefert absolute Obergrenze an $\sigma_l$ (kann eventuell durch Bornsche Näherung verletzt sein)
\end{itemize}

\subsubsection{Optisches Theorem}

Wir erinnern uns dass $P_l(1) = 1$ und damit
\begin{align*}
    f(\theta = 0) &= \sum_l \frac{2 l + 1}{k} e^{i \delta_l} \sin \delta_l \\
    \implies \sigma &= \frac{4\pi}{k} \mathrm{Im} f(\theta = 0)
\end{align*}

\begin{itemize}
    \item Relation Wahrscheinlichkeit $\leftrightarrow$ Imaginärteil der Vorwärtsstreuamplitude
    \item QM Wahrscheinlichkeit $\leftrightarrow$ Wahrscheinlichkeitsamplitude
    \item Interpretation: Teilchenzahlerhaltung
    \begin{align*}
        e^{i \mathbf{k} \mathbf{x}} \stackrel{\text{nach Streuung}}{\implies} e^{i \mathbf{k} \mathbf{x}} + f \frac{e^{i k r}}{r}
    \end{align*}
    Gestreuter Anteil muss aus der Vorwärtsrichtung verschwinden. In $\theta=0$-Richtung muss destruktive Interferenz stattfinden.
    Das Ausmaß ist durch das optische Theorem gegeben.
\end{itemize}

\subsection{Kleine Reichweite, kleine Energie}

Streuung bei niedrigen Energien $k \rightarrow 0, k R_0 \ll 1$
Ansatz für Radialgleichung: $(\partial_r^2 + k^2) u(r) = v_{\mathrm{eff}}(r) u(r)$ für $l=0$, $v_{\mathrm{eff}} = v$.
\begin{enumerate}
    \item Lösen innen: $r < R_0$ mit $u_{\mathrm{in}}(0) = 0 \implies u_{\mathrm{in}}$ eindeutig 
    \item Allgemeine Lösung außen $r > R_0$, $v(r)>0$:
    \begin{align*}
        u_a(r) &= A k r j_0(k r) + B k r n_0(k r) \\
        &= A \sin k r + B \cos k r \\
        &= C \sin (k r + \delta_0) = C \sin kr \cos \delta_0 + C \cos kr \sin \delta_0
    \end{align*}
    \item Anschlussbedingung bei $r=R_0$
    \begin{align*}
        \Delta_0 := \frac{u_{\mathrm{in}'(R_0)}}{u_{\mathrm{in}}} &\stackrel{!}{=} \frac{u_a'(R_0)}{u_a(R_0)} \\ 
        &= k \frac{\cos k R_0 \cos \delta_0 - \sin k R_0 \sin \delta_0}{\sin k R_0 \cos \delta_0 + \cos k R_0 \sin \delta_0} 
    \end{align*}
    ergeben:
    \begin{align*}
        \tan \delta_0 = \frac{k \cos k R_0 - \sin k R_0 \Delta_0}{k \sin k R_0 + \cos k R_0 \Delta_0}
    \end{align*}
    Üblicherweise ergibt sich $\tan \delta_0 = -a_0 k$ für $k \rightarrow 0$ mit der Streulänge $a_0$.
    Für $l=0$ folgt damit 
    \begin{align*}
        \sigma = \sigma_0 + \frac{4\pi}{k^2} \sin^2 \delta_0 = 4 \pi a_0^2
    \end{align*}
\end{enumerate}

\subsection{Kurzzusammenfassung Partialwellenmethode und Ergänzungen}

\begin{align*}
    \Psi(\mathbf{x}) = e^{i \mathbf{k} \mathbf{x}} + f(\theta, \phi) \frac{e^{i k r}}{r}
\end{align*}

Zentralpotential: $\mathbf{L}^2$ und $L_z$ sind erhalten. 
Entsprechend kann das Problem in einzelne $l$ zerlegt werden.

Wir betrachten immer ein festes $l$:
\begin{align*}
    \text{einlaufend} && (e^{i \mathbf{k} \mathbf{x}})_{l, |\mathbf{x}| \rightarrow \infty} &\approx \left(A_l^{(0)} \frac{e^{- i k r}}{r} + B_l^{(0)} \frac{e^{i k r}}{r}\right) P_l(\cos \theta) \\
    \text{komplett} && \Psi_l &\approx \left( A_l \frac{e^{-i k r}}{r} + B_l \frac{e^{i k r}}{r}\right) P_l(\cos \theta) \\
    && f_l &= b_l P_l(\cos \theta)
\end{align*}

Interpretation:
\begin{itemize}
    \item Ebene Welle = Ein- und auslaufende Kugelwelle
    \item Teilchenzahlerhaltung
    \begin{align*}
        |A_l^{(0)}| &= |B_l^{(0)}| \\
        A_l^{(0)} &= -(-1)^l B_l^{(0)}
    \end{align*}
    \item Physikalische Randbedingung: Nur die auslaufende Kugelwelle ändert sich
    \begin{align*}
        A_l^{(0)} &= A_l \\
        B_l^{(0)} &\neq B_l
    \end{align*}
    Teilchenzahlerhaltung
    \begin{align*}
        |B_l^{(0)}| &= |B_l|
    \end{align*}
    \item Ansatz: $B_l = B_l^{(0)} + b_l$
    \begin{align*}
        B_l &= B_l^{(0)} + e^{2 i \delta_l} \\
        b_l &= B_l^{(0)} (e^{2 i \delta_l} - 1) = B_l^{(0)} 2 i \sin \delta_l e^{i \delta_l} \\
        &= \frac{2 l + 1}{k} \sin \delta_l e^{i \delta_l}
    \end{align*}
    \item Noch nicht bewiesen ist, warum die Steuphase $\delta_l$ reell ist. Grund: Reelles Potential führt keine Absorption ein.
\end{itemize}

\section{Mathematische Methoden - Funktionalanalysis}

\textbf{Hilbertraum} $\mathcal{H} = \{ \ket{\psi} \}$
\begin{itemize}
    \item Vektorraum auf $\mathbb{C}$
    \item Positiv definites Skalarprodukt
    \begin{align*}
        \braket{\psi}{\phi}; \braket{\phi}{\phi} = \norm{\ket{\phi}}^2 \geq 0
    \end{align*}
    \item Vollständig (jede Cauchyfolge konvergiert)
    \item Separabel $\exists$ abzählbare Menge $\{ \ket{\psi_n} \}$ dicht in $\mathcal{H}$
\end{itemize}

\textbf{Konvergenz}:
\begin{itemize}
    \item Starke Konvergenz:
    \begin{align*}
        (\ket{\psi_n}) &\stackrel{\text{stark}}{\rightarrow} \ket{\psi} \\
        &\iff \forall \epsilon \exists n_0 : \norm{\ket{\psi} - \ket{\psi_n}} \leq \epsilon \forall n > n_0
    \end{align*}
    \item Schwache Konvergenz:
    \begin{align*}
        (\ket{\psi_n}) &\stackrel{\text{schwach}}{\rightarrow} \ket{\psi} \\
        &\iff \forall \ket{\phi} \in \mathcal{H} : \braket{\phi}{\psi_n} \rightarrow \braket{\phi}{\psi_n}
    \end{align*}
    \item Beispiel: Wellenberg konvergiert schwach gegen $0$, aber nicht stark
\end{itemize}

\textbf{Operatoren}:
\begin{itemize}
    \item Beschränkter Operator $A$ $\iff \exists M > 0 \norm{A \ket{\psi}} \leq M \norm{\ket{\psi}} \forall \ket{\psi}$. In der QM gibt es oft unbeschränkte Operatoren wie $x, p, H, \ldots$. Diese sind oft nur auf einer Teilmenge $D(A) \subset \mathcal{H}$ definiert.
    \item Hermitescher Operator $A$ $\iff \forall \ket{\phi}, \ket{\psi} \in D(A): \braket{A \phi}{\psi} = \braket{\phi}{A \psi}$
    \item Selbstadjungierter Operator $A$ $\iff A$ hermitesch und $D(A) = D(A^\dagger)$
\end{itemize}

\textbf{Spektraltheorem}:
$A$ selbstadjungiert bedeutet, dass man ihn in
\begin{align*}
    A &= \int_{-\infty}^\infty \lambda \mathrm{d}E(\lambda) \\
    \mathrm{d}E(\lambda) &= \text{operatorwertiges Integralmaß} \\
    E(\lambda) &= \int_{-\infty}^{\lambda} \mathrm{d}E(\lambda') = \text{Projektionsoperator} \\
    E(+\infty) &= \int_{-\infty}^\infty \mathrm{d}E(\lambda') = \mathbf{1}
\end{align*}
zerlegen können.
In der Dirac-Notation ergibt sich entsprechend (bzw. ist gerechtfertigt)
\begin{align*}
    A &= \int \lambda \ket{\psi_\lambda} \bra{\psi_\lambda} \mathrm{d}\lambda \\
    &= \sum \lambda \ket{\psi_\lambda} \bra{\psi_\lambda}
\end{align*}
Spektrum $\sigma(A)$ Wertebereich der $\lambda$ mit $\mathrm{d}E(\lambda) \neq 0$, d.h.
\begin{align*}
    \int_{-\infty}^\infty \rightarrow \int_{\sigma(A)}
\end{align*}
reicht aus.
\begin{itemize}
    \item diskrete Eigenwerte bzw. diskretes Spektrum: Integral $\rightarrow$ Summe, wobei $\ket{\psi_\lambda}$ normierbare Eigenvektoren $\in \mathcal{H}$ sind.
    \item kontinuierliche Eigenwerte bzw. kontinuierliches Spektrum: Integral nötig, wobei $\ket{\psi_\lambda}$ nicht normierbare Eigenvektoren sind. Nun ist $\ket{\psi_\lambda}\bra{\psi_\lambda} \mathrm{d}\lambda = \mathrm{d}E(\lambda)$ streng sinnvoll.
\end{itemize}

\textbf{Unitäre Operatoren}
Speziell: $U(t) = e^{i A t}$ wobei $A$ selbstadjungiert ist.
Allgemeine Unitaritätsbedingung: $U(t) U^\dagger(t) \stackrel{\text{nicht selbstverst.}}{=} U^\dagger(t) U(t) = \mathbf{1}$

Wichtige Operatoren für Streuproblem:
\begin{itemize}
    \item Freie Teilchen: freier Hamiltonian $H_0$, $D(H_0) \subset \mathcal{H}$, selbstadjungiert, nur kontinuierliches Spektrum (keine Bindungszustände).
    \item Mit Wechselwirkung: $H =  H_0 + V$, $D(H) \subset \mathcal{H}$, selbstadjungiert, evtl. kontinuierliches (Streuzustände mit positiver Energie) und diskretes Spektrum (Bindungszustände). Das Spektrum ist nach unten beschränkt.
\end{itemize}

Optimal: $D(H) \cap D(H_0)$ dicht in $\mathcal{H}$

Unitäre Operatoren: 
\begin{align*}
    U_0(t) &:= e^{-i H_0 t} \quad (t\in \mathbb{R}) \\
    U(t) &:= e^{-i H t}
\end{align*}

\textbf{Resolvente / Greensche Operatoren}
\begin{align*}
    G_0(z) &= (z \mathbf{1} - H_0)^{-1} \quad (z \in \mathbb{C}) \\
    G(z) &= (z \mathbf{1} - H)^{-1} 
\end{align*}
definiert $\forall z \notin \sigma(H_0)$ bzw. $z \notin \sigma(H)$ insbesondere für $z = r + i \gamma$ mit $\gamma \neq 0$.
\begin{itemize}
    \item $G$, $G_0$ existieren
    \item alle Spektralwerte $\propto (r + i \gamma - E)^{-1}$ und damit $0 < |\text{Spektralwert}| \leq \frac{1}{\gamma}$.
    \item beschränkte, invertierbare Operatoren. operatorwertige, analystische Funktionen von $z$
\end{itemize}

\section{Formell, allgemein Streutheorie}

\subsection{Motivation, Übersicht}

Fragen: 
\begin{itemize}
    \item Mathematisch: Existenz von Integralen, saubere Beweise bei z.B. der Integralgleichung $\propto G(\mathbf{x}- \mathbf{x}')$ 
    \begin{align*}
        \int \mathrm{d}^3 x' \frac{e^{i k |\mathbf{x} - \mathbf{x}_|}}{|\mathbf{x} - \mathbf{x}'|} V(\mathbf{x}') \psi_{\mathbf{k}} (\mathbf{x}')
    \end{align*}
    Selbst wenn $V$ und $\psi$ quadratintegrabel sind, ist das Integral $\propto \frac{1}{r}$ und das Ergebnis damit im Allgemeinen $\notin L^2$.
    \item Physikalisch: Wellenpakete mit kleinen $\Delta x_{\text{trans}}$, $\Delta x_{\text{long}}$ und $\Delta p$ anstatt von ebenen Wellen 
    \item Physikalisch: Bedeutung des bisherigen $\Psi_{\mathbf{k}}$
    \item Physikalisch: Weitere mögliche nützliche Relationen zwischen $H$, $V$, $\Psi_{\mathbf{k}}$, $f$ usw.
    \item Physikalisch: Allgemeinere Situationen (Relativistisch/Mehrteilchen?)
\end{itemize}

\subsection{In-Zustände und S-Martix}

Skizzentranskript: Zwei Teilchenstrahlen treffen aufeinander, Wechselwirkung findet statt, Detektoren messen das Ergebnis.

Beschreibung:
\begin{align*}
    &\text{Impulsoperator}& \mathbf{p} \\
    &\text{ ohne Ww}& H_0 &= \frac{\mathbf{p}^2}{2m} \\
    &\text{ mit Ww}& H&= H_0 + V \\
    &\text{freie Impuls-EZe} & \ket{\psi_{\mathbf{k}}} \\
    && \mathbf{p}  \ket{\psi_{\mathbf{k}}} &= \mathbf{k}  \ket{\psi_{\mathbf{k}}} \\
    && H_0  \ket{\psi_{\mathbf{k}}} &= \frac{\mathbf{k}^2}{2m}  \ket{\psi_{\mathbf{k}}} \\
    &\text{freies Wellenpaket} & \ket{\psi_g(t)} &= \int \mathrm{d}^3k g(\mathbf{k}) e^{-i E_k t} \ket{\psi_{\mathbf{k}}} \\
    && \int \mathrm{d}^3 k |g(\mathbf{k})|^2 &= 1
\end{align*}
wobei $g(\mathbf{k})$ bei $\mathbf{k}=\mathbf{p}$ einen Peak hat. 
Die Unschärfe $\Delta p$ ist klein, im Ortstraum ist die Wellenfunktion bei $\pm \Delta x_{\text{long, trans}}$ lokalisiert.

Forderungen Streuzustände, präpariert von Experimentalphysiker:

Präpariert ist der Zustand $\ket{\psi_g(t)}$ (Schrödingerbild bzgl. $H$) so, dass Zustand für $t \rightarrow -\infty$ wie $\ket{\phi_g(t)}$ aussieht!
Mathematisch ist starke Konvergenz
\begin{align*}
    \norm{\ket{\psi_g(t)} - \ket{\phi_g(t)}} \rightarrow 0 (t \rightarrow - \infty)
\end{align*}

Äquivalent:
\begin{align*}
    \ket{\psi_g(t)} &= U(t) \ket{\psi_g(0)}& \ket{\psi_g(0)} &:= \ket{\psi_g^H} \\
    \ket{\phi_g(t)} &= U_0(t) \ket{\phi_g(0)}& \ket{\phi_g(0)} &:= \ket{\phi_g^H}
\end{align*}
und damit die Konvergenzbedingung:
\begin{align*}
    \norm{U(t) \ket{\psi_g^H} - U_0(t) \ket{\phi_g^H}} &\rightarrow 0 \\
    \iff \norm{U_0^\dagger(t) U(t) \ket{\psi_g^H} - \ket{\phi_g^H}} &\rightarrow 0 \\
    \iff \ket{\psi_g^I(t)} &\rightarrow \ket{\phi_g^H}
\end{align*}

Die obige Gleichung beschreibt einen asymptotisch ($t\rightarrow \infty$) freien Zustand, Zeistentwicklung ``stoppt'', Wechselwirkung irrelevant.

\textbf{Entwicklung im Heisenberg-Bild}

\begin{align*}
    \ket{\psi_g^H} = \int \mathrm{d}^3 k g(\mathbf{k}) \ket{\psi_{\mathbf{k}}^{\text{in}}}
\end{align*}
\textbf{In-Zustände}, Definition von $\ket{\psi_{\mathbf{k}}^{\text{in}}}$.

Bedeutung: $\ket{\psi_{\mathbf{k}}^{\text{in}}}$ definiert über diese Integrale, abstrahiert von Wellenpaketen $\ket{\psi_g^H}$.

Heisenberg-Bild: zeitunabhängig, beschreiben System zu jeder Zeit (``Filmrolle''). 
Zustand hat die Eigenschaft nach Faltung mit $g(\mathbf{k})$ asymptotisch in freie Wellenpakete für $t \rightarrow -\infty$ überzugehen.

Diese $\ket{\psi_{\mathbf{k}}^{\text{in}}}$ sind natürlich keine echten $\mathcal{H}$-Elemente, genausowenig wie $\ket{\phi_{\mathbf{k}}}$, sondern nur die $\ket{\psi_g^H}$.
Die Existenz von $\ket{\psi_{\mathbf{k}}^{\text{in}}}$ ist noch nicht streng bewiesen.
Aber sie gilt in allen relevanten Theorien, inklusive relativistischer QFT. (obwohl dort das Ww-Bild nicht exisitert).

\textbf{Out-Zustände}: analog $\ket{\psi_g(t)}$ mit 
\begin{align*}
    \ket{\psi_g(t)} - \ket{\phi_g(t)} &\rightarrow 0 (t \rightarrow +\infty) \\
    \ket{\psi_g^H} &= \int \mathrm{d}^3 k g(\mathbf{k}) \ket{\psi_{\mathbf{k}}^{\text{out}}}
\end{align*}
Diese Zustände werden von den Detektoren gemessen.

\textbf{S-Matrix}:
\begin{align*}
    \ket{i}&= \text{präparierte Anfangszustand, asympt. freies Teilchen} (t \rightarrow -\infty) \\
    &= \int \mathrm{d}^3 k g_i(t) \ket{\psi_{\mathbf{k}}^{\text{in}}}\\
    \ket{f}&= \text{Endzustand im Detektor, asympt. freies Teilchen} (t \rightarrow +\infty) \\
    &= \int \mathrm{d}^3 k g_f(t) \ket{\psi_{\mathbf{k}}^{\text{out}}}
\end{align*}

\begin{align*}
    S_{f i} &= \text{Wahrscheinlichkeitsamplitude bei Präparation von } \ket{i} \text{ am Ende } \ket{f} \text{ zu messen} \\
    &= \braket{f}{e}
\end{align*}

Astraktion von Wellenpaketen:

\begin{align*}
    S_{\mathbf{k}' \leftarrow \mathbf{k}} &= \braket{\psi_{\mathbf{k}'}^{\text{out}}}{\psi_{\mathbf{k}}^{\text{in}}}
\end{align*}

\subsection{Existenz der M{\o}ller-Operatoren}

\begin{align*}
    \Omega_\pm := \text{M{\o}ller-Operatoren}
\end{align*}

\begin{align*}
    U(t) &= e^{- i H t} & \ket{\psi^S(t)} &= U(t) \ket{\psi^H } \\
    U_0(t) &= e^{- i H_0 t} & \ket{\psi^I(t)} &= U_0^\dagger(t) \ket{\psi^S(t) } \\
    W(t) &= U^\dagger(t) U_0(t) 
\end{align*}

Erhoffe eine ``gutartige'' Situation: Alle Operatoren existieren und haben einen gemeinsamen Definitionsbereich und 
\begin{align*}
    \Omega_\pm &:= \lim_{t \rightarrow \mp \infty} W(t)
\end{align*}
existiert stark.
D.h. $\forall \ket{\psi} \in \mathcal{H}$:
\begin{align*}
    \lim_{t \rightarrow \mp \infty} W(t) \ket{\psi} =: \Omega_\pm \ket{\psi}
\end{align*}
exisitert.

\textbf{Theorem (Breuig, Haag, 1963; Kupsch, Sandhas)}

Starker Limes existiert, falls $|V(r)| \leq \frac{C}{r^{1+\epsilon}} \forall \epsilon>0$.

Beweis für einfacheren Fall: 
\begin{itemize}
    \item $V$ sogar quadratintegrabel $\int V^2 r^2 \mathrm{d}r < \infty$.
    \item Zustand $\ket{\psi} \in \mathcal{H}$ sei Gaußsches Wellenpaket
\end{itemize}

Wellenpaket:
\begin{align*}
    \braket{\mathbf{k}}{\psi} &= e^{- \frac{\mathbf{k}^2}{2}\sigma} = g(\mathbf{k}) \\
    \braket{\mathbf{x}}{\psi} &= \int \frac{\mathrm{d}^3 k}{(2\pi)^3} e^{- \frac{\mathbf{k}^2}{2}\sigma} e^{i \mathbf{k} \mathbf{x}} = \text{const. } \sigma^{-\frac{3}{2}} e^{-\frac{\mathbf{x^2}}{2 r}}
\end{align*}

Für Limes betrachte ``Cauchyfolge'', sehr negative $t_2 < t_1 < 0$ und bilde
\begin{align*}
    \norm{W(t_2) \ket{\psi} - W(t_1) \ket{\psi}} &=: \norm{X}
\end{align*}
\begin{align*}
    X &= (W(t_2) - W(t_1)) \ket{\psi} = \left(\int_{t_1}^{t_2} \mathrm{d} t \frac{\mathrm{d} W(t)}{\mathrm{d}t}\right) \ket{\psi} \\
    &= \int_{t_1}^{t_2} \mathrm{d}t e^{i H t} (i H - i H_0) e^{-i H_0 t} \ket{\psi} \\
    \norm{X} &\leq \int_{t_1}^{t_2} \norm{V e^{-i H_0 t} \ket{\psi}}
\end{align*}

Wir benötigen wir 
\begin{align*}
    \braket{\mathbf{x}}{e^{-i H_0 t} | \psi} &=: \psi^0(t, \mathbf{x}) = \int \mathrm{d}^3 k \braket{\mathbf{x}}{\mathbf{k}} \braket{\mathbf{k}}{e^{-i H_0 t} | \psi} \\
    \psi^0(\mathbf{x}, t) &= \int \frac{\mathrm{d}^3 k}{(2 \pi)^3} e^{-\frac{\mathbf{k}^2}{2} \sigma + i \mathbf{k} \mathbf{x} - i E_k t} \\
     &= \int \frac{\mathrm{d}^3 k}{(2 \pi)^3} e^{-\frac{\mathbf{k}^2}{2}( \sigma + \frac{i}{m}t) i \mathbf{k} \mathbf{x}} \\
     &= \text{const. } (\sigma + \frac{i}{m}t)^{-\frac{3}{2}} e^{-\frac{\mathbf{x}^2}{2}(\sigma + \frac{i}{2m}t)^{-1}} \\
    |\psi^0(\mathbf{x}, t)| &\leq \text{const.}' t^{-\frac{3}{2}} \forall \mathbf{x}
\end{align*}

Damit folgt schließlich
\begin{align*}
    \implies \norm{V \ket{\psi^0(t)}}^2 = \int \mathrm{d}^3 x V^2(\mathbf{x}) |\psi^0(\mathbf{x}, t)|^2 \leq C t^{-3}
\end{align*}
und 
\begin{align*}
    \norm{X} &\leq C |t_1^{-\frac{1}{2}} - t_2^{-\frac{1}{2}}| <  C |t_1^{-\frac{1}{2}}|
\end{align*}
Damit ist $W(t) \ket{\psi}$ eine Cauchyfolge in $\mathcal{H}$ und konvergiert.

\subsection{M{\o}ller, Analyse von Jauch (1958) bzw. Bell (1960)}

Oft, z.B. für ein hinreichend schnell fallendes Potential $V$ gelten:

\textbf{Postulate von Jauch:}
\begin{itemize}
    \item $\lim_{t \rightarrow \mp \infty} W(t) \ket{\psi} =: \ket{\psi_{\pm}} $ existiert im starken Sinne $\forall \ket{\psi} \in \mathcal{H}$
    Entsprechend existieren die M{\o}ller Operatoren, die physikalische Idee ist dass $W(t) \ket{\phi_{\mathbf{k}}} \rightarrow \ket{\psi_{\mathbf{k}}^{\text{in, out}}}$
    \item Die $\ket{\psi_+}$ und die $\ket{\psi_-}$ spannen den selben (Teil-)Raum $R = \{\ket{\psi_+}\} = \{ \ket{\psi_-} \}$ von $\mathcal{H}$ auf.
    \item Es gilt: $H$ hat diskretes Spektrum (echte Eigenwerte, Bindungszustände) in Raum $\mathcal{H}_{\text{Bind}}$ und kontinuierliches Spektrum (Streuzustände) in Rest $R$. 
    Entsprechend ist $\mathcal{H} = \mathcal{H}_{\text{Bind}} \oplus R$, d.h. $\ket{\psi_\pm}$ bilden das gesamte kontinuierliche Spektrum von $H$.
\end{itemize}

\textbf{M{\o}ller-Operatoren}
\begin{align*}
    \Omega_\pm :& \mathcal{H} \rightarrow R \\
& \Omega_\pm \ket{\psi} = \lim_{t \rightarrow \mp \infty} W(t) \ket{\psi} =: \ket{\psi_\pm} \\
\implies & \Omega_\pm = \lim_{t \rightarrow \mp \infty} W(t) (\text{stark})
\end{align*}
\begin{itemize}
    \item $\Omega_\pm$ erhält die Norm: $\norm{\ket{\phi_\pm}} = \lim_{t \rightarrow \mp \infty} \norm{W(t) \ket{\phi}} = \norm{\ket{\phi}} $
    \item $\Omega_\pm$ ist nicht unbedingt unitär: (falls $R \subsetneqq \mathcal{H}$, kann nicht $\Omega_+ \Omega_+^\dagger = \mathbf{1}_{\mathcal{H}}$ gelten)
    \item Stattdessen: 
    \begin{align*}
        \Omega_\pm \Omega_\pm^\dagger &= P_R (\text{Projektionsoperator}) & \Omega_\pm^\dagger \Omega_\pm = \mathbf{1}_{\mathcal{H}} 
    \end{align*}
\end{itemize}

Damit ergibt sich der unitäre $S$-Operator:
\begin{align*}
    \boxed{S = \Omega_-^\dagger \Omega_+} 
\end{align*}
\begin{align*}
    S^\dagger S = S S^\dagger = \mathbf{1}_{\mathcal{H}}
\end{align*}

Zusammenhang mit In-Zuständen: 
Klar $\underbrace{\lim_{t \rightarrow \mp \infty} W(t)}_{\Omega_\pm} \ket{\phi_g^H} = \ket{\psi_g^H}$

Abstrahiert von den Wellenpaketen ergibt sich formal (mathematisch durch Faltung mit beliebigen Wellenpaketen) 
\begin{align*}
    \boxed{\Omega_\pm \ket{\phi_{\mathbf{k}}} = \ket{\psi_{\mathbf{k}}^{\text{in, out}}}}
\end{align*}

Die $S$-Matrix folgt analog
\begin{align*}
    \braket{\phi_{\mathbf{k}'}|S}{\phi_{\mathbf{k}}} = \braket{\psi_{\mathbf{k}'}^{\text{out}}}{\psi_{\mathbf{k}}^{\text{in}}} = S_{\mathbf{k}' \leftarrow \mathbf{k}}
\end{align*}
Alle Größen sind mathematisch sinnvoll, ihre Existenz ist gesichert.

\subsubsection{Intertwining-Eigenschaft, Eigenwerte}

\begin{align*}
    \underbrace{\frac{\mathrm{d}}{\mathrm{d}t}W(t)}_{\rightarrow 0 (t \rightarrow \mp \infty)} = i H W(t) - i W(t) H_0 
\end{align*}
\begin{align*}
    \implies \boxed{H \Omega_\pm = \Omega_\pm H_0}
\end{align*}

Damit lassen sich Aussagen über die Spektren der M{\o}ller-operatoren finden.
\begin{align*}
    H_0 \ket{\phi_{\mathbf{k}}} = E_{\mathbf{k}} \ket{\phi_{\mathbf{k}}} \\
    \implies H \ket{\psi_{\mathbf{k}}^{\text{\text{in, out}}}} = E_{\mathbf{k}} \ket{\psi_{\mathbf{k}}^{\text{\text{in, out}}}}
\end{align*}
als kontinuierliches Spektrum von $H$.

\subsection[Integralgleichung für $\Omega$, Lippmann-Schwinger Gleichung]{Integralgleichung für $\mathbf{\Omega}$, Lippmann-Schwinger Gleichung}

Integralgleichung: 
\begin{align*}
    \Omega_+ &= \lim_{\epsilon \rightarrow 0_+} \left(\epsilon \int_{-\infty}^0 e^{\epsilon t} W(t) \mathrm{d}t \right) \\
    &= \lim_{\epsilon \rightarrow 0_+} \left(\epsilon \int_{-\infty}^0 e^{i H t} e^{-i (H_0 + i \epsilon) t} \mathrm{d}t \right)
\end{align*}
mit $\Sigma_+^\dagger$ und $\Sigma_-$ analog.

Die \textbf{Lippmann-Schwinger-Gleichung} ergibt sich dann mit
\begin{align*}
    \ket{\psi_{\mathbf{k}}^{\text{in, out}}} = \ket{\phi_{\mathbf{k}}} + \lim_{\epsilon \rightarrow 0_+} \frac{1}{E_k - H_0 \pm i \epsilon} V \ket{\psi_\mathbf{k}^{\text{in, out}}}
\end{align*}
\begin{itemize}
    \item Die Integralgleichung folgt streng, falls die Jauch-Postulate gültig sind
    \item Lippmann-Schwinger-Gleichung ist formal und gilt für die Faltung mit Wellenpaketen
\end{itemize}

\subsubsection{Beweis-Skizze}

Wähle $\ket{\psi} \in \mathcal{H}$ o.B.d.A. normiert mit Norm $=1$.

Frage: $\lim_{\epsilon\rightarrow 0+}\left(\underbrace{\epsilon \int_{-\infty}^0 e^{\epsilon t} W(t) \ket{\psi} \mathrm{d}t - \Sigma_+ \ket{\psi}}_{=: X} \right) = 0$ ?

Mit $\tau = \epsilon t$, $\mathrm{d}\tau = \epsilon \mathrm{d} t$ ergibt sich
\begin{align*}
    X &= \int_{-\infty}^0 e^\tau W(\frac{\tau}{\epsilon}) \ket{\psi} \mathrm{d} \tau - \int_{-\infty}^0 e^\tau \mathrm{d} \tau \ket{\psi_+} \\
    &= \int_{-\infty}^0 e^{\tau} (W(\frac{\tau}{\epsilon})\ket{\psi} - \ket{\psi_+}) \mathrm{d} \tau \\
    &= \underbrace{\int_{-\sigma}^0 e^{\tau} \underbrace{(W(\frac{\tau}{\epsilon})\ket{\psi} - \ket{\psi_+})}_{\text{Norm} \leq 2} \mathrm{d} \tau}_{\text{Norm} \leq 2 |\sigma|} + \int_{-\infty}^{-\sigma} e^{\tau} (W(\frac{\tau}{\epsilon})\ket{\psi} - \ket{\psi_+}) \mathrm{d} \tau 
\end{align*}
Sei $\delta > 0$ beliebig. 
Dann können wir beide Summanden im zweiten Integral abschätzen.
Wähle $\sigma < \frac{\delta}{4}$, dann ist die Norm des linken Integrals $\leq \frac{\delta}{2}$.
Nach den Voraussetzungen exisitert ein $T$, so dass $\norm{W(t) \ket{\psi} - \psi_+} \leq \frac{\delta}{4}$ $\forall t < -T$.
Wählen wir schließlich $\frac{\sigma}{\epsilon}>T$, d.h. $\epsilon < \frac{\sigma}{T} < \frac{\delta}{4 T}$ folgt dass die Norm des rechten Integrals $< \frac{\delta}{4}$ ist.

Es gilt also $\forall \epsilon < \frac{\delta}{4 \pi}: \norm{\ldots} < \delta$, d.h. der Limes ist bewiesen.

\subsubsection{Beweisskizze für Lippmann-Schwinger-Gleichung}

Integralgleichung auf $\ket{\phi_{\mathbf{k}}}$ liefert $H_0 = E_k = \frac{k^2}{2m}$.
\begin{align*}
    \Omega_+ \ket{\phi_{\mathbf{k}}} &= \lim \epsilon \int_{-\infty}^0 e^{i H t} e^{-i (E_k + i \epsilon) t} \ket{\phi_{\mathbf{k}}} \\
    &= \lim \epsilon \int_{-\infty}^0 e^{-i (E_k - H + i \epsilon) t} \ket{\phi_{\mathbf{k}}} \\
    &= \lim \epsilon \left[ \frac{1}{-i (E_k - H + i \epsilon)} e^{-i (E_k - H + i \epsilon) t}\right]_{-\infty}^0 \ket{\phi_{\mathbf{k}}} \\
    &= \lim i \epsilon \frac{1}{-i (E_k - H + i \epsilon)} \ket{\phi_{\mathbf{k}}}
\end{align*}
Der entstandene Operator entspricht der Resolvente von $H$.
\begin{align*}
    \boxed{\Omega_+  \ket{\phi_{\mathbf{k}}} = \lim i \epsilon G(E_k + i \epsilon)  \ket{\phi_{\mathbf{k}}}}
\end{align*}

Resolventengleichung:
\begin{align*}
    G - G_0 = G_0 V G = G V G_0 
\end{align*}
aus $G_0 (G_0^{-1} - G^{-1}) G$.

Hier aus $ \ket{\phi_{\mathbf{k}}}$ angewandt:
\begin{align*}
    i \epsilon G(E_k + i \epsilon) &= i \epsilon G_0 (E_k + i \epsilon) + G_0 V i \epsilon G(E_k + i \epsilon) = \frac{i \epsilon}{E_k + i \epsilon - E_k} + G_0 V i \epsilon G(E_k + i \epsilon) \\
    &= \mathbf{1} + G_0 V i \epsilon G(E_k + i \epsilon)
\end{align*}

Damit ergibt sich für die oben eingerahmte Gleichung
\begin{align*}
    \Omega_+  \ket{\phi_{\mathbf{k}}} =  \ket{\phi_{\mathbf{k}}} + G_0 V i \epsilon G \ket{\phi_{\mathbf{k}}} =  \ket{\phi_{\mathbf{k}}} + G_0 V \Omega_+  \ket{\phi_{\mathbf{k}}}
\end{align*}
mit $\Omega_+  \ket{\phi_{\mathbf{k}}} = \ket{\psi_{\mathbf{k}}^{\text{in}}}$ folgt 
\begin{align*}
    \ket{\psi_{\mathbf{k}}^{\text{in}}} = \ket{\phi_{\mathbf{k}}} + \frac{1}{E_k - H_0 + i \epsilon} V \ket{\psi_{\mathbf{k}}^{\text{in}}}
\end{align*}

\subsection{Lippmann-Schwinger-Gleichung und S-Matrix}

Zur Erinnerung:
\begin{align*}
    S_{\mathbf{k} \leftarrow \mathbf{k}'} = \braket{\psi_{\mathbf{k}'}^{\text{out}}}{\psi_{\mathbf{k}}^{\text{in}}}
\end{align*}

Aus der LS-Gleichung folgt 
\begin{align*}
    \psi_{\mathbf{k}}^{\text{in,out}} = \ket{\phi_{\mathbf{k}}} + G(E_k \pm i\epsilon) V \ket{\phi_{\mathbf{k}}} 
\end{align*}
und damit
\begin{align*}
    \ket{\psi_{\mathbf{k}}^{\text{in}}} = \ket{\psi_{\mathbf{k}}^{\text{out}}} + \underbrace{(G(E_k + i \epsilon) - G(E_k - i \epsilon))}_{\text{Zeitentwicklung von }t=-\infty\ldots\infty} V \ket{\phi_{\mathbf{k}}}
\end{align*}
Multiplizieren wir von links mit $\ket{\psi_{\mathbf{k}'}^{\text{out}}}$ ergibt sich
\begin{align*}
    S_{\mathbf{k}' \leftarrow \mathbf{k}} &= \braket{\psi_{\mathbf{k}'}^{\text{out}}}{\psi_{\mathbf{k}}^{\text{out}}} + \ket{\psi_{\mathbf{k}'}^{\text{out}}} \underbrace{\left( \frac{1}{E_k + i \epsilon - E_{k'}} - \frac{1}{E_k - i\epsilon - E_{k'}} \right)}_{\text{Klammer}} V \ket{\phi_{\mathbf{k}}} \\
    \text{Klammer} &= \frac{-2 i \epsilon}{(E_k - E_{k'})^2 + \epsilon^2} \rightarrow -2 i \pi \delta(E_k - E_{k'})
    S_{\mathbf{k}' \leftarrow \mathbf{k}} &= \braket{\psi_{\mathbf{k}'}^{\text{out}}}{\psi_{\mathbf{k}}^{\text{out}}} - 2 \pi i \delta(E_k - E_{k'}) T_{\mathbf{k}' \leftarrow \mathbf{k}}
\end{align*}
Die neue Größe $T$ beschreibt dabei den interessanten Steuanteil
\begin{align*}
    T_{\mathbf{k}' \leftarrow \mathbf{k}} = \braket{\psi_{\mathbf{k}}^{\text{out}}|V}{\phi_{\mathbf{k}}} = \braket{\phi_{\mathbf{k}'}|V}{\psi_{\mathbf{k}'}^{\text{in}}}
\end{align*}

\subsection[Zusammenfassung mit $f(\theta, \phi)$]{Zusammenfassung mit $f(\theta, \phi)$}

Wieder ein hinreichend schnell abfallendes $V$ 

Ortsraum:
\begin{align*}
    \braket{\mathbf{x}}{\phi_{\mathbf{k}}} &= e^{i \mathbf{k} \mathbf{x}} \\    
    \braket{\mathbf{x}}{\psi_{\mathbf{k}}^{\text{in}}} &= \psi_{\mathbf{k}}^{\text{in}}(\mathbf{x}) \\
    \braket{\mathbf{x}|G_0(E_k + i \epsilon)}{\mathbf{x}'} &= -\frac{2m}{4\pi} \frac{e^{i k |\mathbf{x} - \mathbf{x}'}}{|\mathbf{x} - \mathbf{x}'|}
\end{align*}

Ortsraumdarstellung von $G_0(E_k + i \epsilon)$ ist gerade die auslaufende Kugelwelle aus 3.2 und 3.3!

Umgekehrt $\lim_{\epsilon \rightarrow 0_+} G_0(E_k + i \epsilon)$ ist die Operatorversion von $-\frac{2m}{4\pi} \frac{e^{i k |\mathbf{x} - \mathbf{x}'}}{|\mathbf{x} - \mathbf{x}'|}$.

LS-Gleichung allgemein:
\begin{align*}
    \ket{\psi_{\mathbf{k}}^{\text{in}}} = \ket{\phi_{\mathbf{k}}} + \lim_{\epsilon \rightarrow 0_+} G_0(E_k + i \epsilon) V \ket{\phi_{\mathbf{k}}^{\text{in}}} \\
    \psi_{\mathbf{k}}^{\text{in}}(\mathbf{x}) = e^{i \mathbf{k} \mathbf{x}} - \frac{2m}{4\pi} \int \mathrm{d}^3 x' \frac{e^{i k |\mathbf{x} - \mathbf{x}'}}{|\mathbf{x} - \mathbf{x}'|} V(\mathbf{x}') \psi_{\mathbf{k}}^{\text{in}}(\mathbf{x'})
\end{align*}
Aus dem Vergleich zu 3.2.1 folgt 
\begin{align*}
    f(\theta, \phi) = -\frac{m}{2\pi} T_{\mathbf{k}' \leftarrow \mathbf{k}}
\end{align*}

Bornsche Näherung: iteriere LS-Gleichung mit $G_0 \equiv G_0(E_k + i \epsilon)$

\begin{align*}
    \ket{\psi_{\mathbf{k}}^{\text{in}}} &= \ket{\phi_{\mathbf{k}}} + G_0 V \ket{\psi_{\mathbf{k}}^{\text{in}}} \\ 
    &= \ket{\phi_{\mathbf{k}}} + G_0 V \ket{\phi_{\mathbf{k}}} +  G_0 V G_0 V \ket{\phi_{\mathbf{k}}} + G_0 V G_0 V G_0 V \ket{\phi_{\mathbf{k}}} + \ldots
\end{align*}

Optisches Theorem allgemein $\iff$ Unitarität der $S$-Matrix.
\begin{align*}
    S^\dagger S &= \mathbf{1} \\
    \iff (\mathbf{1} + i T^\dagger) (\mathbf{1} - i T) &= \mathbf{1} \\
    \iff -i (T^\dagger - T) &= T^\dagger T
\end{align*}
mit $S = \mathbf{1} - i T$ wobei $\braket{\phi_{\mathbf{k}'}|T}{\phi_{\mathbf{k}}} = 2 \pi \delta(E_{k'} - E_k)$.

\section{Weiteres zur allgemeinen Streutheorie}

\subsection{Formulierung im Ww-Bild}

Physikalische Frage: 
\begin{align*}
    S_{\mathbf{k}' \leftarrow \mathbf{k}} = \braket{\phi_{\mathbf{k}'}^{\text{out}}}{\phi_{\mathbf{k}}^{\text{in}}} = \braket{\phi_{\mathbf{k}'} | S}{\phi_{\mathbf{k}}} 
\end{align*}

Allgemeines Wechselwirkungsbild:
\begin{align*}
    &\text{Schrö.-Bild:} & \ket{\phi^S(t)} \\
    &\text{Heis.-Bild:} & \ket{\phi^H} &= U^\dagger(t) \ket{\phi^S(t)} \\
    &\text{Dirac/Ww.-Bild:} & \ket{\phi^I(t)} &= U_0^\dagger(t) \ket{\phi^S(t)} \\
    &&&=W^\dagger(t) \ket{\phi^H}
\end{align*}

Zeitentwicklung im Ww.-Bild:
\begin{align*}
    \ket{\psi^I (t_1)} = U_I(t_1 ,t_0) \ket{\psi^I(t_0)}
\end{align*}
mit unitärem Operator $U_I (t_1, t_0)$.

Einerseits gilt per Definition
\begin{align*}
    U_I(t_1, t_0) &= W^\dagger (t_1) W(t_0) \\
    U_I(t_0, t_0) &= \mathbf{1}
\end{align*}

Andererseits gilt folgende ``Schrödingergleichung'':
\begin{align*}
    i \frac{\dif}{\dif t} U_I(t, t_0) &= i \frac{\dif}{\dif t} (e^{i H_0 t} e^{-i H t} W(t_0)) \\
    &= e^{i H_0 t} (-H_0+ H) e^{-i H t} W(t_0) \\
    &= e^{i H_0 t} V e^{-i H_0 t} \underbrace{e^{i H_0 t} e^{-i H t} W(t_0)}_{U_I}
\end{align*}
\begin{align*}
    \boxed{i \frac{\dif}{\dif t} U_I = V(t) U_I(t, t_0)}
\end{align*}
mit $V_I(t) = e^{i H_0 t} V e^{-i H_0 t}$

\subsubsection{Zusammenhang mit Streuproblem}
\begin{align*}
    S_{\mathbf{k}' \leftarrow \mathbf{k}} = \braket{\phi_{\mathbf{k}'}^{\text{out}}}{\phi_{\mathbf{k}}^{\text{in}}} 
\end{align*}

Schreibe: 
\begin{align*}
    \ket{\psi_{\mathbf{k}}^{\text{in}}} &= W(t_0) \ket{\psi_{\mathbf{k}}^{\text{in}, I}(t_0)} \\
    \ket{\psi_{\mathbf{k}'}^{\text{out}}} &= W(t_1) \ket{\psi_{\mathbf{k}'}^{\text{out}, I}(t_1)}
\end{align*}
\begin{align*}
    \implies S_{\mathbf{k}' \leftarrow \mathbf{k}} = \braket{\phi_{\mathbf{k}'}^{\text{out},I}(t_1) | U_I(t_1, t_0) }{\phi_{\mathbf{k}}^{\text{in},I}(t_0)} 
\end{align*}
Präparierter Anfangszustand:
\begin{align*}
    \ket{\psi_{\mathbf{k}}^{\text{in}, I}(t_0)} = W^\dagger(t_0) \Omega_+ \ket{\phi_{\mathbf{k}}} \rightarrow \ket{\phi_{\mathbf{k}}} \quad (t \rightarrow -\infty)
\end{align*}


\end{document}
\documentclass[11pt,a4paper]{report}
\usepackage[utf8]{inputenc}
\usepackage[german]{babel}
\usepackage{amsmath}
\usepackage{amsfonts}
\usepackage{amssymb}
\usepackage{slashed}
\usepackage{xfrac}
\usepackage{amsbsy}
\usepackage[left=2cm,right=2cm,top=2cm,bottom=2cm]{geometry}

\setlength{\parindent}{0cm}
\setlength{\parskip}{0.2cm}

\renewcommand{\baselinestretch}{1.2}

\renewcommand{\vec}{\boldsymbol}

\author{Joris Josiek}
\title{Skript QT2}

\setcounter{chapter}{-1}
\begin{document}
\maketitle

\tableofcontents

\chapter{Grundstruktur der Quantenmechanik}

\section{Postulate}

\textbf{Essenz:} Doppelspaltexperiment / Stern-Gerlach-Experiment\par 
\textbf{Zustand:} eindeutig / maximal präpariertes physikalisches System, reproduzierbares Verhalten, eindeutige Zeitentwicklung. Beschreibung durch $\left|\psi\right\rangle$ eines Hilbertraums. Linearkombinationen erlaubt!\par 
\textbf{Observablen:} Operatoren $\hat{A}$ (hermitesch, da reelle Eigenwerte $\leftrightarrow$ mögliche Messwerte)\par 
\textbf{Wahrscheinlichkeit:} Für ein Messergebnis $a_n$ ist die Wahrscheinlichkeit $\left|\left\langle a_n|\psi\right\rangle\right|^2$ (normierte Zustände).\par 
\textbf{Erwartungswert:} (Korrollar) $\langle\psi|\hat{A}|\psi\rangle$\par 
\textbf{Zeitentwicklung:} $\hat{H}$ (Hamiltonoperator), $\hat{H}$ sei nicht expl. zeitabh.
$$i\hbar\frac{\mathrm{d}}{\mathrm{d}t}\langle\psi_1|\hat{A}|\psi_2\rangle = \langle\psi_1|[\hat{A},\hat{H}]|\psi_2\rangle$$
\textbf{Schrödinger-Bild}
$$i\hbar\frac{\mathrm{d}}{\mathrm{d}t}|\psi (t)\rangle = \hat{H}|\psi (t)\rangle$$
\textbf{Heisenberg-Bild}
$$|\psi_H\rangle = e^{i\hat{H}t}|\psi (t)\rangle$$
$$\hat{A}_H(t)=e^{i\hat{H}t}\hat{A}e^{-i\hat{H}t}$$
$$i\hbar\frac{\mathrm{d}}{\mathrm{d}t}\hat{A}_H(t)=[\hat{A}_H(t),\hat{H}]$$

\section{Ortsraum, Teilchen in 1D}

Operatoren $\hat{x}$, $\hat{p}$, $[\hat{x},\hat{p}]=i\hbar$.\par 
EZe: $|x\rangle$, $|p\rangle$ (bilden jeweils Basis)\par 

Wellenfunktionen: $\psi (x) :=\langle x|\psi\rangle$, $\tilde{\psi}(p):=\langle p|\psi\rangle$

\chapter{Relativistische Quantenmechanik}

\newcommand{\dt}{\frac{\mathrm{d}}{\mathrm{d}t}}
\newcommand{\dif}{\mathrm{d}}
\newcommand{\vf}[1]{\mathbf{#1}}


\section{Kontinuierliche Symmetrien (Bsp. Rotationsinvarianz)}

Frage: Was ist Drehimpuls?

\subsection{Drehungen in 3D}

($\rightarrow$ Liegruppe $SO(3)$)\par 

Aktive Drehung: Bsp. $\vf{v}'=R_z(\theta )\vf{v}$ (Drehung um Winkel $\theta$ um $z$-Achse)\par 

Infinitesimale Drehungen, $\theta = \varepsilon\rightarrow 0$:
$$R_z(\varepsilon)=\mathbf{1} -i\varepsilon\begin{pmatrix}
0 & -i & 0 \\
i & 0 & 0 \\
0 & 0 & 0
\end{pmatrix} = \mathbf{1} - i\varepsilon \ell_z$$
$$\ell_z = \begin{pmatrix}
0 & -i & 0 \\
i & 0 & 0 \\
0 & 0 & 0  
\end{pmatrix}\qquad\ell_x = \begin{pmatrix}
0 & 0 & 0 \\
0 & 0 & -i \\
0 & i & 0  
\end{pmatrix}\qquad\ell_y = \begin{pmatrix}
0 & 0 & i \\
0 & 0 & 0 \\
-i & 0 & 0  
\end{pmatrix}\qquad (\ell_k)_{i,j}=-i\varepsilon_{ijk}$$

``Generatoren der zugehörigen Lie-Algebra"\par 

Charakteristische Kommutatorrelation: $[\ell_i, \ell_j]=i\varepsilon_{ijk}\ell_k$\par 

Endliche Drehungen: $R_z(\theta ) = \exp\:(-i\theta\ell_z)$

\subsection{Darstellungen}

Eine Darstellung einer Gruppe ist eine Zuordnung: $R\mapsto D(R)$ = Matrix / linearer Operator, mit 
$$D(R_1R_2) = D(R_1)D(R_2)$$
Physikalische Idee: Viele physikalische Größen $\rightarrow$ angeben, wie sie sich unter Drehungen verhält.
\begin{itemize}
\item Impuls: $\vf{p}\longmapsto \vf{p}'=R\vf{p}$
\item Energie: $E\longmapsto E' = E=D(R)E$ mit $\forall R: D(R)=1$
\item Ladung: $Q\longmapsto Q' = Q$
\item Dichte: $\rho \longmapsto \rho ' :\rho'(R\vf{x})=\rho (\vf{x})$
\item Quantenzustand $|\psi\rangle\longmapsto |\psi '\rangle = \hat{D}(R)|\psi\rangle$
\end{itemize}

Generatoren für Darstellungen: $\theta=\varepsilon\rightarrow 0$\par 
$$D(R_z(\varepsilon ))=\vf{1} - i\varepsilon J_z\qquad\text{(Analog für x, y)}$$
mit Operatoren $J_x, J_y, J_z$ wie $D(R_z(\varepsilon ))$, diese sind spezifisch für die Darstellung.
$$D(R_z(\theta ))=\exp\:(-i\theta J_z)$$
$$[J_i, J_j] = i\varepsilon_{ijk}J_k$$
\emph{Die Generatoren jeder Darstellung erfüllen dieselben Vertauschungsrelationen.}

\subsection{Drehungen in der Quantenmechanik}

\newcommand{\ket}[1]{|#1\rangle}
\newcommand{\braket}[2]{\langle #1|#2\rangle}
\newcommand{\erwop}[3]{\langle #1|#2|#3\rangle}

Darstellung von Drehungen:
$$\hat{D}(R_k(\theta )):\ket{\psi}\mapsto\ket{\psi'}=\hat{D}(R_k(\theta ))\ket{\psi}$$
Gruppenstruktur:
$$\hat{D}(R_1R_2)=\hat{D}(R_1)\hat{D}(R_2)$$
Falls Symmetrie:
$$\braket{\psi '}{\phi '} = \braket{\psi}{\phi} \Leftrightarrow\erwop{\psi}{\hat{D}^\dagger\hat{D}}{\phi}$$
$\hat{D}(R)$ ist ein \textit{unitärer} Operator. $[\hat{D}(R), H]=0$.\par 

Infinitesimale Drehung:
$$\hat{D}(R_k(\varepsilon ))=\vf{1}-i\varepsilon\hat{J}_k$$
Falls Symmetrie:
$$[\hat{J}_k, \hat{H}] = 0\qquad [\hat{J}_i, \hat{J}_j]=i\varepsilon_{ijk}\hat{J}_k$$
Per Definition: $\hat{\vf{J}}$ is Drehimpuls dieser Quantentheorie.\par 

Konsequenzen bei solchen $\hat{\vf{J}}$-Operatoren: (QT1)
$$[\hat{J}_z, \hat{\vf{J}}]=0\qquad \hat{J}_\pm =\hat{J}_x\pm i\hat{J}_y$$
Mögliche Eigenzustände: $\ket{j, m}$ mit $j = 0, \frac{1}{2}, 1, \frac{3}{2}, \ldots$ und $m=-j, \ldots , j$\par 

Einfachste nicht-triviale Darstellung: $j=\frac{1}{2}$, d.h. 2-Zustandssystem $\ket{\pm}:=\ket{j=\frac{1}{2},m=\pm\frac{1}{2}}$.
$$\ket{\psi}=\psi_+\ket{+}+\psi_-\ket{-}$$
$$\psi\overset{R_k(\theta )}{\longmapsto}\psi' = \left(\vf{1}-i\theta\frac{\sigma_k}{2}\right)\psi$$
mit Pauli-Matrizen $\sigma_k$.

\section{Lorentzinvarianz}

\subsection{Lorentzgruppe}

Drehungen: $(t,\vf{r})\longmapsto (t,R(\vf{r}))$\par 
Boosts in x-Richtung:
$$\begin{pmatrix}
t \\ x \\ y \\ z
\end{pmatrix}\longmapsto\begin{pmatrix}
\cosh\beta & \sinh\beta & 0 & 0 \\
\sinh\beta & \cosh\beta & 0 & 0 \\
0 & 0 & 1 & 0 \\
0 & 0 & 0 & 1
\end{pmatrix}\begin{pmatrix}
t \\ x \\ y \\ z
\end{pmatrix}$$
Generatoren: $\ell_x, \ell_y, \ell_z$ wie gehabt. Boosts: $\Lambda_x(\beta ) = \vf{1} - i\beta k_x+\mathcal{O}(\beta^2)$
$$k_x=i \begin{pmatrix}
0 & 1 & 0 & 0 \\
1 & 0 & 0 & 0 \\
0 & 0 & 0 & 0 \\
0 & 0 & 0 & 0 
\end{pmatrix}\qquad k_y=i \begin{pmatrix}
0 & 0 & 1 & 0 \\
0 & 0 & 0 & 0 \\
1 & 0 & 0 & 0 \\
0 & 0 & 0 & 0 
\end{pmatrix}\qquad k_z=i \begin{pmatrix}
0 & 0 & 0 & 1 \\
0 & 0 & 0 & 0 \\
0 & 0 & 0 & 0 \\
1 & 0 & 0 & 0 
\end{pmatrix}$$
6 Generatoren:
Vertauschungsrelationen (und zyklisch):
$$[\ell_x, \ell_y]=i\ell_z$$
$$[k_x,k_y]=-i\ell_z$$
$$[\ell_x,k_y]=ik_z$$

\subsection{Darstellungen}

\textbf{Def. Darstellung:} Matrizen/Operatoren $J_i$, $K_i$, mit $[J_x,J_y]=iJ_z$, $[K_x,K_y]=-iJ_z$, $[J_x,K_y]=iK_z$.\par 

Triviale Darstellung: $J_i=0$, $K_i=0$\par 

Spin $\frac{1}{2}$: $J_i=\sigma^i/2$, $K_i=-i\sigma^i/2$. Die Elemente des 2D Darstellungsraumes nennt man \textit{linkshändige Weyl-Spinoren}. (Andere Variante mit $K_i=+i\sigma^i/2$: Elemente sind \textit{rechtshändige Weyl-Spinoren})\par 

\paragraph{Partität/Raumspiegelung $P$:} $\vf{x}\mapsto -\vf{x}$, $\vf{p}\mapsto -\vf{p}$, $\vf{J}\mapsto\vf{J}$, $\vf{K}\mapsto -\vf{K}$. Falls $P$-Transformation genutzt werden soll, sind beide Darstellungen nötig $\Rightarrow$ 4D komplexer Spinorraum aus \textit{Dirac-Spinoren} notwendig.
$$\Psi =\begin{pmatrix}
\psi_\alpha \\ \overline{\psi}^{\dot{\alpha}}
\end{pmatrix}$$
Darstellung für Diracspinoren:
$$J_i=\begin{pmatrix}
\frac{\sigma^i}{2} & 0 \\ 0 & \frac{\sigma^i}{2}
\end{pmatrix}\qquad K_i =\begin{pmatrix}
-i\frac{\sigma^i}{2} & 0 \\ 0 & i\frac{\sigma^i}{2}
\end{pmatrix}$$
Diracspinoren: 4-komponentige komplexe Spinoren. Einfachste Darstellung mit $P$-Transformation.\par 

Lorentztransformationen und Darstellungen:
$${\Lambda^\mu}_\nu = {\delta^\mu}_\nu +{\omega^\mu}_\nu$$
(mit infinitesimalem und antisymmetrischem $\omega^{\mu\nu}$ (wenn beide Indizes oben!), z.B Drehung, Boost)

$$\Lambda = \vf{1} -\frac{i}{2}\omega^{\mu\nu}L_{\mu\nu}$$

mit $L_{ij} = -L_{ji} = \varepsilon_{ijk}\ell_k$ und $L_{i0}=-L_{0i}=k_i$

Für eine Darstellung $S$:
$$\boxed{S(\Lambda ) := \vf{1}-\frac{i}{2}\omega^{\mu\nu}L_{\mu\nu}}$$

\subsection{Diracspinoren und $\gamma$-Matrizen}
$\psi = (\psi_1,\psi_2,\psi_3,\psi_4)$ = komplexer Diracspinor.\par 

\paragraph{Def $\gamma$-Matrizen:} $\{\gamma^\mu ,\gamma^\nu\}=2g^{\mu\nu}\vf{1}$\par 

Weyl-Form:
$$\gamma^0 :=\begin{pmatrix}
0 & \vf{1}_2 \\ \vf{1}_2 & 0 
\end{pmatrix}\qquad\gamma^i \begin{pmatrix}
0 & \sigma^i \\ -\sigma^i & 0
\end{pmatrix}$$
Die Generatoren $\vf{J}$, $\vf{K}$ lassen sich so ausdrücken:
$$S_{\mu\nu}=\frac{i}{4}[\gamma_\mu,\gamma_\nu]$$
Dies reproduziert die Darstellungsmatrix $L_{\mu\nu}$ der Lorentztransformation.\par 

${\gamma^\mu}^\dagger$: ${\gamma^0}^\dagger = \gamma^0$, ${\gamma^i}^\dagger = -\gamma^i = \gamma^0\gamma^i\gamma^0$
$${S^\dagger}_{\mu\nu} = \gamma^0 S_{\mu\nu} \gamma^0$$
$$S^{-1}(\Lambda )=\vf{1}+\frac{i}{2}\omega^{\mu\nu}S_{\mu\nu}=\gamma^0 S^\dagger (\Lambda )\gamma^0$$
Def \textit{Adjungierter Spinor:} $\overline{\psi}:=\psi^\dagger\gamma^0$\par 
Lorentz:
$$\psi\longmapsto S(\Lambda )\psi$$
$$\overline{\psi}\longmapsto \overline{\psi}S^{-1}(\Lambda )$$
$$\overline{\psi}\psi\longmapsto\overline{\psi}\psi$$
$$\overline{\psi}\gamma^\mu\psi\longmapsto {\Lambda^\mu}_\nu\overline{\psi}\gamma^\nu\psi$$
$$S^{-1}(\Lambda )\gamma^\mu S(\Lambda )={\Lambda^\mu}_\nu\gamma^\nu$$

\section{Überblick über relativistische Wellengleichungen}

Welche Gleichungen wären erlaubt durch Lorentzinvarianz?\par 

Notation: \par 
\begin{itemize}
\item 4-Vektoren: $(x^\mu )=(t,\vf{x})$, $(p^\mu)=(E,\vf{p})$
\item Lorentzinvarianten sind Skalarprodukte, z.B. $p^\mu p_\mu = E^2-\vf{p}^2=:m^2$
\item Ableitungen: $\partial_\mu = \left(\frac{\partial}{\partial x^\mu}\right) = (\partial_t,\nabla )$, $\square = \partial_\mu\partial^\mu = \partial_t -\Delta$
\item Elektrodynamik: $j^\mu = (\rho ,\vf{j})$, $\partial_\mu j^\mu = 0$, $A^\mu = (\phi, \vf{A})$, $F^{\mu\nu}=\partial^\mu A^\nu -\partial^\nu A^\mu$\\
Maxwell: $\partial_\mu F^{\mu\nu} = \mu_0 j^\nu$, homogene Gleichung automatisch durch Potentiale erfüllt.\\
Lorentz-Transf.: $x'^\mu ={\Lambda^\mu}_\nu x^\nu$, $j'^\mu (x')={\Lambda^\mu}_\nu j^\nu (x)$
\end{itemize}

\subsection{Klein-Gordon-Gleichung}

$\phi (x)$ sei Skalarfeld ($\phi\mapsto \phi '$ mit $\phi '(x')=\phi (x)$).

$$\boxed{\square \phi (x) + m^2\phi (x) = 0}$$

\paragraph{Interpretation?}
\begin{itemize}
\item Einfachste relativistische Differentialgleichung
\item ``erraten aus QM'' (mit QM Ersetzungsregeln $E\rightarrow i\partial_t$, $\vf{p}\rightarrow -i\nabla$)
\item Nichtrelativistischer Limes: ein Teilchen, $E\approx m + \text{Korrektur}$. Ansatz:
$$\psi (\vf{x},t)=e^{-imt}\psi_{n.r.} (\vf{x},t)$$
$$\Rightarrow \partial_t^2\psi = (-2im\partial_t\psi_{n.r.}-m^2\psi_{n.r.}+\mathcal{O}(\ddot{\psi}))e^{-imt}$$
$$\Rightarrow 2im\partial_t\psi_{n.r.}=-\Delta\psi_{n.r.}$$
\item Klassische Feldgleichung:
$$\mathcal{L}_{KG}=(\partial^\mu \phi^*)(\partial_\mu\phi )-m^2\phi^*\phi$$
Euler-Lagrange:
$$0=\partial_\rho\frac{\partial\mathcal{L}}{\partial (\partial_\rho\phi^*)}-\frac{\partial\mathcal{L}}{\partial\phi^*}$$
\end{itemize}

\paragraph{Rolle als QM Wellengleichung für ein Teilchen in Ortsdarstellung:}
\mbox{}\par
Schrödinger-Gleichung nicht-relativistisch: $i\partial_t\psi = -\frac{\Delta}{2m}\psi$\\
Klein-Gordon-Gleichung: $-\partial_t^2\phi = (-\Delta +m^2)\phi$\par 

Aufenthaltwahrscheinlichkeitsdichte: Suche $(j^\mu )=(\rho ,\vf{j})$ mit Kontinuitätsgleichung $\partial_\mu j^\mu =0$:
$$\phi^* (\square + m^2)\phi -\phi (\square+m^2)\phi^* = 0$$
$$=\partial_\mu [\phi^*\partial^\mu \phi - \phi\partial^\mu\phi^*]$$
Definiere 4-Stromdichte:
$$j^\mu = \frac{i}{2m}\left[\phi^*\partial^\mu\phi -\phi\partial^\mu\phi^*\right]$$
$$\Rightarrow \vf{j}=-\frac{i}{2m}\left[\phi^*\nabla\phi -\phi\nabla\phi^*\right]$$
$$\Rightarrow \rho = \frac{i}{2m}\left[\phi^*\partial_t\phi - \phi\partial_t\phi^*\right]$$
\paragraph{Interpretation}
\begin{itemize}
\item $\rho$ ist nicht positiv definit! $\rho < 0$ möglich! Also kann $\rho$ nicht als Aufenthaltswahrscheinlichkeit interpretiert werden.
\item Lösungen: $\phi\sim e^{-iEt+i\vf{p}\cdot\vf{x}}$: $\rho = \frac{E}{m}>0$, $\rho\sim e^{+iEt-i\vf{p}\cdot\vf{x}}$: $\rho = -\frac{E}{m}<0$: negative Energie möglich!?
\item Idee: KG-Gl. beschreibt zwei Teilchentypen (Teilchen + Antiteilchen) mit entgegengesetzten Ladungen. Interpretiere $\rho$ als elektrische Ladungsdichte.
\end{itemize}

\subsection{Dirac-Gleichung}

$\psi (x)$ sein ``Dirac-Spinorfeld'' d.h. $\psi\mapsto \psi '$ mit $\psi'(x')=S(\Lambda ) \psi (x)$.
$$S(\Lambda ) = \vf{1}_4-\frac{i}{2}\omega^{\mu\nu}S_{\mu\nu}$$
$$S_{\mu\nu} = \frac{i}{4}[\gamma_\mu ,\gamma_\nu ]$$
$$\{\gamma^\mu ,\gamma^\nu\} = 2g^{\mu\nu}\vf{1}_4$$
Dirac-Gleichung:
$$\boxed{ (i\partial_\mu\gamma^\mu - m)\psi =0}$$

\paragraph{Interpretation:}
\begin{itemize}
\item nicht einfachste Differenzialgleichung
\item erraten von Dirac: gewünscht ``Wurzel aus KG-Gleichung'' (Herleitung $\curvearrowright$ Lit.)
\item $\mathcal{L} = \overline{\psi}(i\partial_\mu \gamma^\mu - m)\psi$
\item Adjungierte Dirac-Gl. $i\partial_\mu \overline{\psi}\gamma^\mu + m\overline{\psi} = 0$
$$\Rightarrow \partial_\mu (\overline{\psi}\gamma^\mu\psi )=0$$
\item Def. $j^\mu = \overline{\psi}\gamma^\mu\psi $, $\rho = \psi^\dagger\psi$ ist positiv-definit
\end{itemize}

\paragraph{Vollständige Darstellung der Lorentztransformationen}\mbox{}\par 

$$\psi '(x)=S(\Lambda )\psi (\Lambda^{-1}x) = (\vf{1}-\frac{i}{2}\omega^{\mu\nu}S_{\mu\nu})\psi (x-\omega x)$$
und
$$\psi '=(\vf{1}-\frac{i}{2}\omega^{\mu\nu}\hat{J}_{\mu\nu})\psi$$
($\hat{J}$ Generatoren der Darstellung der Lorentz-Algebra auf dem Fkt.-Raum der Spinorfelder)
$$\Longrightarrow \hat{J}_{\mu\nu}=i(x_\mu\partial_\nu -x_\nu\partial_\mu )+S_{\mu\nu}$$
$$\hat{J}_{\mu\nu}=\hat{L}_{\mu\nu}+S_{\mu\nu}$$
Analog zur KG-Gl. treten Inkonsistenzen auf, wenn man Diracgl. als 1-Teilchen-Theorie auffasst. Die Probleme sind ähnlich aber nicht gleich.

\section{Physik und Lösungen der Diracgleichung}

\subsection{Freie Lösungen, Impuls-/Spin-Eigenzustände}

\newcommand{\dslash}{\slashed{\partial}}
\newcommand{\pslash}{\slashed{p}}

Dirac-Gleichung: $(i\dslash - m)\psi = 0$\\
Gesamt-Drehimpuls: $\hat{J}_{ij}=\hat{L}_{ij}+S_{ij}$. Spin-EZ: $\pm\frac{1}{2}$\par 

Ansatz: $\psi (x)= w(p)e^{\mp ipx}$ (mit $px = p_\mu x^\mu$)
$$\Rightarrow (\pm \pslash - m)w(p)=0$$
Eigenwertgleichung für $\pslash$ !\par 

Beachte: $\pslash^2 = p^\mu\gamma_\mu p^\nu\gamma_\nu = p^\mu p^\nu \gamma_\mu\gamma_\nu = \frac{1}{2}p^\mu p^\nu \{\gamma_\mu ,\gamma_\nu\} = p^2\vf{1}$\par 
D.h. $\pslash$ hat EWe $\pm \sqrt{p^2}$ vermutlich je 2-fach entartet. Nicht-triviale Lösung der EW-Gl. für $p^2=m^2$ $\rightarrow$ Teilchen mit Ruhemasse $m$ beschrieben.

\paragraph{Bezeichnungen der Lösungen}
$$(\pslash - m)u(p,s)=0$$
$$(\pslash + m)v(p,s)=0$$
Beispiel: $p^2 = m^2$, $(p^\mu )=(E,0,0,p_z)$ in $z$-Richtung, $E^2=p_z^2+m^2$.
$$\pslash = p^\mu \gamma_\mu = E\gamma_0 + p_z\gamma_3 = E\gamma^0-p_z\gamma^3=\begin{pmatrix}
\mathbf{1}E & -p_z\sigma^3 \\ p_z\sigma^3 & -\mathbf{1}E
\end{pmatrix}$$
Es gilt $[\pslash , S_{12}]=0$, d.h. $\pslash$ und $S_z$ haben simultane Eigenzustände. (allg. $\pslash$ und $\frac{\vf{p}\cdot\vf{S}}{|\vf{p}|}$ = Helizitätsoperator simultan Diagonalisierbar).\par 
EW-Gleichung lösen:
$$u(p,+\sfrac{1}{2})=N\cdot\begin{pmatrix}
E+m \\ 0 \\ p_z \\ 0
\end{pmatrix}$$
$$u(p,-\sfrac{1}{2})=N\cdot\begin{pmatrix}
0 \\E+m \\ 0 \\ -p_z 
\end{pmatrix}$$
mit $N=\frac{1}{\sqrt{E+m}}$.
$$v(p, +\sfrac{1}{2})=N\cdot\begin{pmatrix}
p_z \\ 0 \\ E+m \\ 0
\end{pmatrix}$$
$$v(p, -\sfrac{1}{2})=N\cdot\begin{pmatrix}
0 \\ -p_z \\ 0 \\ E+m
\end{pmatrix}$$
Spinoren für andere $\vf{p}$: $\vf{p}=R\vf{p}_z = e^{-\frac{i}{2}\omega^{\mu\nu}L_{\mu\nu}}\vf{p}_z$:
$$u(p,s)=e^{-\frac{i}{2}\omega^{\mu\nu}S_{\mu\nu}}u(p_z,s)$$
\paragraph{Negative Energien}
$$\psi (x)=u(p,s)=e^{-iEt+i\vf{p}\cdot\vf{x}}$$
$$\psi (x)=v(p,s)=e^{+iEt-i\vf{p}\cdot\vf{x}}$$
D.h. Energie $(-E)<0$ für $v$-Lösungen.

\subsection{Mehr zum Drehimpuls}

Man betrachte die Diracgleichung als quantenmechanische 1-Teilchen-Gleichung. (sinnvoll, solange Antiteilchen und QFT Effekte vernachlässigbar sind).\par 

Formulierung analog zur Schrödingergleichung im Ortsraum:
$$(i\dslash - m)\psi = 0$$
Multiplikation mit $\gamma^0$ von links und nach Zeitableitung umstellen:
$$i\partial_t\psi = (-i\gamma^0\gamma^i\partial_i+m\gamma^0)\psi =:\hat{H}^{(0)}_D\psi$$
\paragraph{Drehimpuls} aus Darstellung der Lorentztransformation.
$$\hat{J}_{ij}=i(x_i\partial_j-x_j\partial_i)+\hat{S}_{ij} = \hat{L}_{ij}+\hat{S}_{ij}$$
$$\hat{\vf{J}}=\hat{\vf{L}}+\hat{\vf{S}}$$
Es gilt $[\hat{H}^{(0)}_D,\hat{\vf{J}}]=0$, d.h. Gesamtdrehimpuls erhalten. $[\hat{H}^{(0)}_D,\hat{\vf{L}}]=\gamma^0\gamma_1\partial_y-\gamma^0\gamma_2\partial_x$.
\paragraph{Helizität}
$$\frac{\hat{\vf{S}}\cdot\hat{\vf{p}}}{|\hat{\vf{p}}|}$$
$$[\hat{H}^{(0)}_D, \hat{\vf{S}}\cdot\hat{\vf{p}}]=[\hat{H}^{(0)}_D,\frac{1}{2}\epsilon_{ijk}S_{ij}\hat{p}^k]=\sim\frac{1}{2}\epsilon_{ijk}\gamma^0\gamma_i\partial_j\partial_k=0$$
Es gibt simultane Eigenzustände zu Energie, Impuls, Helizität.

\paragraph{Interpretation der 4 Komponenten von $\psi$}\mbox{}\par 
Zu gegebenem Impuls $\vf{p}$: 4 linear unabhängige Lösungen:
\begin{itemize}
\item $E>0$, Helizität $\pm\frac{1}{2}$
\item $E<0$, Helizität $\pm\frac{1}{2}$
\end{itemize}

\subsection{Kopplung ans elektromagnetische Feld}

Freie Diracgleichung: $(i\gamma^\mu\partial_\mu -m)\psi = 0$\par 
Freie Klein-Gordon-Gleichung: $(-\partial_\mu\partial^\mu - m^2)\phi = 0$\par 
Relativistisches klassisches Teilchen: $L=\frac{1}{2}m\frac{\dif x^\mu}{\dif\tau}\frac{\dif x_\mu}{\dif\tau}$

Kopplung and e.m. Feld soll relativistisch invariant und eichinvariant sein. (Eichung $A^\mu (x)\mapsto A^\mu (x)+\partial^\mu\theta (x)$).\par 

Klassisches Teilchen:
$$L=\frac{1}{2}m\frac{\dif x^\mu}{\dif\tau}\frac{\dif x_\mu}{\dif\tau} - e\frac{\dif x_\mu}{\dif\tau}A^\mu (x)$$
(Einfachse denkbare relativistische WW, Wirkung ist eichinvariant, reproduziert Coulomb- und Lorentzkraft)\par 
Kanonisch konjugierter Impuls:
$$\mathcal{P}^\mu = \frac{\partial L}{\partial\frac{\dif x_\mu}{\dif\tau}}=m\frac{\dif x^\mu}{\dif\tau}-eA^\mu$$
$$\Rightarrow H=\frac{1}{2m}(\mathcal{P}^\mu + eA^\mu )^2$$
Rezept: minimale Kopplung $\mathcal{P}^\mu\rightarrow\mathcal{P}^\mu + eA^\mu$, Klein-Gordon-Gleichung:
$$\boxed{\left[\left( i\partial^\mu + eA^\mu\right)\left( i\partial_\mu +eA_\mu \right) -m^2\right]\phi = 0}$$
Dirac-Gleichung:
$$\boxed{\left(i\dslash + e\slashed{A}-m\right)\psi = 0}$$
Elektromagnetische Stromdichte:
$$j^\mu = e\overline{\psi}\gamma^\mu\psi$$
Eichinvarianz:
\begin{align*}
A^\mu (x) &\longrightarrow A^\mu (x) +\partial^\mu\theta (x) \\
\psi (x) &\longrightarrow e^{ie\theta (x)}\psi(x)
\end{align*}
Eichkovariante Ableitung:
$D^\mu\psi :=(\partial^\mu -ieA^\mu )\psi$. Damit gilt $D^\mu\psi \longrightarrow e^{ie\theta (x)} D^\mu\psi$


\subsection{Nichtrelativistischer Limes}

Nichtrelativistische Schrödingergleichung mit e.m. Feld:
$$(i\partial_t + e\Phi )\psi = \frac{(\hat{\vf{p}}+e\vf{A})^2}{2m}\psi$$
Klein-Gordon-Gleichung:
$$\left[\left( i\partial^\mu + eA^\mu\right)\left( i\partial_\mu +eA_\mu \right) -m^2\right]\phi = 0$$
$(A^\mu ) = (\Phi ,\vf{A})$, $(i\partial^j) = (-i\partial_j ) = (p^j)$.\par 
Ansatz: 
\begin{itemize}
\item $\phi$ ist Energie-EZ, $i\partial_t\phi = E\phi$
\item $E=m+\textit{klein}$, $E>0$
\item $e|A^\mu |\ll m$
\item $|\partial_tA^\mu |\ll |mA^\mu |$
\item $|p|\ll m$
\end{itemize}
Einsetzen in KG-Gl.:
$$\left[ (i\partial_t+e\Phi)(E+e\Phi )-(\hat{\vf{p}}+e\vf{A})^2-m^2\right]\phi = 0$$
Vernachlässigen von $\partial_t\Phi$:
$$\left[ (E+e\Phi )^2 - (\hat{\vf{p}}+e\vf{A})^2-m^2\right]\phi = 0$$
Mit $E+e\Phi = m + (E-m+e\Phi )$ mit Vernachlässigung des Quadrates der letzten Klammer:
$$\left[ 2m(E-m+e\Phi ) - (\hat{\vf{p}}+e\vf{A})^2 \right]\phi = 0$$
Daraus folgt direkt die nichtrelativistische Schrödingergleichung.

\paragraph{Diracgleichung mit e.m. Feld}
$$(i\slashed{D} - m)\psi = 0$$
Ansatz wie oben. Aufteilung des Diracspinors in zwei Paulispinoren:
$$\psi = \begin{pmatrix}
\psi_A \\ \psi_B
\end{pmatrix}$$
$$\begin{pmatrix}
iD_0 -m & iD_i\sigma^i \\
-iD_i\sigma^i & -iD_0 - m
\end{pmatrix}\begin{pmatrix}
\psi_A \\ \psi_B
\end{pmatrix}=0$$
Nach Ansatz: $iD_0\rightarrow E+e\Phi $, $iD_i\sigma^i = -\vec{\sigma}(\hat{\vf{p}}+e\vf{A})$.
\begin{align*}
(E-m+e\Phi )\psi_A - \vec{\sigma}(\hat{\vf{p}}+e\vf{A})\psi_B &= 0 \\
(-E-m-e\Phi )\psi_B + \vec{\sigma}(\hat{\vf{p}}+e\vf{A})\psi_A &= 0
\end{align*}
Eliminiere 
$$\psi_B = \frac{\vec{\sigma}(\hat{\vf{p}}+e\vf{A})}{E+m+e\Phi}\psi_A \cong \left(\frac{1}{2m}+\mathcal{O}(m^{-2})\right)\vec{\sigma}(\hat{\vf{p}}+e\vf{A})$$
$$\Rightarrow\quad (E-m+e\Phi )\psi_A = \frac{1}{2m}\left(\vec{\sigma}(\hat{\vf{p}}+e\vf{A})\right)\left(\vec{\sigma}(\hat{\vf{p}}+e\vf{A})\right)\psi_A$$
\paragraph{Vereinfachung der $\sigma$-Anteile}
$$(\vec{\sigma}\cdot\hat{\vf{O}})(\vec{\sigma}\cdot\hat{\vf{O}}) = \sigma^i\hat{O}^i\sigma^j\hat{O}^j=\sigma^i\sigma^j\hat{O}^i\hat{O}^j$$
$$=\left(\frac{1}{2}\left\{\sigma^i,\sigma^j\right\} + \frac{1}{2}\left[\sigma^i,\sigma^j\right]\right)\hat{O}^i\hat{O}^j = \left(\delta^{ij}+i\epsilon^{ijk}\sigma^k\right)\hat{O}^i\hat{O}^j$$
$$=\hat{\vf{O}}^2+i\epsilon^{ijk}\sigma^k\frac{1}{2}[\hat{O}^i,\hat{O}^j]$$
Hier: $\hat{\vf{O}}=(\hat{\vf{p}}+e\vf{A})$:
$$\cdots = (\hat{\vf{p}}+e\vf{A})^2 + i\epsilon^{ijk}\sigma^k(-i\partial_ieA^j)$$
$$=(\hat{\vf{p}}+e\vf{A})^2 + e\vf{B}\cdot\vec{\sigma}$$
$$\boxed{(E-m+e\Phi )\psi_A=\left[\frac{(\hat{\vf{p}}+e\vf{A})^2}{2m}+\frac{e}{2m}\vec{\sigma}\cdot\vf{B}\right]\psi_A}$$
Pauli-Gleichung enthält Term $\vf{S}\cdot\vf{B}$ ($\vf{S}=\vec{\sigma}/2$) mit Vorfaktor:
$$\boxed{g_s\frac{e}{2m}\vf{S}\cdot\vf{B}\qquad ,\qquad g_s=2}$$

\paragraph{Bedeutung des $g_s$-Terms} Allg. Hamiltonian für magnetischen Dipol $\vec{\mu}$ im $\vf{B}$-Feld:
$$H = -\vec{\mu}\cdot\vf{B}_{ext}$$
Vergleich mit Pauli-Gleichung liefert $\vec{\mu}_s = -g_s\frac{e}{2m}\vf{S}$ mit $g_s = 2$. Das ist ein intrinsisches magnetisches Dipolmoment, proportional zum Spin.\par 

Vergleich mit klassischer Elektrodynamik (rotierende Ladungsverteilung, Ladung $Q$, Masse $M$, Drehimpuls $\vf{L}$) liefert $\vec{\mu}=\frac{Q}{M}\vf{L}$ $\Rightarrow$ Klassisches Ergebnis entspricht $g=1$.

\paragraph{Interpretation des ersten Terms} (identisch in der nicht-relativistischen Schrödingergleichung)
$$\frac{(\hat{\vf{p}}+e\vf{A})^2}{2m}=\underbrace{\frac{\hat{\vf{p}}^2}{2m}}_{E_{kin}}+\underbrace{\frac{e}{2m}(\hat{\vf{p}}\vf{A}+\vf{A}\hat{\vf{p}})+\frac{e^2}{2m}\vf{A}^2}_{\text{e.m. WW}}$$
Bsp. homogenes $\vf{B}$-Feld: setze $\vf{A}(x)=-\frac{1}{2}(\vf{x}\times\vf{B})$, dann $\vf{B}=\nabla\times\vf{A}$.
$$\hat{\vf{p}}\vf{A}+\vf{A}\hat{\vf{p}}=\vf{B}\cdot\hat{\vf{L}}$$
$$\Rightarrow \text{Erster Term } = \frac{\hat{\vf{p}}^2}{2m} +\frac{e}{2m}\vf{B}\cdot\hat{\vf{L}}+\frac{e^2}{2m}\vf{A}^2$$

\subsection{Weitere Konsequenzen: Spin-Bahn-Kopplung}

Höhere Ordnungen im nicht-relativistischen Limes:
\begin{itemize}
\item Spin-Bahn-Kopplung $\sim\vf{L}\cdot\vf{S}$ (Feinstrukturaufspaltung)
\item Darwin-Term
\item Korrektur E-kin. 
\end{itemize}

Saubere Herleitung durch systematische Entwicklung in Potenzen von $m$. $\frac{1}{m}$ sei eine kleine Größe. \\ $\rightarrow$ Foldy-Wouthuysen-Transformation/-Bild.

$$(i\slashed{D}-m)\psi = 0$$
$$\Leftrightarrow i\partial_t\psi = (-e\Phi + m\gamma^0 - iD_i\gamma^0\gamma^i)\psi = H_D\psi$$

Idee: Unitäre Transformation / neues "Bild", Zerlegung in 2-Spinoren.
$$\psi = e^{-iS}\psi '=e^{-iS}\begin{pmatrix}
\psi_A' \\ \psi_B'
\end{pmatrix}$$
$S$ hermitesch, eventuell $t$-abhängig.\par 

Neuer Hamiltonian:
$$i\partial_t\psi' = i\partial_t (e^{iS}\psi ) = (i\partial_t e^{iS})\psi + e^{iS}i\partial_t\psi$$
$$=\left[ (i\partial_te^{iS})e^{-iS}+e^{iS}H_De^{-iS}\right]\psi '$$
$$H_D'=i(i\dot{S}+\frac{i^2}{2}[S,\dot{S}]+\frac{i^3}{6}[S,[S,\dot{S}]]+\ldots )+ H_D + i[S,H_D]+\frac{i^2}{2}[S,[S,H_D]]+\ldots$$

Idee 2: $H_D'$ soll blockdiagonal sein in 2-Spinoren (bis zu bestimmter Ordnung) $\rightarrow$ Gleichung für $\psi_A '$ reicht aus.\par 

Konkret: 
$$H_D=m\gamma^0 + (-e\Phi ) + \begin{pmatrix}
0 & (\vf{p}+e\vf{A})\cdot\vec{\sigma} \\
(\vf{p}+e\vf{A})\cdot\vec{\sigma} & 0
\end{pmatrix}= \underbrace{m\gamma^0}_{\mathcal{O}(m^1)} + \underbrace{\mathcal{E}}_{{\mathcal{O}(m^0)}} + \underbrace{\mathcal{O}}_{{\mathcal{O}(m^0)}}$$
Häufige Umformung: $\gamma^0 O =-O\gamma^0$ mit ungeradem Operator $O$.\par 

1. Schritt: arbeite bis $\mathcal{O}(m^0)$: Setze $S=\mathcal{O}(m^{-1})$
$$H_D'=H_D + i[S,H_D] + \mathcal{O}(m^{-1}) = m\gamma^0+\mathcal{E}+\mathcal{O}+i[S,m\gamma^0+\mathcal{E}+\mathcal{O}] + \mathcal{O}(m^{-1})$$
$$=m\gamma^0 +\mathcal{E}+\mathcal{O} + i[S,m\gamma^0]$$
Lösung: $S=-\frac{i}{2m}\gamma^0\mathcal{O}$\par 
Damit $H_D'$ komplett ausrechnen bis $\mathcal{O}(m^{-2})$:
$$H_D'=H_D+i[S,H_D]-\dot{S}+\frac{i^2}{2}[S,[S,H_D]]-\frac{i}{2}[S,\dot{S}] + \frac{i^3}{6}[S,[S,[S,H_D]]]+\mathcal{O}(m^{-3})$$

Für die einzelnen Terme finden Wirkung
\begin{align*}
    i [S, H_D] &= i \left[-\frac{i}{2m} \gamma^0 \mathcal{O}, m \gamma^0 + \mathcal{E} + \mathcal{O} \right] = -\mathcal{O} + \frac{1}{2m} \gamma^0 [\mathcal{O}, \mathcal{E}] + \frac{1}{m} \gamma^0 \mathcal{O}^2 \\
    - \dot{S} &= \frac{i}{2m} \gamma^0 \dot{\mathcal{O}} \\
    \frac{i}{2} [S, \dot{S}] &= -\frac{i}{8 m^2} [\mathcal{O}, \dot{\mathcal{O}}] \\
    \frac{i^2}{2} [S, [S, H_D]] &= -\frac{1}{2m} \gamma^0 \mathcal{O}^2 - \frac{1}{8 m^2} [\mathcal{O}, [\mathcal{O}, \mathcal{E}]] - \frac{1}{2m^2} \mathcal{O}^3 \\
    \frac{i^3}{3!} [S, [S, [S, H_D]]] &= \frac{1}{6m^2} \mathcal{O}^3
\end{align*}
Der neue Hamiltonian ist nun
\begin{align*}
    H_D' &=\underbrace{m \gamma^0 + \mathcal{E} + \frac{1}{2m} \gamma^0 \mathcal{O}^2 - \frac{1}{8 m^2} [\mathcal{O}, i \dot{\mathcal{O}} + [\mathcal{E}, \mathcal{O}]]}_{\text{gerade} \;=: H_{D,\text{even}}'} + \\
    &= \underbrace{\frac{1}{2m} \gamma^0 (i \dot{\mathcal{O}} + [\mathcal{O}, \mathcal{E}]) - \frac{1}{6m^2} \mathcal{O}^3}_{\text{ungerade} \; =: \mathcal{O}'} \\
    &=: H_{D,\text{even}}' + \mathcal{O}'
\end{align*}

2. Schritt: arbeite bis $\mathcal{O}(m^-1)$: 

In Analogie setzen wir $\psi' = e^{i S'} \psi''$ mit $S' = -\frac{i}{2m} \gamma^0 \mathcal{O}'$ und erhalten
$$H_D'' = H_{D,\text{even}}' + i [S', \mathcal{E}] - \dot{S}' + \mathcal{O}(m^{-3}) := D_{D,\text{even}} + \mathcal{O}''$$

3. Schritt: arbeite bis $\mathcal{O}(m^{-2})$:

Wir setzen wieder $\psi'' = e^{i-i S''} \psi'''$ mit $S'' = -\frac{i}{2m} \gamma^0 \mathcal{O}'' = \mathcal{O}(m^{-3})$.

HIER FEHLT NOCH DIE GLEICHUNG FÜR $H_D'''$

Vollständig ausgerechnet:
$$H_D''' = \underbrace{m\gamma^0 + \mathcal{E} + \frac{1}{2m} \gamma^0 \mathcal{O}^2}_{\mathcal{O}(m^{-1})} - \underbrace{\frac{1}{8m^2} [\mathcal{O}, i \dot{\mathcal{O}} + [\mathcal{O}, \mathcal{E}]]}_{\mathcal{O}(m^{-2})}$$

\begin{itemize}
    \item Terme bis $\mathcal{O}(m^{-1})$ liefern genau den Limes aus 1.4.4 inkl. des $g-2$-Terms
    \item Zusätzliche Terme der relativistischen Korrektur bis $\mathcal{O}(m^{-2})$
\end{itemize}

Wir diskutieren diese Terme anhand des Zentralpotentials mit $\vf{A} = 0$ und $\Psi(\vf{x}, t) = \Psi(r)$ mit $r = |\vf{x}|$.
Es ergeben sich die Terme
\begin{align*}
    \nabla \Psi (r) &= \frac{\vf{x}}{r} \frac{\mathrm{d} \Psi}{\mathrm{d} r} \\
    \vf{E} &= - \nabla \Psi \\
    \mathcal{E} &= e \Psi \\
    \mathcal{O} &= \begin{pmatrix}
        0 & \vec{\sigma} \cdot\vf{p} \\ \vec{\sigma}\cdot\vf{p} & 0 \\
    \end{pmatrix} = -i \begin{pmatrix}
        0 & \vec{\sigma}\cdot\nabla \\ \vec{\sigma}\cdot \nabla & 0 \\
    \end{pmatrix} \\
    [\mathcal{O}, \mathcal{E}] &= -i e \begin{pmatrix}
        0 & \vec{\sigma}\cdot \vf{E} \\ \vec{\sigma}\cdot \vf{E} & 0 \\
    \end{pmatrix} \\
    [\mathcal{O}, [\mathcal{O}, \mathcal{E}]] &= (-i) (-i e) \begin{pmatrix}
        [\vec{\sigma}\cdot \nabla, \vec{\sigma}\cdot \vf{E}] & 0 \\ 0 & [\vec{\sigma}\cdot \nabla, \vec{\sigma}\cdot \vf{E}] \\
    \end{pmatrix}  \\
    [\vec{\sigma}\cdot \nabla, \vec{\sigma}\cdot \vf{E}]  &= \sigma^i \sigma^j (\partial_i E^j + E^j \partial_i) - \sigma^j \sigma^i E^j \partial_i \\
    &= \nabla\cdot \vf{E} + \underbrace{i \vec{\sigma}\cdot (\nabla \times \vf{E})}_{= 0} 
    + \underbrace{i^2 \epsilon^{i j k} \sigma^k E^j \partial_i}_{=2 \vec{\sigma}\cdot (\vf{E} \times \vec{p})} \\
    &= \nabla\cdot\vf{E} - \frac{2}{r} \frac{\mathrm{d}\Psi}{\mathrm{d}r} \vec{\sigma}\cdot \vf{L}
\end{align*}
Wir finden den nun bis zum $\mathcal{O}(m^{-2})$ Term blockdiagonalen Hamiltonian
$$H_D''' = \frac{e}{8 m^2} \nabla\cdot\vf{E} - \frac{e}{2 m^2 r} \frac{\mathrm{d} \Psi}{\mathrm{d}r} \vf{S}\cdot \vf{L}$$
Der obere Block ist 
\begin{align*}
    H_{\text{eff}} &= m + H_{\mathcal{O}(m^{-1})} + H_{\mathcal{O}(m^{-2})} + \ldots \\
    H_{\mathcal{O}(m^{-1})} &= H_{\text{Pauli}} = -e \Psi + \frac{(\vf{p} + e \vf{A})^2}{2 m} + \frac{e}{2m}\vec{\sigma}\cdot \vf{B} \\
    H_{\mathcal{O}(m^{-2})} &= \underbrace{\frac{e}{8m^2}\nabla\cdot \vf{E}}_{\text{Darwin-Term}} - \underbrace{\frac{e}{2m^2 r} \frac{\mathrm{d}\Psi}{\mathrm{d}r} \vf{S}\cdot \vf{L}}_{\text{Spin-Bahn-Kopplung}}
\end{align*}
Diskussion:
\begin{itemize}
    \item Darwin-Term: beim Atom $\nabla\cdot\vf{E} = 4\pi \rho_{\text{Kern}} \propto \delta^{(3)}(\vf{x})$ ergibt sich eine Korrektur für die s-Orbitale, die am Kern eine endliche Aufenthaltswahrscheinlichkeit haben
    \item Spin-Bahn-Koppluns: Wegen dieses Terms $[H_{\text{eff}}, \vf{S}] \neq 0$ und $[H_{\text{eff}}, \vf{L}] \neq 0$, aber $[H_{\text{eff}}, \vf{J}] = 0$.
\end{itemize}



\end{document}































